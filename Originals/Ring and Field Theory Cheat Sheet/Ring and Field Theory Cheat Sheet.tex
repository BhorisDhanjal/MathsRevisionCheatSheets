\documentclass[dvipsnames]{article}
\usepackage{geometry}
\geometry{a3paper, landscape, margin=2in}
\usepackage{url}
\usepackage{multicol}
\usepackage{amsmath}
\usepackage{esint}
\usepackage{amsfonts}
\usepackage{tikz}
\usetikzlibrary{decorations.pathmorphing}
\usepackage{amsmath,amssymb}
\usepackage{enumitem}
\usepackage{colortbl}
\usepackage{mathtools}
\usepackage{amsmath,amssymb}
\usepackage{enumitem}
\usepackage{ dsfont }
\makeatletter

\newcommand*\bigcdot{\mathpalette\bigcdot@{.5}}
\newcommand*\bigcdot@[2]{\mathbin{\vcenter{\hbox{\scalebox{#2}{$\m@th#1\bullet$}}}}}
\makeatother
%Sheet made by Boris, Template is by Drew Ulick https://de.overleaf.com/articles/130-cheat-sheet/ntwtkmpxmgrp
\title{Intro Ring and Field Theory Cheat Sheet}
\usepackage[english]{babel}
\usepackage{palatino}
\usepackage{bm}

\advance\topmargin-1.2in
\advance\textheight3in
\advance\textwidth3in
\advance\oddsidemargin-1.5in
\advance\evensidemargin-1.5in
\parindent0pt
\parskip2pt
\newcommand{\hr}{\centerline{\rule{3.5in}{1pt}}}
%\colorbox[HTML]{e4e4e4}{\makebox[\textwidth-2\fboxsep][l]{texto}
\begin{document}

\begin{center}{\fontsize{35}{60}{\textcolor{teal}{\textbf{Intro Ring and Field Theory Cheat Sheet}}}}\\
\end{center}
\begin{multicols*}{3}

\tikzstyle{mybox} = [draw=teal, fill=white, very thick,
    rectangle, rounded corners, inner sep=10pt, inner ysep=10pt]
\tikzstyle{fancytitle} =[fill=teal, text=white, font=\bfseries]

%------------ Ring and Field Axioms Definitions ---------------
\begin{tikzpicture}
\node [mybox] (box){%
    \begin{minipage}{0.3\textwidth}
    A ring $R$ is a set with two binary operations $+$ and $\times$ satisfying the following axioms:\\
    \textbf{i. $\bm{(R, +)}$ is an abelian group.}\\
    \textbf{ii. Multiplicative associativity:} $(a\times b)\times c = a \times (b \times c)\ \forall a, b, c \in R.$\\
    \textbf{iii. Left and right distributivity:}\\
    $(a+b)\times c=(a \times c)+(b \times c)$ and $a \times (b+c) = (a \times b)+(a \times c).$\\
    In addition to these rings may also have the following optional properties.\\
    \textbf{a. Multiplicative commutativity:} $a \times b = b \times a,\ \forall\ a, b \in R.$\\
    \textbf{b. Multiplicative Identity:} $\exists\ 1 \in R$ s.t. $\forall a \neq 0 \in R, 1 \times a = a \times 1 = a.$\\
    \textbf{c. Multiplicative Inverse:} $\forall\ a\neq 0 \in R\ \exists\ a^{-1} \in R$ s.t. $a \times a^{-1}=a^{-1} \times a = 1.$\\
    \textbf{\textit{FOR THE PURPOSE OF THIS SHEET WE LOOK AT RINGS WITH MULTIPLICATIVE COMMUTATIVITY AND 1$\neq$0.}}\\
    A field $F$ is a set with two binary operations $+$ and $\times$ satisfying the following axioms:\\
    \textbf{i. $\bm{(F, +)}$ is an abelian group with identity 0.}\\
    \textbf{ii. The non-zero elements of $\bm{F}$ form a abelian group under multiplication with identity 1.}\\
    \textbf{iii. Left and right distributivity.}
    \end{minipage}
};
%------------ Ring and Field Axioms Header ---------------------
\node[fancytitle, right=10pt] at (box.north west) {Ring and Field Axioms};
\end{tikzpicture}

%------------ Polynomial Rings  Definitions ---------------
\begin{tikzpicture}
\node [mybox] (box){%
    \begin{minipage}{0.3\textwidth}
    For a ring $R$, $R[x]$ denotes the polynomial ring of a single variable $x$ s.t.the elements of $R[x]$ are of the form\\
    $a_n x^n + a_{n-1} x^{n-1}+ \dots + a_1 x + a_0$ with $n\geq 0$ and $a_i \in R$\\ 
    Polynomial rings can be generalized for multiple variables.
    \end{minipage}
};
%------------ Polynomial Rings Header ---------------------
\node[fancytitle, right=10pt] at (box.north west) {Polynomial Rings};
\end{tikzpicture}

%------------ Zero Divisors, Units and Integral Domains Definitions ---------------
\begin{tikzpicture}
\node [mybox] (box){%
    \begin{minipage}{0.3\textwidth}
    \textbf{i. Zero Divisor:} $a \neq 0 \in R$ is called a zero divisor of $R$ if  $\exists\ b\neq0 \in\ R$ s.t. either $ab=0$ or $ba=0.$\\
    \textbf{ii. Unit:} For a ring $R$ with identity $1\neq 0$, $u \in R$ is called a unit in $R$ if $\exists v \in R$ s.t. $uv=vu=1$.\\
    \textbf{iii. Integral Domain:} A commutative ring with identity $1\neq 0$ is called an integral domain if it has no zero divisors.
    \begin{itemize}[noitemsep,topsep=0pt]
        \item Any finite integral domain is a field.
        \item If $R$ is an integral domain than the polynomial ring of one variable over R, i.e. $R[x],$ is also a integral domain.
    \end{itemize}
    
    \end{minipage}
};
%------------ Zero Divisors, Units and Integral Domains Header ---------------------
\node[fancytitle, right=10pt] at (box.north west) {Zero Divisors, Units and Integral Domains};
\end{tikzpicture}

%------------ Subrings Definitions ---------------
\begin{tikzpicture}
\node [mybox] (box){%
    \begin{minipage}{0.3\textwidth}
    A subring of the ring $R$ is defined as a subgroup of $R$ that is closed under multiplication.
    \end{minipage}
};
%------------ Surbings Header ---------------------
\node[fancytitle, right=10pt] at (box.north west) {Subrings};
\end{tikzpicture}

%------------ Ring Homomorphisms, Isomorphisms and Kernels Definitions ---------------
\begin{tikzpicture}
\node [mybox] (box){%
    \begin{minipage}{0.3\textwidth}
    For rings $R$ and $S$.\\
    \textbf{i. Ring Homomorphism} is a map $\varphi:R\rightarrow S$ satisfying:
    \begin{itemize}[noitemsep,topsep=0pt]
        \item $\varphi(a+b)=\varphi(a)+\varphi(b)\ \forall\ a,b \in R$
        \item $\varphi(ab)=\varphi(a)\varphi(b)\ \forall\ a,b \in R$
    \end{itemize}
    \textbf{ii. Isomorphism} is a bijective ring homomorphism. \\
    \textbf{iii. Kernel} of the ring homomorphism $\varphi$ is the set of elements of $R$ that map to $0$ in $S$.
    \begin{itemize}[noitemsep,topsep=0pt]
        \item The image of $\varphi$ is a subring of S.
        \item The kernel of $\varphi$ is a subring of S. \emph{(For Rings without 1)}
    \end{itemize}
    \end{minipage}
};
%------------ Ring Homomorphisms, Isomorphisms and Kernels Header ---------------------
\node[fancytitle, right=10pt] at (box.north west) {Ring Homomorphisms, Isomorphisms and Kernels};
\end{tikzpicture}

%------------ Ideals Definitions ---------------
\begin{tikzpicture}
\node [mybox] (box){%
    \begin{minipage}{0.3\textwidth}
    \textbf{Ideal:} A subset $I$ of ring $R$ is called an ideal of $R$ if \begin{itemize}[noitemsep,topsep=0pt]
        \item It is a subring of $R$.
        \item It is closed under both left and right multiplication with elements from $R$.
    \end{itemize}
    \textit{Ideals are to rings what normal subgroups are to groups.}
    \end{minipage}
};
%------------ Ideals Header ---------------------
\node[fancytitle, right=10pt] at (box.north west) {Ideals};
\end{tikzpicture}

%------------ Quotient Rings Definitions ---------------
\begin{tikzpicture}
\node [mybox] (box){%
    \begin{minipage}{0.3\textwidth}
    Let $R$ be a ring with ideal $I$. $R/I$ is called a quotient ring if\\
    \textbf{i.} $(r+I)+(s+I)=(r+s)+I$\\
    \textbf{ii.} $(r+I) \times (s+I)=(rs)+I$
    \end{minipage}
};
%------------ Quotient Rings Header ---------------------
\node[fancytitle, right=10pt] at (box.north west) {Quotient Rings};
\end{tikzpicture}

%------------ First Isomorphism and Correspondence Theorem  Definitions ---------------
\begin{tikzpicture}
\node [mybox] (box){%
    \begin{minipage}{0.3\textwidth}
    \textbf{i. First Isomorphism Theorem:} Let $\varphi: R \rightarrow S$ be a ring homomorphism from ring $R$ to $S$ then:
    \begin{itemize}[noitemsep,topsep=0pt]
        \item Kernel of $\varphi$ is an ideal of $R$,
        \item Image of $\varphi$ is a subring of $S$ and,
        \item $R/$ ker $\varphi \cong \varphi(R)$.
    \end{itemize}
    \textbf{ii. Correspondence Theorem:} Let $R$ be a ring, and $I$ be an ideal of $R$. \\
    The correspondence $A\leftrightarrow A/I$ is an inclusion preserving bijection between the set of subrings $A$ of $R$ that contain $I$ and the set of subrings of $R/I$.\\
    \textit{or}\\
    There exists an inclusion preserving biijection between ideals in $R$ containing ker$(\varphi)$ and ideals in $\varphi(R).$
    \end{minipage}
};
%------------ First Isomorphism and Correspondence Theorem Header ---------------------
\node[fancytitle, right=10pt] at (box.north west) {First Isomorphism and Correspondence Theorem};
\end{tikzpicture}

%------------ Principal, Prime and Maximal Ideals  Definitions ---------------
\begin{tikzpicture}
\node [mybox] (box){%
    \begin{minipage}{0.3\textwidth}
    \textbf{i. Principal Ideals:} An ideal generated by a single element is called a principal ideal.\\
    \textbf{ii. Prime Ideals:} If $P\neq R$, then an ideal $P$ is called a prime ideal if  $ab \in P, \text{when } a,b \in R$ then at least one of $a$ and $b$ in an element of $P$. \textit{This is analogous to the definition of prime numbers in number theory}\\
    \textbf{iii. Maximal Ideals:} If $M \neq R$, then an ideal $M$ is called a maximal ideal if the only ideals containing $M$ are $M$ and $R$ itself.\\
    \textit{$\bullet$ Every maximal ideal of $R$ is a prime ideal.}\\
    \textit{$\bullet$ The ideal $P$ is a prime ideal in $R$ iff $R/P$ is an integral domain.}
    \end{minipage}
};
%------------ Principal, Prime and Maximal Ideals Header ---------------------
\node[fancytitle, right=10pt] at (box.north west) {Principal, Prime and Maximal Ideals};
\end{tikzpicture}

%------------ Zorn's Lemma Definitions ---------------
\begin{tikzpicture}
\node [mybox] (box){%
    \begin{minipage}{0.3\textwidth}
 If $S$ is any nonempty partially ordered set in which every chain has an upper bound, then $S$ has a maximal element. 
    \end{minipage}
};
%------------ Zorn's Lemma Header ---------------------
\node[fancytitle, right=10pt] at (box.north west) {Zorn's Lemma};
\end{tikzpicture}

%------------ Ring of Fractions of an Integral Domain Definitions ---------------
\begin{tikzpicture}
\node [mybox] (box){%
    \begin{minipage}{0.3\textwidth}
    Let $R$ be an integral domain. Let $K$ be the ring of fractions of $R$ s.t. \\
    $K=\{\frac{a}{b}| a, b\ \in\ R, b\neq0\}$. $K$ is also called a field of fractions since it always forms a field for any ring $R$.
    \begin{itemize}[noitemsep,topsep=0pt]
        \item $\frac{a}{b}+\frac{c}{d}=\frac{ad+bc}{bd}$, $b,d\ \neq 0$
        \item $\frac{a}{b}\frac{c}{d}=\frac{ac}{bd}$, $b,d\ \neq 0$
    \end{itemize}
    \end{minipage}
};
%------------ Ring of Fractions of an Integral Domain Header ---------------------
\node[fancytitle, right=10pt] at (box.north west) {Ring of Fractions of an Integral Domain};
\end{tikzpicture}

%------------ Chinese Remainder Theorem  Definitions ---------------
\begin{tikzpicture}
\node [mybox] (box){%
    \begin{minipage}{0.3\textwidth}
    The ideals $I$ and $J$ of a ring $R$ are said to be \textbf{comaximal} if $I+J=R.$\\
    \textbf{Chinese Remainder Theorem:} $\forall\ a, b\ \in R,\ \exists\ x\ \in\ R$ s.t. \\$x\equiv a (\text{mod } I)$ and $x\equiv b(\text{mod } J)$
    \end{minipage}
};
%------------ Chinese Remainder Theorem Header ---------------------
\node[fancytitle, right=10pt] at (box.north west) {Chinese Remainder Theorem};
\end{tikzpicture}

%------------ Noetherian Rings  Definitions ---------------
\begin{tikzpicture}
\node [mybox] (box){%
    \begin{minipage}{0.3\textwidth}
   A commutative ring $R$ is called \textbf{Noetherian} if there is no infinite increasing chain of ideals in $R$, i.e. when $I_1 \subseteq I_2 \subseteq I_3 \dots$ is an ascending chain of ideals $\exists\ k\in \mathds{Z}^+$ s.t. $I_k=I_m\ \forall k\geq m.$\\
   It is equivalent to say that $R$ is Noetherian if every ideal of $R$ is finitely generated.
    \end{minipage}
};
%------------ Noetherian Rings Header ---------------------
\node[fancytitle, right=10pt] at (box.north west) {Noetherian Rings};
\end{tikzpicture}

%------------ Hilbert Basis Theorem  Definitions ---------------
\begin{tikzpicture}
\node [mybox] (box){%
    \begin{minipage}{0.3\textwidth}
    If $R$ is a noetherian ring then so is the polynomial ring $R[x]$.\\
    \textit{$R[x_1, x_2, x_3, \dots, x_n]$ for finite $n$ is also noetherian.}
    \end{minipage}
};
%------------ Hilbert Basis Theorem Header ---------------------
\node[fancytitle, right=10pt] at (box.north west) {Hilbert Basis Theorem};
\end{tikzpicture}

%------------ Irreducible and Prime Elements  Definitions ---------------
\begin{tikzpicture}
\node [mybox] (box){%
    \begin{minipage}{0.3\textwidth}
    \textbf{i. Irreducible Element} An element $a$ of ring $R$ is called \textbf{irreducible} if it is  non-zero, not a unit and, \emph{only has trivial divisors (i.e. units and products of units).}\\
    \textbf{ii. Prime Element} An element $a$ of ring $R$ is called \textbf{prime} if it is non-zero, not a unit and, if $a \mid bc$ then either $a \mid b$ or $a \mid c$ for some $b, c\ \in\ R.$\\
   \emph{The concept of primes and irreducible is the same in integers, but they are distinct in general.\\
    In an integral domain, every prime element is irreducible, but the converse holds only in UFDs.}
    \end{minipage}
};
%------------ Irreducible and Prime Elements Header ---------------------
\node[fancytitle, right=10pt] at (box.north west) {Irreducible and Prime Elements};
\end{tikzpicture}

%------------ Norm and Euclidean Domains Definitions ---------------
\begin{tikzpicture}
\node [mybox] (box){%
    \begin{minipage}{0.3\textwidth}
    \textbf{i. Norm:} For a integral domain $R$, any function $N:R\rightarrow \mathds{Z}^+ \cup {0}$ with $N(0)=0$ is called a $norm$ on $R$.\\
    \textbf{ii. Euclidean Domain:} An integral domain $R$ is called an \textbf{Euclidean Domain} if there is a norm $N$ on $R$ s.t. for any two elements $a, b\ \in\ R$, where $b\neq 0\ \exists\ q, r\ \in\ R$ s.t. $\mathbf{a=qb+r\ where\ r=0\ or\ N(r)<N(b).}$\\
    $\bullet$ Any field $F$ is a trivial example of a Euclidean Domain.
    \end{minipage}
};
%------------ Norm and Euclidean Domain Header ---------------------
\node[fancytitle, right=10pt] at (box.north west) {Norm and Euclidean Domain};
\end{tikzpicture}

%------------ Principal Ideal Domains (PIDs)  Definitions ---------------
\begin{tikzpicture}
\node [mybox] (box){%
    \begin{minipage}{0.3\textwidth}
    A \textbf{Principal Ideal Domain (PID)} is an integral domain in which every ideal is principal.\\
    \emph{Every Euclidean Domain is a PID.}\\
    \textbf{Examples:}\\
    $\bullet \mathds{Z}$ is a PID, but $\mathds{Z}[x]$ is not.\\
    $\bullet F[x]$ if $F$ is a field, $\bullet Z[i]$
    \end{minipage}
};
%------------ Principal Ideal Domains (PIDs) Header ---------------------
\node[fancytitle, right=10pt] at (box.north west) {Principal Ideal Domains (PIDs)};
\end{tikzpicture}


%------------ Unique Factorisation Domains (UFDs)  Definitions ---------------
\begin{tikzpicture}
\node [mybox] (box){%
    \begin{minipage}{0.3\textwidth}
    Two elements $a, b\ \in\ R$ are said to be \textbf{associates} in $R$ if they differ by a unit, i.e. $a=ub$ for some unit $u\ \in\ R$.
    A \textbf{Unique Factorisation Domain (UFD)} is an integral domain $R$ in which every nonzero element $r\ \in\ R$ which is not a unit follows the properties:\\
    \textbf{i.} $r$ can we written as a finite product of irreducibles $p_i$ of $R$.\\
    \textbf{ii.} This decomposition is unique up to associates, i.e. if $r=p_1 p_2 \dots p_n$ and $r=q_1 q_2 \dots q_n$ then $m=n$ and for some renumbering of factors there is $p_i$ associate to $q_i$\\
    The above definition can be equivalently stated as:\\
    \emph{A UID is any integral domain in which every non-zero, non-invertible element has a unique factorisation.}\\
    $\bullet$\textbf{Every PID is a UID.}\\
    $\bullet$ $Z[x]$ is a UID, but not a PID.\\
    $\bullet$In a UID every non-zero element is a prime iff it is irreducible.\\
    $\bullet$ Fields $\subset$ Euclidean Domains $\subset$ PIDs $\subset$ UFDs $\subset$ Integral Domains.
    \end{minipage}
};
%------------ Unique Factorisation Domains (UFDs) Header ---------------------
\node[fancytitle, right=10pt] at (box.north west) {Unique Factorisation Domains (UFDs)};
\end{tikzpicture}

%------------ Primitive Polynomials and Gauss' Lemma  Definitions ---------------
\begin{tikzpicture}
\node [mybox] (box){%
    \begin{minipage}{0.3\textwidth}
    A polynomial $f(x) \in \mathds{Z}[x]$ is called \textbf{primitive} if $n=deg(f)>0$, $a_n>0$ and, $gcd(a_0,a_1,\dots,a_n)=1$ for $a_i \in \mathds{Z}$\\
    \textbf{Gauss' Lemma:} If $f(x), g(x) \in \mathds{Z}$ are primitive $\implies$ $fg$ is also primitive.
    \end{minipage}
};

%------------ Primitive Polynomials and Gauss' Lemma Header ---------------------
\node[fancytitle, right=10pt] at (box.north west) {Primitive Polynomials and Gauss' Lemma};
\end{tikzpicture}


%------------ Eisenstein's Criterion  Definitions ---------------
\begin{tikzpicture}
\node [mybox] (box){%
    \begin{minipage}{0.3\textwidth}
    The Eisenstein's Criterion is a test for irreducibility of polynomials.\\
    Let $P$ be a prime ideal of the integral domain $R$ amd, $f(x)=x^n+a_{n-1}x^{n-1}+\dots+a_1x+a_0$ be a polynomial in $R[x]$.\\
    \textbf{Eisenstein's Criterion} states that $f(x)$ is irreducible in R[x] if\\
    $\bullet\ a_{n-1},\dots a_1, a_0$ are elements of $P$ and,\\
    $\bullet\ a_0$ is \textbf{not} an element of $P^2.$\\
    \emph{If Eisenstein's Criterion doesn't directly apply to f(x) try on f(x+1), if f(x+1) is irreducible it implies f(x) is also irreducible.}
    \end{minipage}
};

%------------ Eisenstein's Criterion Header ---------------------
\node[fancytitle, right=10pt] at (box.north west) {Eisenstein's Criterion};
\end{tikzpicture}


%----------------------------Next Page-----------------------------

%------------ Characteristics of Fields  Definitions ---------------
\begin{tikzpicture}
\node [mybox] (box){%
    \begin{minipage}{0.3\textwidth}
    Let $1_F$ denote the identity of $F$.\\
    The \textbf{characteristic} of a field $F$, denoted as $ch(F)$ is defined as the smallest integer $p$ such that $p \cdot 1_f=0$ if such a $p$ exists and is defined as $0$ otherwise.\\
    $\bullet$ $ch(F)$ is either $0$ or a prime $p$, $\bullet$ $\mathds{Q}$ and $\mathds{R}$ have characteristic 0\\
    $\bullet\ F_p=\mathds{Z}/p\mathds{Z}$ has characteristic $p$,  
    \end{minipage}
};
%------------ Characteristics of Fields Header ---------------------
\node[fancytitle, right=10pt] at (box.north west) {Characteristics of Fields};
\end{tikzpicture}

%------------ Field Extensions and Degree  Definitions ---------------
\begin{tikzpicture}
\node [mybox] (box){%
    \begin{minipage}{0.3\textwidth}
    If $K$ is a field containing the subfield $F$, then $K$ is said to be an \textbf{extension field} of $F$. It is denoted as $K/F$.\\
    The \textbf{degree} of a field extension $K/F$ denoted by $[K:F]$ is the dimension of $K$ as a vector space over $F$.
    
    \end{minipage}
};
%------------ Field Extensions and Degree Header ---------------------
\node[fancytitle, right=10pt] at (box.north west) {Field Extensions and Degree};
\end{tikzpicture}

%------------ Irreducible Polynomials in Fields  Definitions ---------------
\begin{tikzpicture}
\node [mybox] (box){%
    \begin{minipage}{0.3\textwidth}
    $\bullet$ For a irreducible polynomial $p(x)\in F$, there exists a field $K$ containing a isomorphic copy of $F$ in which $p(x)$ has a root, i.e. there exists a field extension $K$ of $F$ in which $p(x)$ has a root. A simple way to find this extension is to consider the quotient $K=F[x]/(p(x))$.\\
    $\bullet$ For the above case, let $\theta=x\mod(p(x)) \in K$. Then the elements $1, \theta, \theta^2\dots \theta^{n-1}$ are a basis for $K$ as a vector space over $F$, with $[K:F]=n$.\\
    $\bullet$ For the above case, let $\alpha$ be the root of $p(x)$ s.t. $p(\alpha)=0$. Then, $F(\alpha)\cong F[x]/(p(x))$.
    \end{minipage}
};
%------------ Irreducible Polynomials in Fields Header ---------------------
\node[fancytitle, right=10pt] at (box.north west) {Irreducible Polynomials in Fields};
\end{tikzpicture}

%------------ Algebraic and Transcendental Elements Definitions ---------------
\begin{tikzpicture}
\node [mybox] (box){%
    \begin{minipage}{0.3\textwidth}
    \textbf{i. Algebraic Element:} If $K$ is a field extension over $F$, then $\alpha \in K$ is called \textbf{algebraic} over $F$, if there exists some non-zero polynomial $f(x)$ with coefficients, in $F$, s.t. $f(\alpha)=0.$\\
    \textbf{ii. Transcendental Element:} Elements $\alpha \in K$ which are not algebraic over $F$ are called \textbf{transcendental}.\\
    $\bullet$ If $\alpha$ is algebraic over $F$, then $F[\alpha]=F(\alpha)$, if $\alpha$ is transcendental over $F$, then $F[\alpha]\neq F(\alpha).$
    \end{minipage}
};
%------------ Algebraic and Transcendental Elements Header ---------------------
\node[fancytitle, right=10pt] at (box.north west) {Algebraic and Transcendental Elements};
\end{tikzpicture}

%------------ Algebraic Extensions Definitions ---------------
\begin{tikzpicture}
\node [mybox] (box){%
    \begin{minipage}{0.3\textwidth}
    $\bullet$ Let $\alpha$ be algebraic over $F$. There there exists a unique monic irreducible polynomial $m_{\alpha, F}(x) \in F[x]$ which has $\alpha$ as a root.\\
    $\bullet$ If $L/F$ is an extension of fields and $\alpha$ is algebraic over both $F$ and $L$ then $m_{\alpha, L}(x)$ divides $m_{\alpha, F}(x)$ in $L[x]$.\\
    $\bullet$ If $F(\alpha)$ is the field generated by $\alpha$ over $F$ then, $F(\alpha)\cong F[x]/(m_\alpha (x))$.\\
    $\bullet$ Let $F \subseteq K \subseteq L$ be fields. Then $[L:F]=[L:K][K:F]$
    $\bullet$ Similarly, $[K:F]$ divides $[L:F]$.\\
    $\bullet$ Let $K_1, K_2$ be two finite extensions of field $F$ contained in $K$. Then, $[K_1 K_2:F] \leq [K_1:F][K_2:F]$, but if $[K_1:F]=n, [K_2:F]=m$ and if gcd $(m,n)=1$. Then, $[K_1 K_2: F] = [K_1:F][K_2:F]=nm.$
    \end{minipage}
};
%------------ Algebraic Extensions Header ---------------------
\node[fancytitle, right=10pt] at (box.north west) {Algebraic Extensions};
\end{tikzpicture}

%------------ Splitting Fields Definitions ---------------
\begin{tikzpicture}
\node [mybox] (box){%
    \begin{minipage}{0.3\textwidth}
    \textbf{Splitting Fields:} The extension field $K$ of $F$ is called a splitting field for the polynomial $f(x) \in F[x]$ if $f(x)$ factors completely into linear factors in $K[x]$ but not over any proper subfield of $K$ containing $F$.\\
    $\bullet$ For any field $F$, if $f(x) \in F[x]$. Then, there exists an extension $K$ of $F$ which is a splitting field for $f(x)$.\\
    $\bullet$ A splitting field of a polynomial of degree $n$ over $F$ is of degree at most $n!$ over $F$.\\
    $\bullet$ Any two splitting fields for a polynomial $f(x) \in F[x]$ over a field $F$ are isomorphic.
    $\bullet$ The polynomial $x^n-1$ over $\mathds{Q}$ has in general a splitting field contained in $\mathds{C}$.\\
    $\bullet$ Let $\mathds{Q}(\zeta_n)$ be the cyclotomic field of $n^{th}$ roots of unity. $[\mathds{Q}\zeta_n:\mathds{Q}]=\varphi(n)$ where $\varphi(n)$ is Euler's totient function.
    \end{minipage} 
};
%------------ Splitting Fields Header ---------------------
\node[fancytitle, right=10pt] at (box.north west) {Splitting Fields};
\end{tikzpicture}

%------------ Algebraic Closure of Fields Definitions ---------------
\begin{tikzpicture}
\node [mybox] (box){%
    \begin{minipage}{0.3\textwidth}
    $\bullet$ The field $\Bar{F}$ is called an \textbf{algebraic closure} of $F$ if $\Bar{F}$ is algebraic over $F$ and, if every polynomial $f(x) \in F[x]$ splits completely over $\Bar{F}$.\\
    $\bullet$ A field $K$ is said to be \textbf{algebraically closed} if every polynomial with coefficients in $K$ has a root n $K$. $\Bar{F}$ as defined above is algebraically closed.\\
    $\bullet$ For every field $F$ there exists an algebraically closed field $K$ containing $F$.
    \end{minipage} 
};
%------------ Algebraic Closure of Fields Header ---------------------
\node[fancytitle, right=10pt] at (box.north west) {Algebraic Closure of Fields};
\end{tikzpicture}

%------------ Fundamental Theorem of Algebra Definitions ---------------
\begin{tikzpicture}
\node [mybox] (box){%
    \begin{minipage}{0.3\textwidth}
    The field $\mathds{C}$ is algebraically closed.
    \end{minipage} 
};
%------------ Fundamental Theorem of Algebra Header ---------------------
\node[fancytitle, right=10pt] at (box.north west) {Fundamental Theorem of Algebra};
\end{tikzpicture}

%------------ Finite Fields Definitions ---------------
\begin{tikzpicture}
\node [mybox] (box){%
    \begin{minipage}{0.3\textwidth}
    $\bullet$ For every prime $p \in \mathds{N}$ there exists a field $\mathds{F}_p$ of order $p$, e.g.$\mathds{Z}/p \mathds{Z}$.\\
    $\bullet$ For any finite field $F$, the order of $F$ is $q=p^r$ for some prime $p$ and positive integer $r$.
    \end{minipage} 
};
%------------ Finite Fields Header ---------------------
\node[fancytitle, right=10pt] at (box.north west) {Finite Fields};
\end{tikzpicture}

%------------ Structure Theorem for Finite Fields Definitions ---------------
\begin{tikzpicture}
\node [mybox] (box){%
    \begin{minipage}{0.3\textwidth}
    Let $p$ be a prime integer and let $q=p^r$ for some positive integer $r$. Then the following statements hold.
    \begin{itemize}[noitemsep,topsep=0pt]
        \item There exists a field of order $q.$
        \item Any two fields of order $q$ are isomorphic.
        \item Let $K$ be a field of order $q$. The multiplicative group $K^x$ of non-zero elements of $K$ is a cyclic group of order $q-1$.
        \item Let $K$ be a field of order $q$. The elements of $K$ are the roots of $x^q-x \in \mathds{F}_p[x]$.
        \item A field of order $p^r$ contains a field of order $p^k \iff k|r$
        \item The irreducible factors of $x^q-x$ over $\mathds{F_p}$ are the irreducible polynomials in $\mathds{F_p}[x]$ whose degree divides $r$.
    \end{itemize}
    \end{minipage} 
};
%------------ Structure Theorem for Finite Fields Header ---------------------
\node[fancytitle, right=10pt] at (box.north west) {Structure Theorem for Finite Fields};
\end{tikzpicture}

\end{multicols*}
\end{document}
