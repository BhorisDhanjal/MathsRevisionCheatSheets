\documentclass[dvipsnames]{article}
\usepackage{geometry}
\geometry{a3paper, landscape, margin=2in}
\usepackage{url}
\usepackage{multicol}
\usepackage{esint}
\usepackage{amsfonts}
\usepackage{tikz}
\usetikzlibrary{decorations.pathmorphing}
\usepackage{amsmath,amssymb}
\usepackage{enumitem}
\usepackage{colortbl}
\usepackage{mathtools}
\usepackage{ mathrsfs }
\usepackage{ dsfont }
\usepackage{ gensymb }
\usepackage[activate={true,nocompatibility},final,tracking=true,kerning=true,spacing=true,factor=1100,stretch=10,shrink=10]{microtype}
\makeatletter
\setlist[itemize]{noitemsep, topsep=0pt}
\newcommand*\bigcdot{\mathpalette\bigcdot@{.5}}
\newcommand*\bigcdot@[2]{\mathbin{\vcenter{\hbox{\scalebox{#2}{$\m@th#1\bullet$}}}}}
\makeatother
%Sheet made by Boris, Template is by Drew Ulick https://de.overleaf.com/articles/130-cheat-sheet/ntwtkmpxmgrp
\title{Introductory Complex Analysis Cheat Sheet}
\usepackage[english]{babel}
\usepackage{palatino}
\usepackage{bm}
	\definecolor{cobalt}{rgb}{0.0, 0.28, 0.67}
\advance\topmargin-1.2in
\advance\textheight3in
\advance\textwidth3in
\advance\oddsidemargin-1.5in
\advance\evensidemargin-1.5in
\parindent0pt
\parskip2pt
\newcommand{\hr}{\centerline{\rule{3.5in}{1pt}}}
\newcommand{\R}{\mathbb{R}}
\newcommand{\C}{\mathbb{C}}
\newcommand{\Z}{\mathbb{Z}}
\newcommand{\N}{\mathbb{N}}
\newcommand{\D}{\mathbb{D}}
% \colorbox[HTML]{e4e4e4}{\makebox[\textwidth-2\fboxsep][l]{texto}}
\begin{document}

\begin{center}{\fontsize{35}{60}{\textcolor{cobalt}{\textbf{Introductory Complex Analysis Cheat Sheet}}}}\\
\end{center}
\begin{multicols*}{3}

\tikzstyle{mybox} = [draw=cobalt, fill=white, very thick,
    rectangle, rounded corners, inner sep=10pt, inner ysep=10pt]
\tikzstyle{fancytitle} =[fill=cobalt, text=white, font=\bfseries]

% ------------------------------------------------------
% Start of Sheet
% ------------------------------------------------------

%------------ Field of Complex Numbers Definitions ---------------
\begin{tikzpicture}
\node [mybox] (box){%
    \begin{minipage}{0.3\textwidth}
    We construct the field of complex numbers as the following quotient ring, $\C=\R[x]/\langle x^2+1 \rangle$\\
    \textbf{Algebra of Complex Numbers}
    \begin{itemize}
        \item Addition: $(a +i b)+(c + i d)=(a+c)+i(b+d)$
        \item Multiplication: $(a+ i b)(c + i d)=(a c - b d)+i(a d +b c)$
        \item Division: $\dfrac{a+i b}{c + i d} = \dfrac{(a c + b d)+i(b c - a d)}{c^2+d^2}$
        \item Square root: $\sqrt{a + i b}=\pm \left( \sqrt{\dfrac{a+\sqrt{a^2+b^2}}{2}} + i \dfrac{b}{|b|} \sqrt{\dfrac{-a+\sqrt{a^2+b^2}}{2}}\right)$
        \item $\Re (a+ib)=a, \Im (a+ib)=b$
    \end{itemize}
    \end{minipage}
};
%------------ Field of Complex Numbers Header ---------------------
\node[fancytitle, right=10pt] at (box.north west) {Field of Complex Numbers};
\end{tikzpicture}

% ==============================================================

%------------ Conjugation, Absolute Value Definitions ---------------
\begin{tikzpicture}
\node [mybox] (box){%
    \begin{minipage}{0.3\textwidth}
    \begin{itemize}
        \item \textbf{Complex conjugation: } $\overline{a+ i b}=a-i b$
        \begin{itemize}
            \item $\overline{a+b}=\overline{a}+\overline{b}$
            \item $\overline{ab}=\overline{a} \cdot \overline{b}$
        \end{itemize}
        Geometrically, conjugation is reflection over the real axis.
        \item \textbf{Absolute value: } $|a|=+\sqrt{a\overline{a}}$
        \begin{itemize}
            \item $|ab|=|a| \cdot |b|$
            \item $|a+b|^2=|a|^2+|b|^2+ 2 \Re (a \overline{b})$
            \item $|a-b|^2=|a|^2+|b|^2- 2 \Re(a \overline{b})$
            \item $|a+b|^2+|a-b|^2=2(|a|^2+|b|^2)$
        \end{itemize}
        The absolute value function forms the metric on $\C$. $\C$ is complete under this metric.
    \end{itemize}

    
    
    
    \end{minipage}
};
%------------ Conjugation, Absolute Value Header ---------------------
\node[fancytitle, right=10pt] at (box.north west) {Conjugation, Absolute Value};
\end{tikzpicture}

%------------ Basic Topological definitions in C Definitions ---------------
\begin{tikzpicture}
\node [mybox] (box){%
    \begin{minipage}{0.3\textwidth}
    \textbf{Some basic results:}
       \begin{itemize}
           \item For $z_0 \in \C, r>0$ we denote the ball (i.e. disk) of radius $r$ around $z_0$ to be $B(z_0,r)=\{ z \in \C \mid |z-z_0| < r\}$
           \item A point $z \in \C$ is a \textbf{limit point} of $E \subseteq \C$ if $\forall \varepsilon>0$, $B(z,\varepsilon)\cap E$ contains a point other than $z$.
           \item A subset $E \subseteq \C$ is said to be \textbf{open} if $\forall z \in E, \exists\ r>0$, s.t. $B(z,r)\subset E.$
           \item A subset $E \subseteq \C$ is said to be \textbf{closed}, if $\C \setminus E$ is open in $C$. Or equivalently a set which contains all its limit points.
           
      \end{itemize} 
    \textbf{Some properties of open sets:}
    \begin{itemize}
        \item $\C$ and \O\ are open subsets of $\C$.
        \item All finite intersections of open sets are open sets.
        \item The collection of all open sets on $\C$ form a topology on $\C$.
    \end{itemize}
    \textbf{Interior, closure, density}
    \begin{itemize}
        \item \textbf{Interior: } Let $E \subseteq \C$. The interior of $E$ is defined as, $E\degree$=set of all interior points of $E$, or equivalently, $\cup \{\Omega \mid \Omega \subseteq E \wedge \Omega \text{ is open in }\C \}$
        \item \textbf{Closure: } Let $E \subseteq \C$. The closure of $E$ is defined as $\hat \{F \mid E \subseteq F \wedge F \text{ is closed in }\C \}$
        \item \textbf{Density: }Let $E \subseteq D$, the closure of $E$ in $D$ is $D$. Then $E$ is called dense in $D$.
      
    \end{itemize}
	    \textbf{Path :} A path in a metric space from a point $x \in X$ to $y \in Y$ is a continuous mapping $\gamma: [0,1] \rightarrow X$ s.t. $\gamma(0)=x$ and $\gamma(1)=y$.\\
    \textbf{Separated and Connected}\\
    For a metric space $(X,d)$.
    \begin{itemize}
        \item \textbf{Separated: } $X$ is separated if $\exists$ disjoint non-empty open subsets $A, B$ of $X$ s.t. $X=A \cup B$.
        \item \textbf{Connected: } 
        \begin{itemize}
            \item $X$ is connected if it has no separation.
            \item $X$ is connected $\iff$ $X$ does not contain a proper subset of $X$ which is both open and closed in $X$.
            \item Continuous functions preserve connectedness.
            \item An open subset $\Omega \in \C$ is connected $\iff$ for $z, w \in \Omega$, there exists a path from $z$ to $w$.
        \end{itemize}
    \end{itemize}
    
    \end{minipage}
};
%------------ Basic Topological definitions in C Header ---------------------
\node[fancytitle, right=10pt] at (box.north west) {Basic Topological definitions in $\C$};
\end{tikzpicture}

%------------ Basic Topological definitions in C contd. Definitions ---------------
\begin{tikzpicture}
\node [mybox] (box){%
    \begin{minipage}{0.3\textwidth}
    \textbf{Open cover: }Let $(X,d)$ be a metric space and $E$ be a collection of open sets in $X$. We say that $\mathscr{U}$ is an open cover of a subset $K \subseteq X$, if $K \subset \bigcup \{ \mathscr{U} \mid  \mathscr{U} \in E\}$\\
    \textbf{Compactness: } For some $K \subseteq X$ is compact if for every open cover $E$ of $K$, there exists $E_1, \cdots. E_n \in E$ s.t. $K \subset U_{i=1}^n E_n$, i.e. it is compact if it has a finite open cover.
\begin{itemize}
    \item In a metric space, a compact set is closed.
    \item A closed subset of a compact set is closed.
\end{itemize}
\textbf{Limit point compact: } We say a metric space $X$ is limit point compact if every infinite subset of $X$ has a limit point.
\begin{itemize}
    \item If $X$ is a compact metric space, then it is also limit point compact.
\end{itemize} 
\textbf{Sequentially compact: } We say a metric space $X$ is sequentially compact if every sequence has a convergent sub-sequence.
\begin{itemize}
    \item If $X$ is limit point compact then $X$ is sequentially compact.
    \item Let $X$ be sequentially compact, then $X$ is a compact metric space.
\end{itemize}
 \textbf{Lebesgue number lemma: } Let $X$ be sequentially compact, and let $\mathscr{U}$ be an open cover of $X$. Then $\exists\ \delta>0$ s.t. for $x \in X$, $\exists\ u\in \mathscr{U} $ s.t. $B(x,\delta) \subseteq u. $
    
    \end{minipage}
};
%------------ Basic Topological definitions in C contd. Header ---------------------
\node[fancytitle, right=10pt] at (box.north west) {Basic Topological definitions in C contd.};
    
\end{tikzpicture}

%------------ Isometries on the Complex Plane Definitions ---------------
\begin{tikzpicture}
\node [mybox] (box){%
    \begin{minipage}{0.3\textwidth}
    A function $f: \C \rightarrow \C $ is called an \textbf{isometry} if $|f(z)-f(w)|=|z-w|\ , \forall z,w \in \C$.
    \begin{itemize}
        \item Let $f$ be an isometry s.t. $f(0)=0$, then the inner product $\langle f(z), f(w)\rangle = \langle z, w \rangle\ , \forall z,w \in \C$.
        \item If $f$ is an isometry s.t. $f(0)=0$ then $f$ is a linear map.
        \item The standard argument for $a+ib \in \C$, Arg$(a+ib)=\tan^{-1}\frac{b}{a}$
    \end{itemize}
    \end{minipage}
};  
%------------ Isometries on the Complex Plane Header ---------------------
\node[fancytitle, right=10pt] at (box.north west) {Isometries on the Complex Plane};
\end{tikzpicture}

%------------ Functions on the Complex Plane Definitions ---------------
\begin{tikzpicture}
\node [mybox] (box){%
    \begin{minipage}{0.3\textwidth}
    \textbf{Uniform convergence: } Let $\Omega \subseteq \C$ and $f_1, \cdots, f_n: \Omega \rightarrow \C$ be a set of functions on $\Omega.$ We say, $\{f\}_{n \in \N}$ converges uniformly to $f$ if given $\varepsilon>0, \exists n \in \N$ s.t. $|f_n(x)-f(x)|<\varepsilon, \forall x\in \Omega$ and $n \geq N$.\\
    \textbf{Complex exponential: } For $z \in C$, exp$(z)=\sum_{n=0}^\infty \frac{z^n}{n!}$\\
    \textbf{Trigonometric functions: } For $z\in \C$, $\cos (x)=\frac{e^{iz}+e^{-iz}}{2}$ and $\sin (x)=\frac{e^{iz}-e^{iz}}{2}$\\
    \textbf{Hyperbolic trigonometric functions: } For $z \in \C$, $\cosh(x)=\frac{e^{z}+e^{-z}}{2}$ and $\sinh(z)=\frac{e^{z}-e^{-z}}{2}$
    \end{minipage}
};  
%------------ Functions on the Complex Plane Header ---------------------
\node[fancytitle, right=10pt] at (box.north west) {Functions on the Complex Plane};
\end{tikzpicture}

%------------ Complex differentiability Definitions ---------------
\begin{tikzpicture}
\node [mybox] (box){%
    \begin{minipage}{0.3\textwidth}
    \textbf{Complex derivative: }Let $\Omega \subseteq \C$ and $f: \Omega \rightarrow \C$, we say that $f$ is complex differentiable at a point $z_0 \in \Omega$ if $z_0$ is an interior point and the following limit exists $\lim_{z \rightarrow z_0}\frac{f(z)-f(z_0)}{z-z_0}$. The limit is denoted as $f'(z_0)$ or $\frac{\mathrm{d} f(z)}{\mathrm{d}z}$.\\
    \textbf{Holomorphic functions: }If $f: \Omega \rightarrow \C$ is complex differentiable at every point $z \in \Omega$, then $f$ is said to be a holomorphic on $\Omega$.
    \textbf{Entire function: } Functions which are complex differentiable on $\C$ are called entire functions.
    \begin{itemize}
        \item Complex differentiability implies continuity.
        \item Complex derivatives of a function are linear transformations.
        \item \textbf{Product rule: }If $f,g: \Omega \rightarrow \C$ are complex differentiable at $z_0 \in \Omega$. Then $fg$ is complex differentiable at $z_0$ with derivative $f'(z_0)g(z_0)+g'(z_0)f(z_0).$
        \item \textbf{Quotient rule: }If $f,g: \Omega \rightarrow \C$ are complex differentiable at $z_0 \in \Omega$, and $g$ doesn't vanish at $z_0$. Then $\left(\frac{f}{g}\right)'(z_0)=\frac{f'(z_0)g(z_0)-g'(z_0)f(z_0)}{g(z_0)^2}$
        \item \textbf{Chain rule: }If $f: \Omega \rightarrow \C$ and $g: D \rightarrow \C$ are complex differentiable at $z_0 \in \Omega$, and $f(\Omega)\subseteq D$. Then $g(f(x))'(z_0)=g'(f(z_0))f'(z_0)$
    \end{itemize}
    
    \end{minipage}
};  
%------------ Complex differentiability Header ---------------------
\node[fancytitle, right=10pt] at (box.north west) {Complex differentiability};
\end{tikzpicture}

%------------ Power Series Definitions ---------------
\begin{tikzpicture}
\node [mybox] (box){%
    \begin{minipage}{0.3\textwidth}
    \textbf{Formal Power Series:} A formal power series around $z_0 \in \C$ is a formal expansion $\sum_{n=0}^\infty a_n(z-z_0)^n$, where $a_n \in \C$ and $z$ is indeterminate.\\
    \textbf{Radius of convergence:} For a formal power series $\sum a_n(z-z_0)^n$ the radius of convergence $R \in [0,\infty]$ given by $R= \liminf_{n \rightarrow \infty} |a_n|^{-1/n}$. Using the ratio test is identical i.e. $R= \liminf_{n \rightarrow \infty} \frac{|a_n|}{|a_{n+1}|}$.
    \begin{itemize}
        \item The series converges absolutely when $z \in B(z_0, R)$, and for $r< R$, the series converges uniformly, else if $|z-z_0|>R$ the series diverges.
        \item Let $z \in \C$ s.t. $|z-z_0|>R$, then $\exists$ infinitely many $n \in N$ s.t. $|a_n|^{-1/n}<|z-z_0|$.
    \end{itemize}
    \textbf{Abel's Theorem: }Let $F(z)=\sum_{n=0}^{\infty} a_n(z-z_0)^n$ be a power series with a positive radius of convergence $R$, suppose $z_1=z_0 + Re^{i \theta}$ be a point s.t. $F(z_1)$ converges. Then $\lim_{r \rightarrow R^-}F(z_0+re^{i \theta})=F(z_1)$
    \end{minipage}
};  
%------------ Power Series Header ---------------------
\node[fancytitle, right=10pt] at (box.north west) {Power Series};
\end{tikzpicture}

%------------ Differentiation of Power Series Definitions ---------------
\begin{tikzpicture}
\node [mybox] (box){%
    \begin{minipage}{0.3\textwidth}
    Let $F(z)=\sum_{n=1}^\infty a_n(z-z_0)^n$ be a power series around $z_0$ with a radius of convergence $R$. Then F is \textbf{holomorphic} in $B(z_0, R)$.
    \begin{itemize}
        \item $F(x)'=\sum_{n=1}^{\infty}n a_n (z-z_0)^{n-1}$ with same radius of convergence $R.$
        \item $a_n=\frac{F^n(z_0)}{n!}$
    \end{itemize}
    \textbf{Cauchy product of two power series: } For power series $F(z)=\sum a_n(z-z_0)^n$ and $G(z)=\sum a_n(z-z_0)^n$ with degree of convergence at least $R.$ Then the Cauchy product $F(z)G(z)=\sum c_n(z-z_0)^n$ where $c_n=\sum_{k=0}^n a_k b_{n-k}$ also has degree of convergence at least $R$.
    \end{minipage}
};  
%------------ Differentiation of Power Series Header ---------------------
\node[fancytitle, right=10pt] at (box.north west) {Differentiation of Power Series};
\end{tikzpicture}

%------------ Cauchy-Riemann Differential Equations Definitions ---------------
\begin{tikzpicture}
\node [mybox] (box){%
    \begin{minipage}{0.3\textwidth}
    For a complex function $f(z)=u(z)+iv(z)$,\\ $f'(x)=\frac{\partial u}{\partial x}+i \frac{\partial v}{\partial x}$ or $f'(x)=-i\frac{\partial u}{\partial y}+\frac{\partial v}{\partial y}$\\
    Therefore, we get the two \textbf{Cauchy-Riemann Differential equations},\\
    $\bullet\ \dfrac{\partial u}{\partial x}=\dfrac{\partial v}{\partial y} \hspace{2cm} \bullet\ \dfrac{\partial u}{\partial y}=-\dfrac{\partial v}{\partial x}$\\
    A function is holomorphic \textbf{iff} it satisfies
    the Cauchy-Riemann equations.
    \textbf{Wirtinger derivatives: }\\
    $\bullet\ \dfrac{\partial}{\partial z}=\dfrac{1}{2}\left(\dfrac{\partial }{\partial x}+\dfrac{1}{i}\dfrac{\partial}{\partial y}\right)\hspace{2cm} \bullet\ \dfrac{\partial}{\partial \overline{z}}=\dfrac{1}{2}\left(\dfrac{\partial }{\partial x}-\dfrac{1}{i}\dfrac{\partial}{\partial y}\right)$\\
If $f$ is holomorphic at $z_0$ then, $\frac{\partial f}{\partial \overline{z}}=0$ and $f'(z_0)=\frac{\partial f}{\partial z}(z_0)=2\frac{\partial u}{\partial z}(z_0)$

    \end{minipage}
};  
%------------ Cauchy-Riemann Differential Equations Header ---------------------
\node[fancytitle, right=10pt] at (box.north west) {Cauchy-Riemann Differential Equations};
\end{tikzpicture}

%------------ Harmonic Functions Definitions ---------------
\begin{tikzpicture}
\node [mybox] (box){%
    \begin{minipage}{0.3\textwidth}
    \textbf{Laplacian: }Define $\Delta=\frac{\partial^2}{\partial x^2}+\frac{\partial^2}{\partial y^2}$.\\
    \textbf{Harmonic function: }Let $u: \Omega \rightarrow \R$ be a twice differentiable function. We say that $u$ is a harmonic function if $\Delta u=0$\\
    For any holomorphic function $f$, $\Re(f), \Im(f)$ are examples of harmonic functions, but there are harmonic functions which are not holomorphic.\\
    \textbf{Boundary of a set: }For a metric space $X$, $\Omega \in X$, \\
    the boundary of $\Omega=\partial \Omega= \overline{\Omega}\cap \overline{\Omega^C}$\\
    \textbf{Maximum principle for harmonic functions: }Let $u: \Omega \rightarrow \R$ be a twice differentiable harmonic function. Let $k \subset \Omega$ be a compact sub   set of $\Omega$. Then,
    $\sup_{z\in k} u(z)=\sup_{z \in \partial k} u(z)$ and $\inf_{z\in k} u(z)=\inf_{z \in \partial k} u(z)$\\
    \textbf{Maximum principle for holomorphic functions: }Let $\Omega \subseteq \C$ be open and connected and let $f: \Omega \rightarrow \C$ be a holomorphic function. Then, for compact $k \subseteq \Omega$, we have, 
    $\sup_{z \in k}|f(z)|=\sup_{\partial k} |f(z)|$\\
    \textbf{Harmonic conjugate: }Let $u: \Omega \rightarrow \R$ be a twice differentiable harmonic function. We say that $v: \Omega \rightarrow \R$ is a harmonic conjugate of $u$ if $f=u+iv$ is holomorphic.\\
    $\bullet$ For a harmonic function from $\C$ to $\R$ there exists a uniquely determined harmonic conjugate from $\C$ to $\R$ (up to constants).\\
    \end{minipage}
};  
%------------ Harmonic Functions Header ---------------------
\node[fancytitle, right=10pt] at (box.north west) {Harmonic Functions};
\end{tikzpicture}

%------------ Riemann Sphere Definitions ---------------
\begin{tikzpicture}
\node [mybox] (box){%
    \begin{minipage}{0.3\textwidth}
    \textbf{Extended complex plane:} $\widehat{\C}=\C \bigcup \{\infty\}$\\
    Consider $S^2$, associate every point $z=x+iy$ with a line $L$ that connects to the point $P=(0,0,1)$. $L=(1-t)z+tP$, where $t \in \R$.\\
    The point at which $L$ for some $z$ touches $S^2$ is given as $\left ( \dfrac{2x}{|z|^2+1},\dfrac{2y}{|z|^2+1},\dfrac{|z|^2-1}{|z|^2+1} \right )$, associate $P$ with $\infty$. This gives a stereographic projection of the complex plane unto $S^2$. This sphere is known as the Riemann sphere.
    \end{minipage}
};  
%------------ Riemann Sphere Header ---------------------
\node[fancytitle, right=10pt] at (box.north west) {Riemann Sphere};
\end{tikzpicture}

%------------ Möbius transformations Definitions ---------------
\begin{tikzpicture}
\node [mybox] (box){%
    \begin{minipage}{0.3\textwidth}
    A map $S(z)=\dfrac{az+b}{cz+d}$ for $a,b,c,d \in \C$ is called a Möbius transformation if $ad-bc \neq 0$.\\
    Every mobius transformation is holomorphic at $\C \setminus \{-d/c\}$, i.e. every point other than is zero. 
    \begin{itemize}
        \item The set of all mobius transformations is a group under transposition.
        \item $S$ forms a bijection with $\widehat{\C}$
    \end{itemize}
    Every mobius transformation can be written as composition of, 
    \begin{enumerate}
        \item Translation: $S(z)=z+b, b \in \C$
        \item Dilation: $S(z)=az, a\neq 0,a=e^{i \theta}$
        \item Inversion: $S(z)=1/z$
    \end{enumerate}
    
    
    
    \end{minipage}
};  
%------------ Möbius transformations Header ---------------------
\node[fancytitle, right=10pt] at (box.north west) {Möbius transformations};
\end{tikzpicture}

%------------ Curves in C Definitions ---------------
\begin{tikzpicture}
\node [mybox] (box){%
    \begin{minipage}{0.3\textwidth}
    A continuous parametrized curve is a continuous map $\gamma:[a,b] \rightarrow \C$ for $a,b\in \R$.
    \begin{itemize}
        \item If $a=b$ the curve is trivial.
        \item $\gamma(a)$ is initial point and $\gamma(b)$ is terminal point.
        \item $\gamma$ is said to be closed if $\gamma(a)=\gamma(b)$.
        \item $\gamma$ is said to be simple if  it is injective, i.e. doesn't "cross" itself.
        \item A curve $-\gamma$ is a reversal of $\gamma$ if $\gamma:[-a,-b]\rightarrow \C$ and if $-\gamma(t)=\gamma(-t)$
        \item $\gamma$ is said to be continuously differentiable if $\gamma'(t_0)$ (defined usually) exists and is continuous.
    \end{itemize}
    \textbf{Reparametrization:} We say a curve $\gamma_2:[a_2,b_2]\rightarrow \C$ is a continuous reparametrization of $\gamma_1:[a_1,b_1] \rightarrow \C$, if there exists a homeomorphism $\varphi: [a_1,b_1] \rightarrow [a_2,b_2]$ s.t.$\varphi(a_1)=a_2, \varphi(b_1)=b_2$ and $\gamma_2(\varphi(t))=\gamma_1(t) \forall t \in [a_1,b_1]$.
    \begin{itemize}
        \item Reparametrization is an equivalence relation.
    \end{itemize}
    
    \textbf{Arc length:} Arc length of curve $\gamma=|\gamma|=\sup \sum_{i=0}^n|\gamma(x_{i+1}-\gamma(x_i))|$ for all partitions of $[a,b]$.
    \begin{itemize}
        \item A curve that has a finite arc length is called \textbf{rectifiable}.
        \item $\displaystyle |\gamma|=\int_{a}^b|\gamma'(t)|\,dt$
    \end{itemize}
    
    \end{minipage}
};  
%------------ Curves in C Header ---------------------
\node[fancytitle, right=10pt] at (box.north west) {Curves in $\C$};
\end{tikzpicture}


%------------ First Fundamental Theorem of Calculus Definitions ---------------
\begin{tikzpicture}
\node [mybox] (box){%
    \begin{minipage}{0.3\textwidth}
Let $f:\Omega \rightarrow \C$ be a continuous function. Let $F:\Omega \rightarrow \C$ be called the anti-derivative of $f$, i.e. $F$ is holomorphic in $\Omega$ and $F'(z)=f(z), \forall z\in\Omega$.
 For a rectifiable curve $\gamma$, $\int_\gamma f(z) dz=F(z_1)-F(z_0)$, where $z_0$ is the initial point and $z_1$ is the terminal point.
    \end{minipage}
};  
%------------ First Fundamental Theorem of Calculus Header ---------------------
\node[fancytitle, right=10pt] at (box.north west) {First Fundamental Theorem of Calculus};
\end{tikzpicture}

%------------ Second Fundamental Theorem of Calculus Definitions ---------------
\begin{tikzpicture}
\node [mybox] (box){%
    \begin{minipage}{0.3\textwidth}
Let $f:\Omega \rightarrow \C$ be a continuous function such that $\int_\gamma f=0.$ Whenever $\gamma$ is a closed polygonal path contained in $\Omega$. For fixed $z_0 \in \Omega$, define a path $\gamma_1$ from $z_0$ to $z_1$ such that $F(z_1)=\int_{\gamma_1} f(z)\, dz$. Then $F$ is a well defined holomorphic function s.t. $F'(z_1)=f(z_1)\ \forall z_1 \in \Omega$
    \end{minipage}
};  
%------------ Second Fundamental Theorem of Calculus Header ---------------------
\node[fancytitle, right=10pt] at (box.north west) {Second Fundamental Theorem of Calculus};
\end{tikzpicture}


%------------ Properties of complex integration Definitions ---------------
\begin{tikzpicture}
\node [mybox] (box){%
    \begin{minipage}{0.3\textwidth}
    For continuously differentiable curves $\gamma:[a,b] \rightarrow \C$, and $\sigma:[b,c] \rightarrow \C$
    \begin{itemize}
        \item For a reparametrization $\widehat{\gamma}$ of $\gamma$ we can say that $\int_\gamma f(z)\, dz=\int_{\widehat{\gamma}} f(z)\, dz$
        \item $\int_{-\gamma} f(z)\, dz=-\int_{\gamma} f(z)\, dz$
        \item $\int_{\gamma+\sigma}f(z)\, dz=\int_{\gamma}f(z)\, dz+\int_{\sigma}f(z)\,dz$
        \item $\int_\gamma f(z)\, dz=\int_a^b f(\gamma(t))\gamma'(t)\, dt$
        \item If $f$ is bounded by $M$ then $\int_\gamma f(z)\, dz \leq M|\gamma|$
        \item For $c\in \C$, we have, $\int_\gamma (cf+g)(z)\, dz=c\int_\gamma f(z)\, dz+\int_\gamma g(z)\, dz$
    \end{itemize}
    \end{minipage}
};  
%------------ Properties of complex integration Header ---------------------
\node[fancytitle, right=10pt] at (box.north west) {Properties of complex integration};
\end{tikzpicture}

%------------ Homotopy of curves Definitions ---------------
\begin{tikzpicture}
\node [mybox] (box){%
    \begin{minipage}{0.3\textwidth}
    Consider two curves $\gamma_0, \gamma_1 \rightarrow \Omega$ with the same initial and end point $[a,b]$.\\
    We say that $\gamma_0$ is homotopic to $\gamma_1$ ($\gamma_0 \sim \gamma_1$) if there exists a continuous map \\$H: [0,1]\times[a,b] \rightarrow \Omega$ s.t. $H(0,t)=\gamma_0 (t)$ and $H(1,t)=\gamma_1 (t),\ \forall t \in [a,b].$\\
    $H(s,a)=z_0, H(s,b)=z_1\ \forall s \in [0,1]$\\
    For \textbf{closed curves} $\gamma_0$ at $z_0$ and $\gamma_1$ at $z_1$, we say that $\gamma_0$ is homotopic to $\gamma_1$ as closed curves if there exists a continuous map $H:[0,1]\times[a,b]\rightarrow \Omega$, s.t. $H(0,t)=\gamma_0(t), H(1,t)=\gamma_1(t),\ \forall t \in [a,b]$. And $H(s,a)=H(s,b),\ \forall s\in[0,1].$
    \begin{itemize}
        \item Homotopy is an equivalence relation.
    \end{itemize}
    \end{minipage}
};  
%------------ Homotopy of curves Header ---------------------
\node[fancytitle, right=10pt] at (box.north west) {Homotopy of curves};
\end{tikzpicture}

%------------ Cauchy-Goursat Theorem Definitions ---------------
\begin{tikzpicture}
\node [mybox] (box){%
    \begin{minipage}{0.3\textwidth}
    \textbf{Cauchy-Goursat theorem: } If a curve $\gamma_0$ is homotopic to a reparametrization of $\gamma_1$ then, the integral of some function $f:\Omega \rightarrow \C$ is homotopy invariant, i.e., $\displaystyle \int_{\gamma_0}f=\int_{\gamma_1}f$\\
    \textbf{Alternative statement: }Let $f: \Omega \rightarrow \C$ be holomorphic on $\Omega$, and $\gamma_0:[a,b]\rightarrow \Omega$ is a rectifiable curve which is null-homotopic (i.e. homotopic to a constant map). Then, $ \displaystyle \int_{\gamma_0}f(z)\, dz=0$
    \end{minipage}
};  
%------------ Cauchy-Goursat Theorem Header ---------------------
\node[fancytitle, right=10pt] at (box.north west) {Cauchy-Goursat Theorem};
\end{tikzpicture}


%------------ Cauchy's theorem for convex domains Definitions ---------------
\begin{tikzpicture}
\node [mybox] (box){%
    \begin{minipage}{0.3\textwidth}
    Let $\Omega \subseteq \C$ be a convex and open set and $f:\Omega \rightarrow \C$ be holomorphic on $\Omega$. Then $f$ has an anti derivative $F$ on $\Omega$, and if $\gamma$ is a closed rectifiable curve on $\Omega$ then $\int_\gamma f=0$. 
    \end{minipage}
};  
%------------ Cauchy's theorem for convex domains Header ---------------------
\node[fancytitle, right=10pt] at (box.north west) {Cauchy's theorem for convex domains};
\end{tikzpicture}

%------------ Cauchy's integral formula ---------------
\begin{tikzpicture}
\node [mybox] (box){%
    \begin{minipage}{0.3\textwidth}
    Let $f:\Omega \rightarrow \C$ be holomorphic. Fix $z_0 \in \Omega$ and let $r>0$ be s.t. $\overline{B(z_0,r)}\subseteq \Omega$. Suppose $\gamma$ is a closed curve in $\Omega \setminus\{z_0\}$ s.t. $\gamma$ is homotopic to a reparametrization to $\gamma_1$ where $\gamma_1(t)=z_0+re^{it}$ for $t \in [0, 2\pi].$ Then,
    $$f(z_0)=\frac{1}{2\pi i}\int_\gamma \frac{f(z)}{z-z_0}\, dz$$
    \end{minipage}
};  
%------------ Cauchy's integral formula Header ---------------------
\node[fancytitle, right=10pt] at (box.north west) {Cauchy's integral formula};
\end{tikzpicture}

%------------ Complex analytic function ---------------
\begin{tikzpicture}
\node [mybox] (box){%
    \begin{minipage}{0.3\textwidth}
    An alternative statement, we say $f:\Omega \rightarrow \C$ is complex analytical if given $z_0 \in \Omega, \exists\ B(z_0,r)\subseteq\Omega$ s.t. the formal power series $\sum_{n=0}^\infty a_n (z-z_0)^n$ converges in $B(z_0,r)$ to $f$.\\
    Let $f:\Omega \rightarrow \C$ be holomorphic on $\Omega$. Suppose for $z_0 \in \Omega, \overline{B(z_0,r)}\subset \Omega$, then for every $n\in \N$, let $a_n=\frac{1}{2\pi i}\int_\gamma \frac{f(z)}{(z-z_0)^{n+1}}\, dz$ where $\gamma(t)=z_0+re^{it}$ for $t\in[0,2\pi]$.
    Then the power series $\sum_{n=0}^\infty a_n(z-z_0)^n$ converges in $B(z_0,r)$ to $f(z)$.\\
    \textbf{Corollary:} If $f:\Omega\rightarrow\C$ is holomorphic then $f'$ is also holomorphic. Therefore $f$ is infinitely differentiable.
    \end{minipage}
};  
%------------ Complex analytic function Header ---------------------
\node[fancytitle, right=10pt] at (box.north west) {Complex analytic function};
\end{tikzpicture}

%------------ Factor theorem for analytic function ---------------
\begin{tikzpicture}
\node [mybox] (box){%
    \begin{minipage}{0.3\textwidth}
    For a analytic function $f:\Omega \rightarrow \C$ s.t. $f(z_0)=0$ at $z_0 \in \Omega, \exists$ a unique analytic function  $g: \Omega \rightarrow \C$ s.t. $f(z)=(z-z_0)g(z)$
    \end{minipage}
};  
%------------ Factor theorem for analytic function Header ---------------------
\node[fancytitle, right=10pt] at (box.north west) {Factor theorem for analytic function};
\end{tikzpicture}



%------------ Principle of analytical continuation ---------------
\begin{tikzpicture}
\node [mybox] (box){%
    \begin{minipage}{0.3\textwidth}
    \begin{itemize}
        \item Let $\Omega$ be open and connected subset of $\C$. and $f,g:\Omega \rightarrow \C$ be analytic functions on $\Omega$. Suppose $f,g$ agree on a non-empty subset of $\Omega$, and this subset has an accumulation point. Then $f\equiv g$ on $\Omega$. 
        \item A consequence to this is that, non-trivial holomorphic functions have isolated zeros.
    \end{itemize}
    \end{minipage}
};  
%------------ Principle of analytical continuation Header ---------------------
\node[fancytitle, right=10pt] at (box.north west) {Principle of analytical continuation};
\end{tikzpicture}


%------------ Higher-order Cauchy integral formula ---------------
\begin{tikzpicture}
\node [mybox] (box){%
    \begin{minipage}{0.3\textwidth}
    Let $f:\Omega \rightarrow \C$ be analytic on $\Omega$ and $z_0 \in \Omega$ with $\overline{B(z_0,r)}\subseteq \Omega$. Let $\gamma$ be a closed curve in $\Omega\setminus \{z_0\}$ that is homotopic to a reparametrization of $\gamma_1$ where $\gamma_1(t)=z_0+re^{it}$ for $t\in[0,2\pi]$. Then,
    $$f^{(n)}(z_0)=\frac{n!}{2 \pi i}\int_\gamma \frac{f(z)}{(z-z_0)^{n+1}}\, dz $$
    \textbf{Cauchy estimates: }If $|f(z)|\leq M\ \forall z\in \gamma([0,2\pi])$ then, $\forall n \in \N$, then we have $|f^{(n)}(z_0)|\leq\frac{Mn!}{r^n}$
    \end{minipage}
};  
%------------ Higher-order Cauchy integral formula Header ---------------------
\node[fancytitle, right=10pt] at (box.north west) {Higher-order Cauchy integral formula};
\end{tikzpicture}

%------------ Liouville's Theorem ---------------
\begin{tikzpicture}
\node [mybox] (box){%
    \begin{minipage}{0.3\textwidth}
    Let $f$ be a entire function which is bounded. Then $f$ is a constant function.
    \end{minipage}
};  
%------------ Liouville's Theorem Header ---------------------
\node[fancytitle, right=10pt] at (box.north west) {Liouville's Theorem};
\end{tikzpicture}


%------------ Fundamental Theorem of Algebra ---------------
\begin{tikzpicture}
\node [mybox] (box){%
    \begin{minipage}{0.3\textwidth}
    Let $p(z)=a_0+a_1z+\cdots+a_nz^n$ be a non constant polynomial s.t. $a_i\in \C,a_n\neq0$. Then $\exists z_1,z_2,\dots,z_n$ s.t. $p(z)=a_n(z-z_1)\dots(z-z_n).$
    \end{minipage}
};  
%------------ Fundamental Theorem of Algebra Header ---------------------
\node[fancytitle, right=10pt] at (box.north west) {Fundamental Theorem of Algebra};
\end{tikzpicture}

%------------ Morera's Theorem ---------------
\begin{tikzpicture}
\node [mybox] (box){%
    \begin{minipage}{0.3\textwidth}
    Let $f:\Omega \rightarrow \C$ be a continuous function such that, $\int_\gamma f(z)\, dz=0, \forall$ closed polygonal paths $\gamma \in \Omega$. Then $f$ is holomorphic on $\Omega$.
    \end{minipage}
};  
%------------ Morera's Theorem Header ---------------------
\node[fancytitle, right=10pt] at (box.north west) {Morera's Theorem};
\end{tikzpicture}


%------------ Uniform limit of holomorphic functions ---------------
\begin{tikzpicture}
\node [mybox] (box){%
    \begin{minipage}{0.3\textwidth}
    Let $f_n:\Omega \rightarrow \C$ be a holomorphic on $\Omega, \forall n \in \N$ s.t. $f_n$ converges uniformly on compact sets to $f$. Then $f$ is holomorphic.
    \end{minipage}
};  
%------------ Uniform limit of holomorphic functions Header ---------------------
\node[fancytitle, right=10pt] at (box.north west) {Uniform limit of holomorphic functions};
\end{tikzpicture}

%------------ Winding number ---------------
\begin{tikzpicture}
\node [mybox] (box){%
    \begin{minipage}{0.3\textwidth}
    Let $\gamma:[a,b]\rightarrow \C$ be a closed curve and let $z_0$ be a point not in the image of $\gamma$. Then the winding number of $\gamma$ around $z_0$ is
    $$W_\gamma(z_0)=\frac{1}{2 \pi i}\int_\gamma \frac{dz}{z-z_0} $$
    \begin{itemize}
        \item Winding number is invariant over homotopy.
        \item Let $z_0$ be a point not in the image of $\gamma$ then $\exists r>0$ s.t. for $z\in B(z_0,r),W_\gamma(z_0)=W_\gamma(z)$
        \item The winding number is always an integer.
        \item The winding number is locally constant.
    \end{itemize}
    \textbf{Generalized Cauchy Integral formula}: Let $f:\Omega \rightarrow \C$ be holomorphic on $\Omega$ and $\gamma:[a,b]\rightarrow \Omega$ be a closed curve which is null homotopic. Then for $z_0$ not in the image of $\gamma$,
    $$f(z_0)W_\gamma(z_0)=\frac{1}{2\pi i}\int_\gamma \frac{f(z)}{(z-z_0)}\, dz$$
    \end{minipage}
};  
%------------ Winding number Header ---------------------
\node[fancytitle, right=10pt] at (box.north west) {Winding number};
\end{tikzpicture}


%------------Open Mapping Theorem ---------------
\begin{tikzpicture}
\node [mybox] (box){%
    \begin{minipage}{0.3\textwidth}
    \begin{itemize}
        \item $f:\Omega \rightarrow \C$ be holomorphic on $\Omega$. Then $G: \Omega \times \Omega \rightarrow \C$ given by 
    $$G(z,w)=\left\{\begin{matrix}
\frac{f(z)-f(w)}{z-w} & \text{when }z \neq w \\
 f'(z) & \text{when }z=w 
\end{matrix}\right.$$ then $G$ is continuous.
\item Let $f:\Omega \rightarrow \C$ be holomorphic on some open set. Suppose $z_0 \in \Omega$ s.t. $f'(z_0)\neq 0$. Then $\exists$ a neighbourhood $U$ of $z_0 \in \Omega$ s.t. $f$ restricted to $U$ is injective. And $V=f(U)$ is an open set and the inverse $g:V\rightarrow U$ of $ f$ is holomorphic.
\item Let $f:\Omega \rightarrow \C$ be a non-constant holomorphic function on open, connected set $\Omega$. Let $z_0 \in \Omega$ and $w_0=f(z_0).$ Then $\exists$ a neighbourhood $U$ of $z_0$ and bijective holomorphic function $\varphi$ on $U$ s.t. $f(z)=w_0+(\varphi(z))^m$ for $z\in U$ and some integer $m>0$. And $\varphi$ maps $U$ unto $B(0,r)$ for some $r>0.$
    \end{itemize}

\textbf{Open Mapping Theorem: }Let $f:\Omega \rightarrow \C$ be a non-constant holomorphic function on open connected set $\Omega$, then 
$f(\Omega)$ is an open set.
\end{minipage}
};  
%------------ Open Mapping Theorem Header ---------------------
\node[fancytitle, right=10pt] at (box.north west) {Open Mapping Theorem};
\end{tikzpicture}

%------------ Schwarz reflection principle ---------------
\begin{tikzpicture}
\node [mybox] (box){%
    \begin{minipage}{0.3\textwidth}
    Let $\Omega$ be a open connected set which is symmetric w.r.t $\R$. Then define the following,
    \begin{itemize}
        \item $\Omega_+=\{z\in \Omega \mid \Im(z)>0 \}$
        \item $\Omega_-=\{z \in \Omega \mid \Im(z)<0\}$
        \item $I=\{ z \in \Omega \mid \Im(z)=0\}$
    \end{itemize}
    \textbf{Schwarz reflection principle: } Let $\Omega$ be defined as above. Then if $f:\Omega_+ \bigcup I \rightarrow \C$ which is continuous on $\Omega_+ \bigcup I$ and holomorphic on $\Omega_+$. Suppose for $f(x)\in \R,\ \forall x \in I$ then there exists $g:\Omega \rightarrow \C$ holomorphic on $\Omega$ s.t. $g(z)=f(z)$ for $z \in \Omega_+\bigcup I$
    \end{minipage}
};  
%------------ Schwarz reflection principle Header ---------------------
\node[fancytitle, right=10pt] at (box.north west) {Schwarz reflection principle};
\end{tikzpicture}


%------------ Singularity of a holomorphic function ---------------
\begin{tikzpicture}
\node [mybox] (box){%
    \begin{minipage}{0.3\textwidth}
    \begin{itemize}
        \item \textbf{Isolated singularity: }If $f$ is holomorphic on $B(z_0,R)\setminus \{z_0\}$ for some $R>0$ then $z_0$ is called an isolated singularity.
        \item \textbf{Removable singularity: }Let $z_0$ be an isolated singularity of a holomorphic function $f$ as defined above. It is called removable if there exists holomorphic function $g$ on $B(z_0,R)$ s.t. $g(z)=f(z)$ on $B(z_0,R)\setminus \{z_0\}.$
        \item \textbf{Riemann removable singularity theorem: }Let $z_0$ be an isolated singularity of a function $f$, then $z_0$ is a removable singularity if and only if $f$ is locally bounded around $z_0$.
        \item \textbf{Pole: }If $z_0$ is an isolated singularity as defined above and if $\displaystyle \lim_{z \rightarrow z_0} |f(z)|=\infty$ then $z_0$ is called a pole of $f$.
        \item \textbf{Essential singularity: }A singularity that is neither removable nor a pole.
    \end{itemize}
    \end{minipage}
};  
%------------ Singularity of a holomorphic function Header ---------------------
\node[fancytitle, right=10pt] at (box.north west) {Singularity of a holomorphic function};
\end{tikzpicture}

%------------ Doubly infinite series ---------------
\begin{tikzpicture}
\node [mybox] (box){%
    \begin{minipage}{0.3\textwidth}
    Let $z_n$ be a function defined for $n=,0, \pm1, \pm2,\cdots$, then it is doubly infinite.
    \begin{itemize}
        \item A doubly infinite series converges if $\sum_{n=0}^\infty a_n$ and $\sum_{n=1}^\infty a_{=n}$ both converge.
    \item Splitting up the series in similar manners you can define absolute and uniform convergence. 
    \end{itemize}
    
    \end{minipage}
};  
%------------ Doubly infinite series Header ---------------------
\node[fancytitle, right=10pt] at (box.north west) {Doubly infinite series};
\end{tikzpicture}

%------------ Annulus series ---------------
\begin{tikzpicture}
\node [mybox] (box){%
    \begin{minipage}{0.3\textwidth}
    An annulus $A(z_0,R_1,R_2)$ around a point $z_0$ for $0\leq R_1 \leq R_2$ is the set of all $z \in \C$ s.t. $R_1 \leq |z-z_0| \leq R_2.$
    \end{minipage}
};  
%------------ Annulus Header ---------------------
\node[fancytitle, right=10pt] at (box.north west) {Annulus};
\end{tikzpicture}

%------------ Laurent series expansion ---------------
\begin{tikzpicture}
\node [mybox] (box){%
    \begin{minipage}{0.3\textwidth}
    Let $f$ be a function holomorphic on $A(z_0, R_1, R_2)$, then there exists $a_n \in \C$ for $n \in \Z$ s.t.
    $$ f(z)=\sum_{n=-\infty}^\infty a_n(z-z_0)^n$$ where the doubly infinite series converges absolutely and uniformly in some $A(z_0,r_1,r_2)$ when $R_1<r_1<r_2<R_2$.\\
    $$a_n=\frac{1}{2\pi i}\int_\gamma \frac{f(z)}{(z-z_0)^{n+1}}\,dz$$
    where $\gamma(z)=z_0+re^{it}$ for $t\in[0,2\pi]$ and $R_1<r<R_2$.\\
    \textbf{Important results}
    \begin{itemize}
        \item $f$ has a removable singularity at $z_0 \iff a_n=0$ for $n<0$ in the Laurent series expansion of $f$
        \item $f$ has a pole at $z_0$ of order $m \iff a_n=0$ for $n<-m$ in the Laurent series expansion of $f$.  
        \item $f$ has a essential singularity at $z_0 \iff a_n\neq 0$ for infinitely many negative integers $n$.
    \end{itemize}
    \end{minipage}
};  
%------------ Laurent series expansion Header ---------------------
\node[fancytitle, right=10pt] at (box.north west) {Laurent series expansion};
\end{tikzpicture}

%------------ Casorati–Weierstrass theorem ---------------
\begin{tikzpicture}
\node [mybox] (box){%
    \begin{minipage}{0.3\textwidth}
    Let $z_0$ be an essential singularity of $f$ then given $\alpha \in \C$, there exists a sequence $z_n \in B(z_0, R)\setminus \{z_0\}$ s.t. $z_n \rightarrow z_0$ and $f(z_n)\rightarrow \alpha.$\\
    \begin{itemize}
        \item Alternatively, $f$ approaches any given value arbitrarily closely in any neighborhood of an essential singularity.
    \end{itemize}
    \end{minipage}
};  
%------------ Casorati–Weierstrass theorem Header ---------------------
\node[fancytitle, right=10pt] at (box.north west) {Casorati–Weierstrass theorem};
\end{tikzpicture}

%------------ Meromorphic functions ---------------
\begin{tikzpicture}
\node [mybox] (box){%
    \begin{minipage}{0.3\textwidth}
    Let $\Omega$ be a open connected subset of $\C$ and let $S \subset \Omega$. Let $f: \Omega \setminus S \rightarrow \C$ be holomorphic on $\Omega.$ We say that $f$ is a meromorphic function on $\Omega$ if, 
    \begin{itemize}
        \item $S$ is a discrete set.
        \item $f$ either has removable singularities or poles at point of $S$.
    \end{itemize}
    \end{minipage}
};  
%------------ Meromorphic functions Header ---------------------
\node[fancytitle, right=10pt] at (box.north west) {Meromorphic functions};
\end{tikzpicture}

%------------ Operations on meromorphic functions ---------------
\begin{tikzpicture}
\node [mybox] (box){%
    \begin{minipage}{0.3\textwidth}
    Let $\mathcal{M}(\Omega)$ denote the equivalence classes of meromorphic functions over $\Omega$.
    \begin{itemize}
        \item  We say that two meromorphic functions $f:\Omega\setminus S_1$ and $g: \Omega \setminus S_2$ are equivalent if $f(z)=g(z)$ on $\Omega \setminus (S_1 \bigcup S_2)$.
        \item For $f, g \in \mathcal{M}(\Omega)$, define $f+g$ to be the equivalence class of $(f+g):\Omega \setminus (S_1 \bigcup S_2) $
        \item Similarly, $fg$ is the equivalence class of $fg: \Omega \setminus (S_1 \bigcup S_2).$
    \end{itemize}
    \textbf{The space of all meromorphic functions is a field.}
    \end{minipage}
};  
%------------ Operations on meromorphic functions Header ---------------------
\node[fancytitle, right=10pt] at (box.north west) {Operations on meromorphic functions};
\end{tikzpicture}



%------------ Order of meromorphic functions ---------------
\begin{tikzpicture}
\node [mybox] (box){%
    \begin{minipage}{0.3\textwidth}
    The order of a meromorphic function is defined as follows,
    \begin{itemize}
        \item If $z_0 \in S$ is a removable singularity then the order of $f$ at $z_0$ is the order of the zero at $z_0$ of $f$, i.e., $f(z)=(z-z_0)^m g(z)$ then $m$ is the order.
        \item If $z_0 \in S$ is a pole and the pole is of order $m$ then order of $f$ at $z_0$ is $-m$.
        \item If $f \equiv 0$ then $\text{Ord}_{z_0}=\infty.$
        \item $\text{Ord}_{z_0}(f+g)\geq \min (\text{Ord}_{z_0}(f), \text{Ord}_{z_0}(g))$
        \item $\text{Ord}_{z_0}(fg)=\text{Ord}_{z_0}(f)+\text{Ord}_{z_0}(g)$
    \end{itemize}

    \end{minipage}
};  
%------------ Order of meromorphic functions Header ---------------------
\node[fancytitle, right=10pt] at (box.north west) {Order of meromorphic functions};
\end{tikzpicture}

%------------ Residue of a function ---------------
\begin{tikzpicture}
	\node [mybox] (box){%
		\begin{minipage}{0.3\textwidth}
			\textbf{Residue of a function:} Let $f:\Omega\setminus S \rightarrow\C$ be a holomorphic function, where $ \Omega $ is an open set and $ S $ is a discrete subset of $ \Omega $. Then for $ z_0 \in S $, let $r>0$ be s.t. $ \overline{B(z_0,r)}\subseteq \Omega$ and $ B(z_0,r)=\{ z_0 \}.$ Then in $ B(z_0,r) \setminus \{ z_0 \} $, consider the Laurent series expansion of $ f $ given by $ f(z)=\sum_{n=-\infty}^{\infty} a_n(z-z_0)^n$. We define the residue of $ f $ at $ z_0 $ to be $ \text{Res}(f,z_0)= a_{-1}$.
			\begin{enumerate}
				\item If $ z_0 $ is a removable singularity then Res$ (z_0)=0 $.
				\item If $ z_0$ is a pole of order $m$ then $ (z-z_0)^m f(z)=g(z) $, where $ g(z)\neq0  $ on $ B(z_0,r)\setminus\{z_0\} $then, Res$ (z_0)= a_{m-1}=\frac{g^{(m-1)}(z_0)}{(m-1)!}.$
			\end{enumerate}
		\end{minipage}
	};  
	%------------ Residue of a function Header ---------------------
	\node[fancytitle, right=10pt] at (box.north west) {Residue of a function	};
\end{tikzpicture}


%------------ Residue theorem ---------------
\begin{tikzpicture}
	\node [mybox] (box){%
		\begin{minipage}{0.3\textwidth}
			Let $ \Omega $ be an open connected subset of $ \C  $ and $ S $ be a finite subset of $ \Omega $ and let $ f:\Omega\setminus S \rightarrow \C  $ be a holomorphic function. Let $ \gamma  $ be a null homotopic closed curve on $ \Omega  $. Then,
				\[ \dfrac{1}{2 \pi i} \int_\gamma f(z)\, dz= \sum_{j=1}^{k} W_\gamma(z_j) \text{Res}(f,z_j)\] where $ S=\{z_1,\cdots, z_k\} $ and $ W_\gamma $ is the winding number.
		\end{minipage}
	};  
	%------------ Residue theorem Header ---------------------
	\node[fancytitle, right=10pt] at (box.north west) {Residue theorem};
\end{tikzpicture}

%------------ Log derivative ---------------
\begin{tikzpicture}
	\node [mybox] (box){%
		\begin{minipage}{0.3\textwidth}
			For a holomorphic function $ f:\Omega \rightarrow \C  $. Define the log derivative of $ f $ to be the meromorphic function $ \frac{f'(z)}{f(z)}. $
			\begin{enumerate}
				\item $\frac{(fg)'}{fg}=\frac{f'}{f}+\frac{g'}{g}$
				\item $ \frac{(f/g)'}{(f/g)}=\frac{f'}{f} - \frac{g'}{g}$
				\item When $ f $ has a pole of order $ m $ at $ z_0 $ then for $f(z)=\frac{g(z)}{(z-z_0)^m} $ the log derivative of $ f $ is $ \frac{g'(z)}{g(z)} -\frac{m}{(z-z_0)}$
			\end{enumerate}
		\end{minipage}
	};  
	%------------ Log derivative Header ---------------------
	\node[fancytitle, right=10pt] at (box.north west) {Log derivative};
\end{tikzpicture}


%------------ Argument principle ---------------
\begin{tikzpicture}
	\node [mybox] (box){%
		\begin{minipage}{0.3\textwidth}
			Let $ f: \Omega\setminus S \rightarrow \C$ be a meromorphic function s.t. $ f $ has zeros of order $ d_1,\dots,d_n $ at $ z_1, \dots z_n $ after removing the removable singularities. And $ f $ has poles of order $ e_1, \dots, e_m $ at points $ w_1,\dots, w_m $. Let $ \gamma$ be a closed curve which is null homotopic in $ \Omega $ s.t. the zeros and poles don't lie in the image of $ \gamma $. Then,
			\[ \dfrac{1}{2 \pi i} \int_\gamma \frac{f'(z)}{f(z)}\, dz = \sum_{i=0}^{n} d_i W_\gamma(z_i)- \sum_{j=1}^{m}e_j W_\gamma (w_j)\]
		\end{minipage}
	};  
	%------------ Argument principle Header ---------------------
	\node[fancytitle, right=10pt] at (box.north west) {Argument principle};
\end{tikzpicture}


%------------ Rouche's theorem ---------------
\begin{tikzpicture}
	\node [mybox] (box){%
		\begin{minipage}{0.3\textwidth}
			Let $ \gamma  $ be a closed curve which is null homotopic in $ \Omega  $. Let $ f,g $ be functions holomorphic in $ \Omega $ and $ |g(z)|<|f(z)| $ on $ \gamma  $ then $ f $ and $ f+g $ have the same number of zeros counting multiplicities on the interior of $ H([0,1]\times[a,b]) $ where $ H $ is the null homotopy from $ \gamma $ to a constant path.
		\end{minipage}
	};  
	%------------ Rouche's theorem Header ---------------------
	\node[fancytitle, right=10pt] at (box.north west) {Rouche's theorem};
\end{tikzpicture}


%------------ Branch of the complex logarithm ---------------
\begin{tikzpicture}
	\node [mybox] (box){%
		\begin{minipage}{0.3\textwidth}
			Let $ \Omega $ be an open connected subset of $ \C\setminus\{0\} $. Define a branch of the logarithm on $\Omega$ as a function $ f:\Omega \rightarrow \C $ s.t. $ \exp (f(z))=z, \forall z\in \Omega.	 $\\
			For $\Omega=\C \setminus \{\Re(x) \leq 0\}$ define the standard branch to be 
			\[ \text{Log}(z)=\ln|z|+i \text{Arg}(z)	 \]
			As defined above Log$ (z) $ is holomorphic on $\Omega$.
		\end{minipage}
	};  
	%------------ Branch of the complex logarithm Header ---------------------
	\node[fancytitle, right=10pt] at (box.north west) {Branch of the complex logarithm};
\end{tikzpicture}

%------------ Schwarz lemma ---------------
\begin{tikzpicture}
	\node [mybox] (box){%
		\begin{minipage}{0.3\textwidth}
			Let $ \D $ denote the open unit disc. Let $ f: \D \rightarrow \D  $ be a holomoprhic function s.t. $ f(0)=0 $. Then, \[ |f(z)| \leq |z|, \forall z \in \D, \text{ and } |f'(z)|\leq 1	  \]
			Also, if $ |f(z)|=|z| $ for some $ z \in \D $ or if $ |f'(0)|=1 $ then $ \exists \lambda \in \C, |\lambda|=1 $ s.t. $ f(z)=\lambda z $.
		\end{minipage}
	};  
	%------------ Schwarz lemma Header ---------------------
	\node[fancytitle, right=10pt] at (box.north west) {Schwarz lemma};
\end{tikzpicture}


%------------ Automorphism ---------------
\begin{tikzpicture}
	\node [mybox] (box){%
		\begin{minipage}{0.3\textwidth}
			A function $ f:\Omega \rightarrow \Omega $ is an automorphism if $ f $ is holomorphic and has a holomorphic inverse.	
		\end{minipage}
	};  
	%------------ Automorphism Header ---------------------
	\node[fancytitle, right=10pt] at (box.north west) {Automorphism};
\end{tikzpicture}


%------------ Automorphisms of the unit disc ---------------
\begin{tikzpicture}
	\node [mybox] (box){%
		\begin{minipage}{0.3\textwidth}
			Define a function $ \varphi_\alpha: \D \rightarrow \C $ defined as $ \varphi_\alpha(z)=\frac{z-\alpha}{1-\overline{\alpha}z}$.\\
			Let $ f:\D \rightarrow \D $ be an automorphism. Then there exists $ \alpha \in \D $ and $ \lambda \in \partial \D $ s.t. \[  f(z)=\lambda \varphi_\alpha (z)\]
		\end{minipage}
	};  
	%------------ Automorphisms of the unit disc Header ---------------------
	\node[fancytitle, right=10pt] at (box.north west) {Automorphisms of the unit disc};
\end{tikzpicture}

%------------ Phragmén–Lindelöf method ---------------
\begin{tikzpicture}
	\node [mybox] (box){%
		\begin{minipage}{0.3\textwidth}
			Let $ \Omega =\{z\in\Omega: a < \Re(z) < b\} $. Let $ f:\overline{\Omega} \rightarrow \C $, s.t. $ f $ is continuous on $ \overline{\Omega} $ and holomorphic on $ \Omega $. Suppose for some $ z=x+iy $, we have $|f(z)|<B $ and let\\
			 $ M(x)=\sup\{|f(x+iy)|:-\infty < y< \infty \} $. Then,
			 \[ M(x)^{b-a}\leq M(a)^{b-x}M(b)^{x-a	} \]
			 And further \[ |f(z)|\leq M(x)\leq \max \{M(a),M(b)\}=\sup_{z\in \partial \Omega} |f(z)|\]
		\end{minipage}
	};  
	%------------ Phragmén–Lindelöf method Header ---------------------
	\node[fancytitle, right=10pt] at (box.north west) {Phragmén–Lindelöf method};
\end{tikzpicture}


%------------ Schwarz-Pick theorem ---------------
\begin{tikzpicture}
	\node [mybox] (box){%
		\begin{minipage}{0.3\textwidth}
			First define $ \rho (z,w)=\left| \frac{z-w}{1-\overline{w}z} \right|$ for $z,w \in \D$. Let $ f:\D \rightarrow \D $ be holomorphic. Then,
			\[ \rho(f(z),f(w))\leq \rho (z,w)\ \forall z,w \in \D \]
			and,
			\[ \frac{|f'(z)|}{1-|f(z)|^2} \leq \frac{1}{1-|z|^2}\ \forall z \in \D\]
		\end{minipage}
	};  
	%------------ Schwarz-Pick theorem Header ---------------------
	\node[fancytitle, right=10pt] at (box.north west) {Schwarz-Pick theorem};
\end{tikzpicture}


%------------ Lifting of maps ---------------
\begin{tikzpicture}
	\node [mybox] (box){%
		\begin{minipage}{0.3\textwidth}
			Let $ X, Y, Z $ be open subsets of $ \C $ and let $ f: Y \rightarrow X $ and $ g: Z \rightarrow X $ be continuous maps. Then we say, a map $ \widetilde{g}:Z\rightarrow Y $ is a lift of $ g $ w.r.t. $ f $ if $ f \circ \widetilde{g}=g$. \\
			
			\textbf{Uniqueness of lifts: }Let $ X, Y, Z $ be open connected subsets of $ \C $ and let $ f:Y \rightarrow X$ be a \emph{local} homeomorphism. Let $ g:Z\rightarrow X $ be a continuous map. Let $ \widetilde{g_1}$ and $ \widetilde{g_2} $ be lifts of $ g $ w.r.t. $ f $ and suppose they are equal at some point in $ Z $. Then $ \widetilde{g_1} \equiv \widetilde{g_2}$.
			
				
			\begin{itemize}
				\item Let $ f: Y \rightarrow X$ be a holomorphic map s.t. $ f'(y)\neq 0 $ on $ Y $. Let $ g: Z \rightarrow X $ be a holomorphic map s.t. $ \widetilde{g} : Z \rightarrow Y$ is a lift of $ g $ w.r.t. $ f $. Then $ \widetilde{g} $ is holomorphic.
				\item Let $ X,Y $ be open subsets of $ \C $ let, $ f: Y \rightarrow X$ be a local homeomorphism. Let $ \gamma_0, \gamma_1 $ be curves in $ X $ from $ z_1 $ to $ z_2 $ which are homotopic. Suppose that for every $ s\in [0,1] $, we can lift $ \gamma_s(t)=H(s,t) $ to a path $ \widetilde{\gamma_s}:[a,b] \rightarrow Y$ w.r.t. $ f $ s.t. $ \widetilde{\gamma_s} (a)=\widetilde{z_1},\ \forall s \in [0,1].$ Then $ \widetilde{\gamma_0}, \widetilde{\gamma_1}$ are homotopic in $ Y $.
			\end{itemize}
		\end{minipage}
	};  
	%------------ Lifting of maps Header ---------------------
	\node[fancytitle, right=10pt] at (box.north west) {Lifting of maps};
\end{tikzpicture}


%------------ Covering spaces ---------------
\begin{tikzpicture}
	\node [mybox] (box){%
		\begin{minipage}{0.3\textwidth}
			Let $ X, Y $ be open subsets of $ \C $. We say that a continuous map $ f: Y \rightarrow X $ is a covering map if given $ x \in X $ there exists a neighbourhood $ U $ of $ X $ and open sets $ \{V_\alpha\}_{\alpha \in A} $ in $ Y $ s.t. $ f^{-1} (U) = \coprod_{\alpha \in A} V_\alpha$ (disjoint union of $ V_\alpha $) and $ f|_{V_\alpha}:V_\alpha \rightarrow U$ is a homeomorphism. Then $ Y $ is called a cover of $ X $.
			\begin{itemize}
				\item Let $ f: Y \rightarrow X $ be a covering map and $ \gamma [a,b]\rightarrow X  $ be a curve from $ x_0$ to $ x_1 $ in $ X $. Suppose $ y_0 \in f^{-1}(\{x_0\}) $. Then there exists a unique lift $ \widetilde{\gamma} [a,b]\rightarrow Y	$ of $ \gamma $ w.r.t. $ f $ s.t. $ \widetilde{\gamma} (a)=y_0.$
				\item For connected $ X $ let $ f: Y \rightarrow X $ be a covering map. Suppose $ x_0,x_1 \in X $. Then the cardinality of $ f^{-1} (x_0)$ is the same as the cardinality of $ f^{-1}(x)$.
				\item For open subsets $ X,Y $ of $ \C $ let, $ f: Y \rightarrow X $ be a covering map from $ Y $ to $ X $. Let $ Z $ be an open connected subset of $ \C $, which is simply connected and locally connected. Suppose $ g: Z \rightarrow X $ is a continuous map. Then given $ z_0 \in C $ and $ y_0 \in Y $ s.t. $ g(z_0)=f(y_0) $, then there exists a unique lift $ \widetilde{g}:Z \rightarrow Y $ of $ g $ w.r.t $ f $.
				\item Let $ \Omega $ be a simply connected, locally connected, open connected subset of $ \C $ and $ g: \Omega \rightarrow \C^* $ be a holomorphic map. Then there exists a lift $ \widetilde{g} : \Omega \rightarrow \C$ s.t. $ \exp(\widetilde{g}) =g.$
			\end{itemize}
			
		\end{minipage}
	};  
	%------------ Covering spaces Header ---------------------
	\node[fancytitle, right=10pt] at (box.north west) {Covering spaces};
\end{tikzpicture}


%------------ Bloch's theorem ---------------
\begin{tikzpicture}
	\node [mybox] (box){%
		\begin{minipage}{0.3\textwidth}
			\begin{itemize}
				\item For $ f:\D \rightarrow \C $ s.t. $ f(0)=0, f'(0)=1 $ and $ |f(z)|\leq M\ \forall z \in \D$. Then $ B(0,\frac{1}{6M})\subseteq f(\D).$
				\item Let $ f:B(0,R) \rightarrow \C $ be holomorphic s.t. $ f(0)=0, f'(0)=\mu $ for some $ \mu>0 $ and $ f|(z)|\leq M\ \forall z \in B(0,R). $ Then, $ B(0,\frac{R^2 \mu^2}{6M}) \subseteq f(B(0,R))$.
			\end{itemize}
			\textbf{Bloch's theorem: }Let $ \Omega $ be an open connected subset of $\C $ s.t. $ \overline{\D} \subset \Omega.$  Let$ f:\Omega \rightarrow \C $ s.t. $ f(0)=0, f'(0)=1 .$ Then there exists a ball $ B' $ contained in $ \D $ s.t. $ f|_{B'} $ is injective and $ B(0,\frac{1}{72}) \subseteq f(B') \subseteq f(\D)$.	
		\end{minipage}
	};  
	%------------ Bloch's theorem Header ---------------------
	\node[fancytitle, right=10pt] at (box.north west) {Bloch's theorem};
\end{tikzpicture}


%------------ Little Picard's theorem ---------------
\begin{tikzpicture};
	\node [mybox] (box){%
		\begin{minipage}{0.3\textwidth}
			\begin{itemize}
				\item Let $ \Omega $ be an open connected subset of $ \C $ which is simply connected. Let $ f: \Omega \rightarrow \C $ which omits 0 and 1. Then there exists a holomorphic function $ g:\Omega \rightarrow \C $ s.t. $ f(z)=-\exp(\pi i \cosh (2g(z))) $
				\item The function $ g $ as defined above doesn't contain any disk of radius $ 1 $.
			\end{itemize}
			\textbf{Little Picard's theorem: }If $ f $ is an entire function which omits two points, then $ f $ is a constant function.
		\end{minipage}
	};  
	%------------ Little Picard's theorem Header ---------------------
	\node[fancytitle, right=10pt] at (box.north west) {Little Picard's theorem};
\end{tikzpicture}

% ==============================================================
\end{multicols*}
\end{document}
