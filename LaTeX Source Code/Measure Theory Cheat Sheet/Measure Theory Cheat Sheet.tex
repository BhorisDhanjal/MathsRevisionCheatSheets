\documentclass[dvipsnames]{article}
\usepackage[usenames]{xcolor}
\usepackage{geometry}
\geometry{a3paper, landscape, margin=2in}
\usepackage{url}
\usepackage{multicol}
\usepackage{esint}
\usepackage{amsfonts}
\usepackage{tikz}
\usetikzlibrary{decorations.pathmorphing}
\usepackage{amsmath,amssymb}
\usepackage{enumitem}
\usepackage{colortbl}

\usepackage{mathtools}
\usepackage{ mathrsfs }	
\usepackage{ dsfont }
\usepackage{ gensymb }

\usepackage[activate={true,nocompatibility},final,tracking=true,kerning=true,spacing=true,factor=1100,stretch=10,shrink=10]{microtype}
\makeatletter
\setlist[itemize]{noitemsep, topsep=0pt}
\usepackage{quiver}

\newcommand*\bigcdot{\mathpalette\bigcdot@{.5}}
\newcommand*\bigcdot@[2]{\mathbin{\vcenter{\hbox{\scalebox{#2}{$\m@th#1\bullet$}}}}}
\makeatother
%Sheet made by Boris, Template is by Drew Ulick https://de.overleaf.com/articles/130-cheat-sheet/ntwtkmpxmgrp
\usepackage[english]{babel}
\usepackage{palatino}
\usepackage{bm}
	\definecolor{cobalt}{rgb}{0.0, 0.28, 0.67}
\advance\topmargin-1.2in
\advance\textheight3in
\advance\textwidth3in
\advance\oddsidemargin-1.5in
\advance\evensidemargin-1.5in
\parindent0pt
\parskip2pt
\newcommand{\hr}{\centerline{\rule{3.5in}{1pt}}}
\newcommand{\R}{\mathbb{R}}
\newcommand{\Q}{\mathbb{Q}}
\newcommand{\C}{\mathbb{C}}
\newcommand{\Z}{\mathbb{Z}}
\newcommand{\N}{\mathbb{N}}
\newcommand{\D}{\mathbb{D}}
\newcommand{\F}{\mathbb{F}}
% \colorbox[HTML]{e4e4e4}{\makebox[\textwidth-2\fboxsep][l]{texto}}

\begin{document}
\definecolor{orange}{RGB}{255, 115,0}
	\begin{center}{\fontsize{35}{60}{\textcolor{Dandelion}{\textbf{Measure Theory Cheat Sheet}}}}\\
\end{center}
\begin{multicols*}{3}

\tikzstyle{mybox} = [draw=Dandelion, fill=white, very thick,
    rectangle, rounded corners, inner sep=10pt, inner ysep=10pt]
\tikzstyle{fancytitle} =[fill=Dandelion, text=white, font=\bfseries]

% ------------------------------------------------------
% Start of Sheet
% ------------------------------------------------------

%------------ Topology ---------------
\begin{tikzpicture}
\node [mybox] (box){%
    \begin{minipage}{0.3\textwidth}
	    A collection $ T  $ os subsets of a set $ X $ is said to be a \textbf{topology} in X if $ T  $ satisfies the following properties,
	    \begin{itemize}
	    	\item $ \emptyset \in T  $ and $ X \in T  $
	    	\item Closed under finite intersections
	    	\item Closed under arbitrary unions
	    \end{itemize}
    
    Members of $ T $ are called open sets.
    
    If $ X,Y $ are topological spaces then $ f:X\to Y $ is continuous if $ f^{-1}(V) $ is open in $ X $ for all open sets $ V \in Y $.
    \end{minipage}
};
%------------ Topology Header ---------------------
\node[fancytitle, right=10pt] at (box.north west) {Topology};
\end{tikzpicture}

%------------ Sigma algebra ---------------
\begin{tikzpicture}
	\node [mybox] (box){%
		\begin{minipage}{0.3\textwidth}
			A collection $ F $ of subsets of $ X $ is called a $ \sigma- $algebra if the following properties hold
			\begin{itemize}
				\item $ X \in F $
				\item If $ A \in F $ then $ A^C=A-X \in F $
				\item Closed under unions
			\end{itemize}
		\end{minipage}
	};
%------------ Sigma algebra Header ---------------------
	\node[fancytitle, right=10pt] at (box.north west) {$ \sigma $-algebra};
\end{tikzpicture}

%------------ Measureability ---------------
\begin{tikzpicture}
	\node [mybox] (box){%
		\begin{minipage}{0.3\textwidth}
			\begin{itemize}
				\item If $ F $ is a $ \sigma- $algebra of $ X $ then $ X $ is a \textbf{measurable space} and members of $ F $ are \textbf{measurable sets} in $ X $.
				\item If $ X $ is a measurable space and $ Y $ is a topological space, then $ f:X\to Y $ is said to be \textbf{measurable} if $ f^(-1)(V) $ is a measurable set in $ X $ for all open sets $ V $ in $ Y $.
			\end{itemize}
		\textbf{Characteristic function:} It is a measurable function defined as follows .If $ E $ is a measurable set in $ X $ define $ \chi_E(x)= \begin{cases}
			1 & \text{if } x \in E\\
			0 & \text{if } x \not\in E
		\end{cases} $
		\end{minipage}
	};
	%------------ Measureability Header ---------------------
	\node[fancytitle, right=10pt] at (box.north west) {Measureability};
\end{tikzpicture}

%------------ Borel sigma algebra ---------------
\begin{tikzpicture}
	\node [mybox] (box){%
		\begin{minipage}{0.3\textwidth}
			\textbf{Generated $ \mathbf{\sigma-} $algebra:} For any collection of subsets $ F $ of $ X $ there exists a smallest $ \sigma- $algebra which contains $ F $. It is the intersection of all $ \sigma- $algebras containing $ F $.
			
			\textbf{Borel $\mathbf{\sigma-}$ algebra: }For a topological space $ X $ the $ \sigma- $algebra generated by the family of open sets of $ X $. Elements of a Borel $ \sigma-$algebra are called Borel sets.
			
			\textbf{Borel mapping: }A map between two topological spaces $ f:X\to Y $if the inverse image of an open set in $ Y $ is an element of the Borel $ \sigma- $algebra of $ X $.
			
			\begin{itemize}
				\item If $ f:X\to [-\infty, \infty] $ and $ F $ is a $ \sigma-  $algebra of $ X $, then $ f $  is measurable if $ f^{(-1)}((a,\infty)) \in F $ for all $ a $.
			\end{itemize}
		\end{minipage}
	};
%------------ Borel sigma algebra Header ---------------------
	\node[fancytitle, right=10pt] at (box.north west) {Borel $ \sigma- $algebra};
\end{tikzpicture}

%------------ Pointwise convergence and measurability ---------------
\begin{tikzpicture}
	\node [mybox] (box){%
		\begin{minipage}{0.3\textwidth}
			\begin{itemize}
				\item If $ f_n:X \to [-\infty, \infty] $ is measurable for all $ n \in \N $ then $ \sup, \inf, \limsup, \liminf $ of $ f_n $ are also measurable.
				\item o the limit of every pointwise convergent sequence of measurable functions is measurable.
				\item If $ f $ is measurable then so is $ f^+=\max\{f,0\}, f^{-1}=-\min\{f,0\}$
			\end{itemize}
		\end{minipage}
	};
	%------------ Pointwise convergence and measurability Header ---------------------
	\node[fancytitle, right=10pt] at (box.north west) {Pointwise convergence and measurability};
\end{tikzpicture}

%------------ Simple functions ---------------
\begin{tikzpicture}
	\node [mybox] (box){%
		\begin{minipage}{0.3\textwidth}
			A complex function whose range consists of only finitely many points. If $ \alpha_1 \dots, \alpha_n $ are the distinct values of the simple function $ s $ and $ A_i={x:s(x)=\alpha_i } $ then \[ s=\sum_{i=1}^n \alpha_i \chi_{A_i} \]
			\begin{itemize}
				\item Every measurable function $ f:X \to [0,\infty] $ can be written as a pointwise limit of a sequence of simple functions.
			\end{itemize}
		\end{minipage}
	};
%------------ Simple functions Header ---------------------
	\node[fancytitle, right=10pt] at (box.north west) {Simple functions};
\end{tikzpicture}

%------------ Positive measure ---------------
\begin{tikzpicture}
	\node [mybox] (box){%
		\begin{minipage}{0.3\textwidth}
			A \textbf{positive measure} $ \mu  $ is a measure along with the following additional properties,
			\begin{itemize}
				\item Its range is in $ [0,\infty] $
				\item Countable additivity: $ \mu(\bigcup_{i=1}^\infty A_i)=\sum_{i=1}^\infty \mu(A_i) $
			\end{itemize}
		A \textit{measure} space refers to a measurable space with a positive measure.
		\end{minipage}
	};
%------------ Positive measure Header ---------------------
	\node[fancytitle, right=10pt] at (box.north west) {Positive measure};
\end{tikzpicture}

%------------ Arithmetic in [0,infty]---------------
\begin{tikzpicture}
	\node [mybox] (box){%
		\begin{minipage}{0.3\textwidth}
			We understand $ a+\infty=\infty  $ for $ 0\leq a \leq \infty $ and $ a \cdot \infty = 
				\infty \text{ if } 0 < a \leq \infty $ else $ 0 $.
		\end{minipage}
	};
	%------------ Arithmetic in [0,infty] Header ---------------------
	\node[fancytitle, right=10pt] at (box.north west) {Arithmetic in $
		[0,\infty]$};
\end{tikzpicture}

%------------ Lebesgue Integral ---------------
\begin{tikzpicture}
	\node [mybox] (box){%
		\begin{minipage}{0.3\textwidth}
			If $ X $ is a set with $ \sigma-$algebra $ F $ and positive measure $ \mu $. Then for a measurable simple function $ s:X \to [0,\infty] $ as defined previously, its Lebesgue integral over $ E \in F $ is defined as follows \[ \int_E s\, d\mu=\sum_{i=1}^n \alpha_i \mu \left(A_i \bigcap E\right) \]
			
		\end{minipage}
	};
%------------ Lebesgue integral Header ---------------------
	\node[fancytitle, right=10pt] at (box.north west) {Lebesgue integral};
\end{tikzpicture}

%------------ Lebesgue integrable functions ---------------
\begin{tikzpicture}
	\node [mybox] (box){%
		\begin{minipage}{0.3\textwidth}
			Define $ L^1(\mu) $ to be the collection of all complex measurable functions $ f $ on $ X $ for which $ \int_X |f| \,d \mu < \infty $ known as the Lebesgue integrable functions.
			
			For functions $ f $ with range in $ [-\infty, \infty] $ define $ \int_E f \,d \mu= \int_E f^+ \,d \mu - \int_E f^{-} \,d \mu $
		\end{minipage}
	};
	%------------ Lebesgue integrable functions Header ---------------------
	\node[fancytitle, right=10pt] at (box.north west) {Lebesgue integrable functions};
\end{tikzpicture}

%------------ Zero measure ---------------
\begin{tikzpicture}
	\node [mybox] (box){%
		\begin{minipage}{0.3\textwidth}
			We say a property holds ``almost everywhere (a.e.)'' if it holds everywhere except on a set of measure zero.
			
			If any two function $ f=g $ a.e. then their Lebesgue integrals are the same. Set of measure zero don't impact the value of the Lebesgue integral
		\end{minipage}
	};
	%------------ Zero measure Header ---------------------
	\node[fancytitle, right=10pt] at (box.north west) {Zero measure};
\end{tikzpicture}

%------------ Monotone convergence theorem ---------------
\begin{tikzpicture}
	\node [mybox] (box){%
		\begin{minipage}{0.3\textwidth}
			Let $ \{f_n\} $ be a sequence of measurable functions on $ X $ if $ 0\leq f_1(x)\leq f_2(x)\leq \dots \leq \infty $ a.e. and $ f_n(x)\to f(x) $ as $ n \to \infty $ a.e., then $ f $ is measurable and
			\[ \lim_{n\to \infty} \int_X f_n\, d \mu = \int_X \lim_{n \to \infty} f_n\, d \mu  \]
			
			
		\end{minipage}
	};

%------------ Monotone convergence theorem Header ---------------------
	\node[fancytitle, right=10pt] at (box.north west) {Monotone convergence theorem};
\end{tikzpicture}
%------------ Consequences of MCT ---------------
\begin{tikzpicture}
	\node [mybox] (box){%
		\begin{minipage}{0.3\textwidth}
			\begin{itemize}
				\item Applying MCT to sequence of partial sums of a convergent series $ f(x)=\sum_{n=1}^\infty f_n(x) $ we get, \[ \int_X f \,d \mu = \sum_{n=1}^\infty \int_X f_n \,d \mu \]
				\item If $ f:X\to[0,\infty] $ is measurable with $ \sigma- $algebra $ F $ of $ X $ and $ \phi(E)=\int_E f \,d \mu $ for $ E \subset F$ then, $ \phi  $ is a measure on $ F $ and \[ \int_X g \,d \phi = \int_X gf \,d \mu \] for every measurable $ g:X \to [0,\infty]$.
			\end{itemize}
		\end{minipage}
	};
	%------------ Consequences of MCT Header ---------------------
	\node[fancytitle, right=10pt] at (box.north west) {Consequences of MCT};
\end{tikzpicture}
%------------ Fatou's lemma ---------------
\begin{tikzpicture}
	\node [mybox] (box){%
		\begin{minipage}{0.3\textwidth}
			If $ f_n:X \to [0,\infty] $ is measurable for all $ n \in \N  $ then,\[ \int_X \liminf_{n \to \infty}f_n \, d \mu \leq \liminf_{n \to \infty} \int_X f_n \, d \mu \]
		
		\end{minipage}
	};
%------------ Fatou's lemma Header ---------------------
	\node[fancytitle, right=10pt] at (box.north west) {Fatou's lemma};
\end{tikzpicture}

%------------ Dominated convergence theorem ---------------
\begin{tikzpicture}
	\node [mybox] (box){%
		\begin{minipage}{0.3\textwidth}
			If $ \{f_n\} $ us a sequence of complex measurable functions on $ X $ with $ \lim_{n\to \infty} f_n(x)=f(x) $ pointwise. If there exists a function $ g \in L^1(\mu) $ such that $ |f_n(x)| \leq g(x) $ for all $ n \in N$ and $ x \in X $ then $ f\in L^(1)(\mu) $ and \[ \lim_{n\to \infty} \int_X f \,d \mu=\int_X \lim_{n \to \infty} f_n(x) \,d \mu \]
		\end{minipage}
	};
%------------ Dominated convergence theorem Header ---------------------
	\node[fancytitle, right=10pt] at (box.north west) {Dominated convergence theorem};
\end{tikzpicture}
%------------ Compact support ---------------
\begin{tikzpicture}
	\node [mybox] (box){%
		\begin{minipage}{0.3\textwidth}
			A function has compact support if it is zero outside of a compact set, i.e. $ f\in C_c(X) $ if $ f $ has compact support on $ X $.
		\end{minipage}
	};
	%------------ Compact support Header ---------------------
	\node[fancytitle, right=10pt] at (box.north west) {Compact support};
\end{tikzpicture}

%------------ Locally compact Hausdorff spaces ---------------
\begin{tikzpicture}
	\node [mybox] (box){%
		\begin{minipage}{0.3\textwidth}
			
		\end{minipage}
	};
	%------------ Locally compact Hausdorff spaces Header ---------------------
	\node[fancytitle, right=10pt] at (box.north west) {Locally compact Hausdorff spaces};
\end{tikzpicture}


%%------------ Caratheodory's extension theorem ---------------
%\begin{tikzpicture}
%	\node [mybox] (box){%
%		\begin{minipage}{0.3\textwidth}
%			
%		\end{minipage}
%	};
%%------------ Caratheodory's extension theorem Header ---------------------
%	\node[fancytitle, right=10pt] at (box.north west) {Caratheodory's extension theorem};
%\end{tikzpicture}

%------------ Riesz representation theorem ---------------
\begin{tikzpicture}
	\node [mybox] (box){%
		\begin{minipage}{0.3\textwidth}
			Let $ X $ is a locally compact Hausdorff space and $ T $ is a positive linear functional on $ C_c(X) $. Then there exists a $ \sigma- $algebra $ F $ of $ X $ which contains all Borel sets in $ X $ and there exists a unique positive measure $ \mu  $ on $ F $ such that for every $ f\in C_c(X) $\[ Tf=\int_X f \,d\mu \]
			additionally the following properties hold,
			\begin{itemize}
				\item $ \mu(K)<\infty  $ for every compact $ K\subset X $
				\item $ \mu(E)=\inf\{\mu(V): E\subset V, V \text{open} \} $ for all $ E \in F $.
				\item $ \mu(E)=\sup\{\mu(K): K \subset E, K \text{compact} \} $ holds for every open set $ E \subset F $ with $ \mu(E)<\infty $
				\item If $ E \in F $, $ A\subset E $ and $ \mu(E)=0 $ then $ A \in F $.
			\end{itemize}
		\end{minipage}
	};
%------------ Riesz representation theorem Header ---------------------
	\node[fancytitle, right=10pt] at (box.north west) {Riesz representation theorem};
\end{tikzpicture}

%------------ Borel measures ---------------
\begin{tikzpicture}
	\node [mybox] (box){%
		\begin{minipage}{0.3\textwidth}
			
		\end{minipage}
	};
%------------ Borel measures Header ---------------------
	\node[fancytitle, right=10pt] at (box.north west) {Borel measures};
\end{tikzpicture}

%------------ Lebesgue measure ---------------
\begin{tikzpicture}
	\node [mybox] (box){%
		\begin{minipage}{0.3\textwidth}
			
		\end{minipage}
	};
%------------ Lebesgue measure Header ---------------------
	\node[fancytitle, right=10pt] at (box.north west) {Lebesgue measure};
\end{tikzpicture}

%------------ Convexity ---------------
\begin{tikzpicture}
	\node [mybox] (box){%
		\begin{minipage}{0.3\textwidth}
			
		\end{minipage}
	};
%------------ Convexity Header ---------------------
	\node[fancytitle, right=10pt] at (box.north west) {Convexity};
\end{tikzpicture}

%------------ Jensen's inequality ---------------
\begin{tikzpicture}
	\node [mybox] (box){%
		\begin{minipage}{0.3\textwidth}
			
		\end{minipage}
	};
%------------ Jensen's inequality Header ---------------------
	\node[fancytitle, right=10pt] at (box.north west) {Jensen's inequality};
\end{tikzpicture}

%------------ Holder's inequality ---------------
\begin{tikzpicture}
	\node [mybox] (box){%
		\begin{minipage}{0.3\textwidth}
			
		\end{minipage}
	};
%------------ Holder's inequality Header ---------------------
	\node[fancytitle, right=10pt] at (box.north west) {H\"{o}lder's inequality};
\end{tikzpicture}


%------------ Minkowski's inequality ---------------
\begin{tikzpicture}
	\node [mybox] (box){%
		\begin{minipage}{0.3\textwidth}
			
		\end{minipage}
	};
%------------ Minkowski's inequality Header ---------------------
	\node[fancytitle, right=10pt] at (box.north west) {Minkowski's inequality};
\end{tikzpicture}

%------------ Lp norms ---------------
\begin{tikzpicture}
	\node [mybox] (box){%
		\begin{minipage}{0.3\textwidth}
			
		\end{minipage}
	};
%------------ Lp norms Header ---------------------
	\node[fancytitle, right=10pt] at (box.north west) {$ L^p $ norms};
\end{tikzpicture}

%------------ Total variation ---------------
\begin{tikzpicture}
	\node [mybox] (box){%
		\begin{minipage}{0.3\textwidth}
			
		\end{minipage}
	};
%------------ Total variation Header ---------------------
	\node[fancytitle, right=10pt] at (box.north west) {Total variation};
\end{tikzpicture}

%------------ Complex measure ---------------
\begin{tikzpicture}
	\node [mybox] (box){%
		\begin{minipage}{0.3\textwidth}
			
		\end{minipage}
	};
%------------ Complex measure Header ---------------------
	\node[fancytitle, right=10pt] at (box.north west) {Complex measure};
\end{tikzpicture}

%------------ Absolutely continuity and mutually singular ---------------
\begin{tikzpicture}
	\node [mybox] (box){%
		\begin{minipage}{0.3\textwidth}
			
		\end{minipage}
	};
%------------ Absolutely continuity and mutually singular Header ---------------------
	\node[fancytitle, right=10pt] at (box.north west) {Absolutely continuity and mutually singular};
\end{tikzpicture}

%------------ Lebesgue-Radon-Nikodym theorem ---------------
\begin{tikzpicture}
	\node [mybox] (box){%
		\begin{minipage}{0.3\textwidth}
			
		\end{minipage}
	};
%------------ Lebesgue-Radon-Nikodym theorem Header ---------------------
	\node[fancytitle, right=10pt] at (box.north west) {Lebesgue-Radon-Nikodym theorem};
\end{tikzpicture}


%------------ Product measures ---------------
\begin{tikzpicture}
	\node [mybox] (box){%
		\begin{minipage}{0.3\textwidth}
			
		\end{minipage}
	};
%------------ Product measures Header ---------------------
	\node[fancytitle, right=10pt] at (box.north west) {Product measures};
\end{tikzpicture}
% ==============================================================
\end{multicols*}
\end{document}
