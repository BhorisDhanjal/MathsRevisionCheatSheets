\documentclass[dvipsnames]{article}
\usepackage{geometry}
\geometry{a3paper, landscape, margin=2in}
\usepackage{url}
\usepackage{multicol}
\usepackage{esint}
\usepackage{amsfonts}
\usepackage{tikz}
\usetikzlibrary{decorations.pathmorphing}
\usepackage{amsmath,amssymb}
\usepackage{enumitem}
\usepackage{colortbl}
\usepackage{mathtools}
\usepackage{ mathrsfs }	
\usepackage{ dsfont }
\usepackage{ gensymb }

\usepackage[activate={true,nocompatibility},final,tracking=true,kerning=true,spacing=true,factor=1100,stretch=10,shrink=10]{microtype}
\makeatletter
\setlist[itemize]{noitemsep, topsep=0pt}
%Template by Drew Ulick https://de.overleaf.com/articles/130-cheat-sheet/ntwtkmpxmgrp
\usepackage[english]{babel}
\usepackage{lmodern}
	
\advance\topmargin-1.2in
\advance\textheight3in
\advance\textwidth3in
\advance\oddsidemargin-1.5in
\advance\evensidemargin-1.5in
\parindent0pt
\parskip2pt
\newcommand{\hr}{\centerline{\rule{3.5in}{1pt}}}
%\colorbox[HTML]{e4e4e4}{\makebox[\textwidth-2\fboxsep][l]{texto}
\begin{document}

\begin{center}{\fontsize{35}{60}{\textcolor{purple}{\textbf{Group Theory Cheat Sheet}}}}\\
\end{center}
\begin{multicols*}{3}

\tikzstyle{mybox} = [draw=purple, fill=white, very thick,
    rectangle, rounded corners, inner sep=10pt, inner ysep=10pt]
\tikzstyle{fancytitle} =[fill=purple, text=white, font=\bfseries]

%------------ Group Definitions ---------------
\begin{tikzpicture}
\node [mybox] (box){%
    \begin{minipage}{0.3\textwidth}
    A group is an ordered pair $(G, *)$ where $G$ is a set and $*$ is a binary operation on $G$ satisfying the following axioms:
    \begin{itemize}
    	\item \textbf{Associativity:} $(a * b) * c = a * (b * c) \forall a, b, c \in G$
    	\item \textbf{Identity:} $\exists e \in G$, called an identity element of $G$, s.t. $\forall a \in G$ we have $a*e = e * a = a$.
    	\item \textbf{Inverse} $\forall a  \in G \exists \ a^{-1} \in G$, called an inverse of $a$, s.t. $a *a^{-1} = a^{-1} * a = e.$
    \end{itemize}
    Closure is guaranteed due to the definition of binary operation.
    
    Identity and inverses are unique.
    \end{minipage}
};
%------------ Group Definitons Header ---------------------
\node[fancytitle, right=10pt] at (box.north west) {Group Axioms};
\end{tikzpicture}
%
%%------------ Some Properties of Groups ---------------
%\begin{tikzpicture}
%\node [mybox] (box){%
%    \begin{minipage}{0.3\textwidth}
%    \begin{itemize}
%    	\item A group G is called \textbf{abelian} if $a * b = b * a \forall a, b \in G$
%    \end{itemize}
%    \textbf{iii. Cancellation property} suppose that a * b = a * c,  $\forall$ a, b, c $\in$ G, $\Rightarrow$ b = c\\
%    \textbf{iv. Uniqueness of Inverse and Identity}\\ • The identity of G is unique\\
%    • $\forall$ a $\in$ G, $a^{-1}$ is uniquely determined\\
%    • $(a^{-1})^{-1}$ = a $\forall$ a $\in$ G\\
%    • $(a * b)^{-1}$ = $(b^{-1})\ *\ (a^{-1})$\\
%    • for any $a_1 , a_2, ... , a_n$ $\in$ G the value of $a_1 * a_2 * · · · *a_n$ is independent of how the expression is bracketed
%    \end{minipage}
%};
%%------------ Mixing Header ---------------------
%\node[fancytitle, right=10pt] at (box.north west) {Some Properties of Groups};
%\end{tikzpicture}

%------------ Some special groups ---------------
\begin{tikzpicture}
\node [mybox] (box){%
    \begin{minipage}{0.3\textwidth}
    	\begin{itemize}
    		\item A group is called \textbf{abelian} if it is commutative.
    		\item The group of all symmetries of a $n-$sided regular polygon is called the \textbf{dihedral group}. It is represented as, \[ D_{2n}= \langle r,s \mid r^{n}=s^{2}=1, rs=sr^{-1} \rangle. \] 
    		\item The group of all bijections on a set on $n$ elements is called the \textbf{symmetric} group denoted as $S_n$.
    		\item The \textbf{Klein-4} group is a group with 4 elements in which each element is a self inverse.
    	\end{itemize}
    \end{minipage}
};
%------------ Some Special Groups Space Header ---------------------
\node[fancytitle, right=10pt] at (box.north west) {Some Special Groups};
\end{tikzpicture}

%------------ Homomorphisms and Isomorphisms ---------------
\begin{tikzpicture}
\node [mybox] (box){%
    \begin{minipage}{0.3\textwidth}
    Let $(G , *)$ and $(H, \circ)$ be groups. A map $\varphi : G \to H,\ s.t.\ \varphi(x * y) = \varphi(x) \circ \varphi(y)\ \forall\ x, y \in G$ is called a \textbf{homomorphism.}\\
    A bijective homomorphism is called an \textbf{isomorphism}.

    \end{minipage}
};
%------------ Homomorphisms and Isomoprhisms Header ---------------------
\node[fancytitle, right=10pt] at (box.north west) {Homomorphisms and Isomorphisms};
\end{tikzpicture}

%------------ Subgroups ---------------------
\begin{tikzpicture}
	\node [mybox] (box){%
		\begin{minipage}{0.3\textwidth}
			For a Group $G$ a subset $H \subseteq G$, is a \textbf{subgroup} of $G$, i.e. $H \leq G$ if it is non empty and a group with the binary operation of $G$ restricted to $H$.
			
			Alternatively, a subset of a group is a subgroup if it is non empty and closed under products and inverses, i.e. for $H \subseteq G$
			\begin{itemize}
				\item $H \neq \emptyset$
				\item $xy^{-1} \in H, \forall x,y \in H$
			\end{itemize}
			
			A subgroup $N$ of $ G$ is called \textbf{normal}, denoted as $N \trianglelefteq G$ if $gng^{-1} \in N, \forall\ g \in G,  n \in N.$
			
		\end{minipage}
	};
	%------------ Subgroups Header ---------------------
	\node[fancytitle, right=10pt] at (box.north west) {Subgroups};
\end{tikzpicture}

%------------ Group Actions ---------------
\begin{tikzpicture}
\node [mybox] (box){%
    \begin{minipage}{0.3\textwidth}
    A \textbf{group action} of a group $(G,*)$ on a set $X$ is a map $\circ:G \times X \to X$ satisfying the following properties,
    \begin{itemize}
    	\item \textbf{Identity:} $e \circ x = x, \forall x \in X$
    	\item \textbf{Compatibility:} $g \circ (h \circ x)=g*h \circ x, \forall g,h \in G, x \in X$
    \end{itemize}
    Alternatively a group action on a set $X$ can be thought of as a homomorphism from $G$ to the symmetric group of $X$.
    
    The \textbf{conjugation action} is a homomorphism $\varphi_x: G \to G$ for some fixed $x \in G$ is defined as $\varphi_x(g)=xgx^{-1}$.

    \end{minipage}
};
%------------ Group Actions Header ---------------------
\node[fancytitle, right=10pt] at (box.north west) {Group Actions};
\end{tikzpicture}


%------------ Stablizer, Kernel of Group Action ---------------------
\begin{tikzpicture}
	\node [mybox] (box){%
		\begin{minipage}{0.3\textwidth}
			If $G$ acts on a set $X$ then we define the following,
			
			\textbf{Orbit} of a $x \in X$ is the set $Gx=\{gx \in X \mid g \in G\}$. Alternatively its the equivalence class induced by the group action.
			
			\textbf{Stablizer} of an element $x \in X$ is the set of elements from the group which leave $x$ fixed, i.e. $G_x=\{g \in G \mid gx=x\}$
			
			\textbf{Kernel} of a group action is the kernel of the associated group homomorphism i.e., $\{g \in G \mid gx=x, \forall x \in X\}$ 
			
			An action is called \textbf{transitive} if it has only one orbit, and is called \textbf{faithful} if its kernel is trivial.
			
			\textbf{Conjugacy classes} of $G$ is the equivalence classes of G when it acts on itself with conjugation. i.e. $\{gag^{-1}\ |\ g \in G \}$
		\end{minipage}
	};
	%------------ Stabl4izer, Kernel of Group Action Header ---------------------
	\node[fancytitle, right=10pt] at (box.north west) {Stablizer, Kernel of Group Action};
\end{tikzpicture}

%------------ Centralizers and Normalizers Content ---------------
\begin{tikzpicture}
\node [mybox] (box){%
    \begin{minipage}{0.3\textwidth}
    \textbf{Centralizer} of $A \subseteq G$ in $G$ is a subset of $G$ defined as $$C_G(A) = \{ g \in G\ |\ gag^{-1} = a\ \forall\ a \in A \}.$$ It is the set of all elements of $G$ which commute with every element of $A$.
    
    \textbf{Center} of $G$ is the subset of G defined as $$Z(G) = \{ g \in G\ |\ gx = xg\ \forall\ x \in G \}.$$ It is the set of elements commutating with all the elements of G. It is the kernel of the conjugation action.
    
    \textbf{Normalizer} of $A$ in $G$ is defined as
    $$N_G(A)=\{ g \in G\ |\ gAg^{-1} = A\}$$ 
    $gAg^{-1}=\{gag^{-1}\ |\ a \in A\}$. Note that $C_G(A) \leq N_G(A).$
   The normalizer of a subset is its stabilizer under conjugation action.
    \end{minipage}
};
%------------ Centralizers and Normalizers Header ---------------------
\node[fancytitle, right=10pt] at (box.north west) {Centralizers and Normalizers};
\end{tikzpicture}


%------------ Cyclic Groups and Cycle Notation Content ---------------
\begin{tikzpicture}
\node [mybox] (box){%
    \begin{minipage}{0.3\textwidth}
    A Group $H$ is \textbf{cyclic} if $\exists\ x \in H\ \text{s.t. } H = \{ x^n\ |\ n \in \mathbb{Z} \}$\\
    For the above case we say $H =\left < x \right >$ and that $H$ is generated by $x$.
    \begin{itemize}
    	\item A cyclic group can have more than one generator.
    	\item All cyclic groups are abelian.
    	\item If $H=\left<x\right>$ then $|H| = |x|,$ if $|H|=n<\infty$ then $x^n=1$
    	\item Any two cyclic groups of the same order are isomorphic.\
    \end{itemize}
   \hrule
   \vspace{0.2cm}
   Two-Line to Cycle notation for permutations\\
   $\left(\begin{matrix}
1 & 2 & 3 & 4 & 5\\ 
2 & 5 & 4 & 3 & 1
\end{matrix}\right)=(125)(34)=(34)(125)=(34)(512)=(15)(25)(34)$\\
Here, the last form is a case of 2-cycle (transposition).
    \end{minipage}
};
%------------ Cyclic Groups Header ---------------------
\node[fancytitle, right=10pt] at (box.north west) {Cyclic Groups and Cycle Notation};
\end{tikzpicture}

%------------ Parity of Permutations and Alternating Groups Content ---------------------
\begin{tikzpicture}
	\node [mybox] (box){%
		\begin{minipage}{0.3\textwidth}
			The parity of any permutation $\sigma$ is given by the parity of the number of its 2-cycles (transpositions).
			\hrule
			\vspace{0.2cm}
			\textbf{Alternating Groups}:\\
			An alternating group is the group of even permutations of a finite set of length $n$. It is denoted by $A_n$ it's order is $\frac{n!}{2}$
			
		\end{minipage}
	};
	%------------ Parity of Permutations and Alternating Groups Header ---------------------
	\node[fancytitle, right=10pt] at (box.north west) {Parity of Permutations and Alternating Groups};
\end{tikzpicture}


%------------ Cosets and Quotient Groups Content ---------------
\begin{tikzpicture}
\node [mybox] (box){%
    \begin{minipage}{0.3\textwidth}
    For any $N\leq G$ and any $g \in G$\\
    • $gN=\{gn\ |\ n\in N\}=\{g, gh_1, gh_2 \dots \}$ and,\\
    • $Ng=\{ng\ |\ n\in N\}=\{g, h_1g, h_2g \dots \}$
    are called a left coset and a right coset respectively.\\
    \hrule
    \vspace{0.2cm}
    For a Group $G\ \text{and}\ N \trianglelefteq G$, the \textbf{quotient group} of N in G (i.e. G/N), is the set of cosets of N in G.
    
    The definition of a normal subgroup is the same as left and right cosets being equal.
    
    \end{minipage}
};
%------------ Cosets and Quotient Groups Header ---------------------
\node[fancytitle, right=10pt] at (box.north west) {Cosets and Quotient Groups};
\end{tikzpicture}

%------------ Lagrange's Theorem and some results ---------------
\begin{tikzpicture}
\node [mybox] (box){%
    \begin{minipage}{0.3\textwidth}
    \textbf{Lagrange's Theorem}: For a finite group $G$ and $H \leq G$,\\
    • The order of $H$ divides the order of $G$, and,\\
    • The number of left cosets of $H$ in $G$ equals $\frac{|G|}{|H|}$\\
    \hrule
    \vspace{0.2cm}
    \textbf{Some important results}\\
    • If $G$ is a finite group and $x \in G$, then the order of $x$ divides the order of $G$, and $x^{|G|}=e\ \forall\ x \in G$\\
    • If $G$ is a group of prime order, then $G$ is cyclic
    \end{minipage}
};
%------------ Lagrange's Theorem and some results ---------------------
\node[fancytitle, right=10pt] at (box.north west) {Lagrange's Theorem and some results};
\end{tikzpicture}
\
%------------ Cauchy's Theorem Content ---------------
\begin{tikzpicture}
\node [mybox] (box){%
    \begin{minipage}{0.3\textwidth}
    \textbf{Cauchy's Theorem}: If $G$ is a finite group and $p$ is a prime dividing $|G|$ then $G$ has an element of order $p$.
    \end{minipage}
};
%------------ Cauchy's Theorems Header ---------------------
\node[fancytitle, right=10pt] at (box.north west) {Cauchy's Theorem};
\end{tikzpicture}
%------------ The Isomorphism Theorems Content ---------------------
\begin{tikzpicture}
\node [mybox] (box){%
    \begin{minipage}{0.3\textwidth}
	\textbf{First Isomorphism Theorem:}\\
	If $\varphi:G \to H$ is a homomorphism of groups. Then $\mathrm{ker} \varphi \trianglelefteq G$ and,
	$G/\mathrm{ker}\varphi \cong \varphi(G)$.
	
	\textbf{Second Isomorphism Theorem:}\\
	For a group $G$ with, $A, B \leq G$ and, $A \trianglelefteq N_G(B)$.
Then $AB \leq G, \\ B\trianglelefteq AB, A \cap B \trianglelefteq A$ and, $AB/B \cong A/ A \cap B$\\
	\textbf{The Third Isomoprhism Theorem:}\\
	For a group $G$ with, $H, K \trianglelefteq G$ and, $H\leq K$. Then $K/H \trianglelefteq G/H$ and, $\frac{G/H}{K/H}\cong G/K$
	\end{minipage}
};
%------------ The Isomorphism Theorems Header ---------------------
\node[fancytitle, right=10pt] at (box.north west) {The Isomorphism Theorems};
\end{tikzpicture}




%------------ Class equations and Orbit-stabilizer Theorem Content ---------------
\begin{tikzpicture}
\node [mybox] (box){%
    \begin{minipage}{0.3\textwidth}
    \textbf{Class equation} of a finite group $G$ is written as:\\
    $|G|=|Z(G)|+|\sum(\text{Conjugancy classes of G})|$\\
    \textbf{Oribit-stabilizer Theorem:}\\
    For a group $G$ acting on a set $X$, for any $x\ \in\ X$ we have,  $|Gx||G_x|=|G|$
    
    \end{minipage}
};
%------------ Class equations and Orbit-stabilizer Theorem Header ---------------------
'\node[fancytitle, right=10pt] at (box.north west) {Class equations and Orbit-stabilizer Theorem};
\end{tikzpicture}

%------------ Cayley's Theorem Content ---------------
\begin{tikzpicture}
\node [mybox] (box){%
    \begin{minipage}{0.3\textwidth}
    \textbf{Cayley's Theorem:}\\
    Every group is isomorphic to a subgroup of some symmetric group. If $G$ is a group of order $n$, then G is isomoprhic to a subgroup of $S_n$
    \end{minipage}
};
%------------ Cayley's Theorem Header ---------------------
'\node[fancytitle, right=10pt] at (box.north west) {Cayley's Theorem};
\end{tikzpicture}

%------------ Automorphisms Content ---------------
\begin{tikzpicture}
\node [mybox] (box){%
    \begin{minipage}{0.3\textwidth}
    \textbf{Automorphism} of $G$ is defined as an isomorphism from $G$ onto itself.\\
    The set of all automorphisms of $G$ is denoted by Aut($G$)
    \end{minipage}
};
%------------ Automorphisms Header ---------------------
'\node[fancytitle, right=10pt] at (box.north west) {Automorphisms};
\end{tikzpicture}

%------------ p-groups and Sylow p-groups Content ---------------
\begin{tikzpicture}
\node [mybox] (box){%
    \begin{minipage}{0.3\textwidth}
    \textbf{• p-group} is defined as a group of order $p^a$ for some $a \geq 1$. Sub-groups of $G$ which are p-groups are called p-subgroups.\\
    \textbf{• Sylow p-group} is defined as a group of order $p^am$, where $p \nmid m$, a subgroup of order $p^a$ is called a Sylow p-subgroup of $G$. $Syl_p(G)$ is the set of Sylow p-subgroups of $G$.
    \end{minipage}
};
%------------ p-groups and Sylow p-groups Header ---------------------
'\node[fancytitle, right=10pt] at (box.north west) {p-groups and Sylow p-groups};
\end{tikzpicture}


%------------ The Sylow Theorems Content ---------------
\begin{tikzpicture}
\node [mybox] (box){%
    \begin{minipage}{0.3\textwidth}
    \textbf{First Sylow Theorem:}\\
    If $p$ divides $|G|$, then $G$ has a Sylow p-subgroup.\\
    \textbf{Second Sylow Theorem:}\\
    All Sylow p-subgroups of $G$ are conjugate to each other for a fixed $p$.\\
    \textbf{Third Sylow Theorem:}\\
    $n_p \equiv 1 (\text{mod}\ p)$, where $n_p$ is the number of Sylow p-subgroups of G.
    \end{minipage}
};
%------------ The Sylow Theorems Header ---------------------
'\node[fancytitle, right=10pt] at (box.north west) {The Sylow Theorems};
\end{tikzpicture}

%------------ Commutators Content ---------------
\begin{tikzpicture}
	\node [mybox] (box){%
		\begin{minipage}{0.3\textwidth}
			For a group $G$ and $x,y \in G$ call $[x,y]=x^{-1}y^{-1}xy$ the \textbf{commutator} of $x$ and $y$.
			
			For subsets $A,B \subseteq G$ define the group generated by its commutators as $[A,B]=\langle [a,b] \mid a\in A, b \in B\rangle$.
			
			Similarly $G'=[G,G]$ is the subgroup of $G$ generated by all commutators, its called the commutator subgroup of $G$, or its \textbf{derived subgroup}.	
			
			The following are some useful properties of commutators,
			\begin{itemize}
				\item $xy=yx \iff [x,y]=1$
				\item $H \trianglelefteq G \iff [H,G] \leq H$
				\item $G/G'$ is abelian and its the largest abelian quotient of $G.$ It is called the \textbf{abelianization} of $G$.
				\item Any homomorphism from $G$ to an abelian group $A$ factors through $G'$. i.e. its universal 
			\end{itemize}
		\end{minipage}
	};
	%------------ Commutators Header ---------------------
	'\node[fancytitle, right=10pt] at (box.north west) {Commutators};
	
\end{tikzpicture}

%------------ Direct products Content ---------------
\begin{tikzpicture}
	\node [mybox] (box){%
		\begin{minipage}{0.3\textwidth}
			The \textbf{direct product} $G_1 \times G_2 \times \cdots $of groups $G_1, G_2, \dots$ is set of sequences $(g_1,g_2, \cdots)$ with $g_i \in G_i$ and operation $*$ defined component wise.
			
			If $H, K$ are normal subgroups of $G$ with $H \cap K =1$ then $HK \cong H \times K$
		\end{minipage}
	};
	%------------ Direct products Header ---------------------
	'\node[fancytitle, right=10pt] at (box.north west) {Direct products};
	
\end{tikzpicture}

%------------ Semidirect products Content ---------------
\begin{tikzpicture}
	\node [mybox] (box){%
		\begin{minipage}{0.3\textwidth}
			Let $H, K$ be groups and $\varphi $ be a homomorphism from $K \to \textrm{Aut}(H)$. The semidirect product of $H $ and $K$ with respect to $\varphi $ denoted as $H \rtimes_\varphi H$ is defined as an ordered pair $(h,k), h \in H, k \in K$ with multiplication defined as $(h_1,k_1)(h_2,k_2)=(h_1k_1h_2, k_1k_2).$
		\end{minipage}
	};
	%------------ Semidirect products Header ---------------------
	'\node[fancytitle, right=10pt] at (box.north west) {Semidirect products};
	
\end{tikzpicture}


%------------ Fundamental theorem of finite abelian groups Content ---------------
\begin{tikzpicture}
	\node [mybox] (box){%
		\begin{minipage}{0.3\textwidth}
			Every finite abelian group can be written as a direct product of cyclic groups of prime power order.
		\end{minipage}
	};
%------------ Fundamental theorem of finite abelian groups Header ---------------------
	'\node[fancytitle, right=10pt] at (box.north west) {Fundamental theorem of finite abelian groups};
\end{tikzpicture}

%------------ Schur-Zassenhaus Theorem Content ---------------
\begin{tikzpicture}
	\node [mybox] (box){%
		\begin{minipage}{0.3\textwidth}
			If $G$ is a finite group and $N \trianglelefteq G	$ and $(|G|, |G/N|)=1$ then $G=N \rtimes G/N$
		\end{minipage}
	};
	%------------ Schur-Zassenhaus Theorem Header ---------------------
	'\node[fancytitle, right=10pt] at (box.north west) {Schur-Zassenhaus Theorem};
	
\end{tikzpicture}

%------------ Simple groups and Composition Series Content ---------------
\begin{tikzpicture}
	\node [mybox] (box){%
		\begin{minipage}{0.3\textwidth}
			A group $G$ is called \textbf{simple} if $|G|>1$ and its only normal subgroups are $1$ and $G$.
			
			A sequence of subgroups as follows (called a subnormal sequence),\[ 1=N_0 \trianglelefteq N_1 \trianglelefteq \cdots \trianglelefteq N_{k-1} \leq N_k = G \] is called a \textbf{composition series} if $N_{i+1}/N_i$ is simple, for $i=0,\dots k-1$. The quotient groups are called composition factors.
		\end{minipage}
	};
	%------------ Simple groups and Composition Series Header ---------------------
	'\node[fancytitle, right=10pt] at (box.north west) {Simple groups and Composition Series};
\end{tikzpicture}

%------------ Jordan H\"{o}lder theorem Content ---------------
\begin{tikzpicture}
	\node [mybox] (box){%
		\begin{minipage}{0.3\textwidth}
			Every finite non trivial group $G$ has a composition series and the composition factors are unique up to permutation.
		\end{minipage}
	};
	%------------ Jordan H\"{o}lder theorem Header ---------------------
	'\node[fancytitle, right=10pt] at (box.north west) {Jordan H\"{o}lder theorem};
\end{tikzpicture}


%------------ Solvable groups Content ---------------
\begin{tikzpicture}
	\node [mybox] (box){%
		\begin{minipage}{0.3\textwidth}
			A group $G$ is called \textbf{solvable} if there is a chain of normal subgroups with each subsequent quotient group being abelian.
			
			Alternatively if there exists $G^{(n)}=1$ for some $n\geq 0$ where we define the \textbf{commutator/derived series} as follows,
			\[ G^{(0)}=G, G^{(1)}=[G,G], \text{ and } G^{(i+1)}=[G^{(i)},G^{(i)}]\]
			
			\begin{itemize}
				\item If $N \trianglelefteq G$ and $N, G/N$ are solvable then $G$ is solvable.
				\item Burnsides theorem: If $|G|=p^aq^b$ for primes $p,q$ then $G$ is solvable
				\item Feit-Thompson: All finite groups of odd order are solvable.
			\end{itemize}
			
			
		\end{minipage}
	};
%------------ Solvable groups Header ---------------------
	'\node[fancytitle, right=10pt] at (box.north west) {Solvable groups};
\end{tikzpicture}

%------------ Nilpotent groups Content ---------------
\begin{tikzpicture}
	\node [mybox] (box){%
		\begin{minipage}{0.3\textwidth}
			A group is called \textbf{nilpotent} if its \textbf{lower central series} defined below terminates,
			\[ G=G_0\trianglerighteq G_1 \trianglerighteq \cdots \trianglerighteq G_n = 1, G_{i+1}=[G_i,G].\]
			and $n$ is called its nilpotency class.
			
			
		\end{minipage}
	};
	%------------ Nilpotent groups Header ---------------------
	'\node[fancytitle, right=10pt] at (box.north west) {Nilpotent groups};
\end{tikzpicture}


\end{multicols*}
\end{document}


© 2016 GitHub, Inc. Terms Privacy Security Status Help