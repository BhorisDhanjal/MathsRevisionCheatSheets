\documentclass[dvipsnames]{article}
\usepackage{geometry}
\geometry{a3paper, landscape, margin=2in}
\usepackage{url}
\usepackage{multicol}
\usepackage{esint}
\usepackage{amsfonts}
\usepackage{tikz}
\usetikzlibrary{decorations.pathmorphing}
\usepackage{amsmath,amssymb}
\usepackage{enumitem}
\usepackage{colortbl}
\usepackage{mathtools}
\usepackage{ mathrsfs }	
\usepackage{ dsfont }
\usepackage{ gensymb }

\usepackage[activate={true,nocompatibility},final,tracking=true,kerning=true,spacing=true,factor=1100,stretch=10,shrink=10]{microtype}
\makeatletter
\setlist[itemize]{noitemsep, topsep=0pt}
\usepackage{quiver}
\newcommand*\bigcdot{\mathpalette\bigcdot@{.5}}
\newcommand*\bigcdot@[2]{\mathbin{\vcenter{\hbox{\scalebox{#2}{$\m@th#1\bullet$}}}}}
\makeatother
%Sheet made by Boris, Template is by Drew Ulick https://de.overleaf.com/articles/130-cheat-sheet/ntwtkmpxmgrp
\usepackage[english]{babel}
\usepackage{lmodern}
\usepackage{bm}
	\definecolor{cobalt}{rgb}{0.0, 0.28, 0.67}
\advance\topmargin-1.2in
\advance\textheight3in
\advance\textwidth3in
\advance\oddsidemargin-1.5in
\advance\evensidemargin-1.5in
\parindent0pt
\parskip2pt
\newcommand{\hr}{\centerline{\rule{3.5in}{1pt}}}
\newcommand{\R}{\mathbb{R}}
\newcommand{\Q}{\mathbb{Q}}
\newcommand{\C}{\mathbb{C}}
\newcommand{\Z}{\mathbb{Z}}
\newcommand{\N}{\mathbb{N}}
\newcommand{\D}{\mathbb{D}}
\newcommand{\F}{\mathbb{F}}
% \colorbox[HTML]{e4e4e4}{\makebox[\textwidth-2\fboxsep][l]{texto}}
\begin{document}
\definecolor{orange}{RGB}{255, 115,0}
\begin{center}{\fontsize{35}{60}{\textcolor{orange}{\textbf{Category Theory Cheat Sheet}}}}\\
\end{center}
\begin{multicols*}{3}
\tikzset{every loop/.style={min distance=5mm,in=20,out=90,looseness=1}}
\tikzstyle{mybox} = [draw=orange, fill=white, very thick,
    rectangle, rounded corners, inner sep=10pt, inner ysep=10pt]
\tikzstyle{fancytitle} =[fill=orange	, text=white, font=\bfseries]

% ------------------------------------------------------
% Start of Sheet
% ------------------------------------------------------

%------------ Category ---------------
\begin{tikzpicture}
\node [mybox] (box){%
    \begin{minipage}{0.3\textwidth}
	    A \textit{category} consists of the following,
	    \begin{itemize}
	    	\item Objects: A,B,C,\dots
	    	\item Arrows/Morphisms: f,g,h,\dots
	    	\item For each $ f $ there exists, $ \mathrm{dom}(f) , \mathrm{cod}(f)$ called domain and codomain of $ f $. We write $ f: A \to B $ to indicate $ A=\mathrm{dom}(f) $ and $ B=\mathrm{cod}(f) $.
	    	\item Given  $ f: A \to B$ and $ g: B \to C $ there exists, $ g \circ f: A \to C $ called the \textit{composite} of $ f $ and $ g $.
	    	\item For each $ A $, there exists $ 1_A:A\to A $ called the \textit{identity arrow} of $ A $.
	    	\item Arrows should also satisfy the following,
	    	\begin{itemize}
	    		\item Associativity: $ h \circ(g \circ f) = (h \circ g) \circ f,$ for all $ f:A \to B, g:B \to C, h: C \to D $.
	    		\item Unit: $ f\circ 1_A=f=1_B\circ f, $ for all $ f:A \to B $.	
	    	\end{itemize}
	    \end{itemize}
    \[\begin{tikzcd}
    	A \arrow[loop above,"1_A"] & {} & B \arrow[loop above,"1_B"]\\
    	&&& {} \\
    	&& C \arrow[loop below,"1_C"] && D \arrow[loop below,"1_D"]
    	\arrow["f", from=1-1, to=1-3]
    	\arrow["g", from=1-3, to=3-3]
    	\arrow["{g \circ f}"', from=1-1, to=3-3]
    	\arrow["{h \circ g}", from=1-3, to=3-5]
    	\arrow["h"', from=3-3, to=3-5]
    \end{tikzcd}\]
    \end{minipage}
};
%------------ Category Header ---------------------
\node[fancytitle, right=10pt] at (box.north west) {Category};
\end{tikzpicture}


%------------ Functor ---------------
\begin{tikzpicture}
	\node [mybox] (box){%
		\begin{minipage}{0.3\textwidth}
		A \textit{functor} $ F: C \to D $ between $ C $ and $D $ is a mapping of objects and arrows to arrows, such that
		\begin{itemize}
			\item $ F(f:A\to B) =F(f):F(A)\to F(B)$
			\item $ F(1_a)=1_{F(A)} $
			\item $ F(g \circ f) = F(g)\circ F(f)$.
		\end{itemize}
		\end{minipage}
	};
	%------------ Functor Header ---------------------
	\node[fancytitle, right=10pt] at (box.north west) {Functor};
\end{tikzpicture}

%------------ Isomorphism ---------------
\begin{tikzpicture}
	\node [mybox] (box){%
		\begin{minipage}{0.3\textwidth}
			In any category $ \mathbf{C}, $ an arrow $ f:A\to B$ is called an \textit{isomorphism} if there exists an arrow $ g:B \to C $ s.t. $ g \circ f=1_A  $ and $ f \circ g=1_B $. We say, $ g=f^{-1} $. And that $ A \cong B $, i.e., $ A $ is isomorphic to $ B $.
		\end{minipage}
	};
	%------------ Isomorphism Header ---------------------
	\node[fancytitle, right=10pt] at (box.north west) {Isomorphism};
\end{tikzpicture}

%------------ Monoid ---------------
\begin{tikzpicture}
	\node [mybox] (box){%
		\begin{minipage}{0.3\textwidth}
			A set $ M $ with binary operation $ \cdot $ is called a \textit{monoid} if it is associative and has an identity $ u \in M $, i.e., for $ x,y,z \in M $
			\begin{itemize}
				\item $ x\cdot (y\cdot z)=(x \cdot y)\cdot z $
				\item $ u\cdot x=x=x\cdot u $.
			\end{itemize}
		
		A monoid with an inverse for each element is called a \textit{group}.
		\begin{itemize}
			\item \textit{Cayley's theorem: }Every group G is isomorphic to a group of permutations.
		\end{itemize}
		\end{minipage}
	};
	%------------ Monoid Header ---------------------
	\node[fancytitle, right=10pt] at (box.north west) {Monoid};
\end{tikzpicture}

%------------ Constructions on categories ---------------
\begin{tikzpicture}
	\node [mybox] (box){%
		\begin{minipage}{0.3\textwidth}
		\begin{itemize}
			\item \textbf{Product category: } The product of two categories $ \mathbf{C} $ and $ \mathbf{D}$ written as $ \mathbf{C}\times \mathbf{D} $ has objects of the form $ (C,D) $ for $ C \in \mathbf{C} $ and $ D \in \mathbf{D} $, and arrows of the form $ (f,g) :(C,D)\to (C', D')$ for $ f:C \to C' \in \mathbf{C} $ and $ g:D\to D' \in \mathbf{D} $.
			
			Composition and units are defined componentwise, i.e. $ (f',g')\circ(f,g)=(f'\circ f, g' \circ g) $ and $ 1_{(C,D)}=(1_C,1_D)$.
			\item \textbf{Opposite/Dual category: }For category $ \mathbf{C} $ its opposite category $ \mathbf{C}^{\mathrm op} $ has the same objects as $ \mathbf{C} $ but an arrow $ f:C \to D $ in $ \mathbf{C}^{\mathrm op} $ is an arrow $ f: D \to C $ in $ \mathbf{C} $.
			
			For notational simplicity we say $ f^*:D^*\to C^* $ in $ \mathbf{C}^{\mathrm op} $ for $ f:C \to D $ in $ \mathbf{C} $. Composition and units are therefore defined as follows, $ f^*\circ g*=(g\circ f)^*$ and $ (1_{C^*} =1_C)^*$
		\end{itemize}
		\end{minipage}
	};
	%------------ Constructions on categories Header ---------------------
	\node[fancytitle, right=10pt] at (box.north west) {Constructions on categories};
\end{tikzpicture}

%------------ Constructions on categories contd. ---------------
\begin{tikzpicture}
	\node [mybox] (box){%
		\begin{minipage}{0.3\textwidth}
			\begin{itemize}
				\item \textbf{Arrow category: }For category $ \mathbf{C} $ its arrow category $ \mathbf{C}^\to $ has the arrows of $ \mathbf{C} $ as objects and an arrow $ g $ from $ f:A\to B $ to $f':A' \to B' $ in $ \mathbf{C}^\to  $ is the following commutative square	
				\[\begin{tikzcd}
					A & {A'} \\
					B & {B'}
					\arrow["{g_1}", from=1-1, to=1-2]
					\arrow["f"', from=1-1, to=2-1]
					\arrow["{g_2}"', from=2-1, to=2-2]
					\arrow["{f'}", from=1-2, to=2-2]
				\end{tikzcd}\]
			where $ g_1,g_2 $ are arrows in $ \mathbf{C} $, i.e. an arrow is a pair of arrows $ g=(g_1,g_2) $ s.t. $ g_2 \circ f = f' \circ g_1 $. The identity of an object $ f:A \to B $ is the pair $ (1_A,1_B) $ and composition is componentwise.	
				\item \textbf{Slice category: }For category $ \mathbf{C} $ its slice category over $ C \in \mathbf{C} $ is $ \mathbf{C}/C$ where its objects are all arrows $ f\in \mathbf{C} $ s.t. $ \mathrm{cod} (f)=C$. An arrow $ a $ from $ f:X\to C $ to $ f':X' \to C $ is an arrow $ a: X \to X' $ in $ \mathbf{C} $ s.t. $ f'\circ a=f  $, i.e.
				\[\begin{tikzcd}
					X && {X'} \\
					& C
					\arrow["a", from=1-1, to=1-3]
					\arrow["f"', from=1-1, to=2-2]
					\arrow["{f'}"', from=2-2, to=1-3]
				\end{tikzcd}\]
			
				If $ g:C \to D $ is any arrow then there exists a composition functor $ g_*:\mathbf{C}/C \to \mathbf{C}/D $ defined as $ g_*(f) = g \circ f$. Therefore the slice category of $ \mathbf{C} $ with any of its objects is a functor from $ \mathbf{C} \to \mathbf{\mathrm Cat}$. This is called the \textit{forgetful} functor as the base object is ''forgetten".
			\end{itemize}
		\end{minipage}
	};
	%------------ Constructions on categories contd. Header ---------------------
	\node[fancytitle, right=10pt] at (box.north west) {Constructions on categories contd.};
\end{tikzpicture}

%------------ Free monoid ---------------
\begin{tikzpicture}
	\node [mybox] (box){%
		\begin{minipage}{0.3\textwidth}
			For a set $ A $ a \textit{word} over $ A $ is any finite sequence of its elements. The \textit{Kleene closure} of $ A $ is defined to be the set of all words over $ A $. Define the binary operation of concatenation on $ A* $. Since it is associative, $ A $ along with $ * $ with the empty word $ ``-'' $ is a monoid called the \textbf{free monoid} on the set $ A $.		
			
			\textbf{Universal mapping property (UMP) of free monoid: }Let $ M(A) $ be the free monoid on a set $ A $. There is a function $ i: A\to |M(A)|, $ and given any monoid $ N $ and any function $ f:A \to |N| $, there is a unique monoid homomorphism $ \overline{f}:M(A)\to N$ s.t. $ |\overline{f}|\circ i=f $.
			\[\begin{tikzcd}
				{M(A)} && {|M(A)|} & {|N|} \\
				N && A 
				\arrow["{\overline{f}}", dashed, from=1-1, to=2-1]
				\arrow["i", from=2-3, to=1-3]
				\arrow["f", from=2-3, to=1-4]
				\arrow["{|\overline{f}|}", from=1-3, to=1-4]
			\end{tikzcd}\]
			$ A* $ has the UMP of the free monoid on $ A $.
		\end{minipage}
	};
	%------------ Free monoid Header ---------------------
	\node[fancytitle, right=10pt] at (box.north west) {Free monoid};
\end{tikzpicture}

%------------ Free category ---------------
\begin{tikzpicture}
	\node [mybox] (box){%
		\begin{minipage}{0.3\textwidth}
		A directed graph $ G $ ``generates'' a free category $ \mathbf{C}(G) $ whose objects are the vertices of the graph and its arrows are paths. Composition of arrows is defined as concatenation of paths.
		
		\textbf{UMP of} $ \mathbf{C}(G)$ There is a graphic homomorphism $ i:G\to |\mathbf{C}(G)| ,$ and given any category $ \mathbf{D} $ and any graph homomorphism $ h: G \to |\mathbf{D}|, $ there is a unique functor $ \overline{h}:\mathbf{C}(G) \to \mathbf{D} $ with $ \overline{h} \circ i=h $.
		\[\begin{tikzcd}
			{\mathbf{C}(G)} && {|\mathbf{C}(G)|} & {|\mathbf{D}|} \\
			\mathbf{D} && G
			\arrow["{\overline{h}}", dashed, from=1-1, to=2-1]
			\arrow["i", from=2-3, to=1-3]
			\arrow["h", from=2-3, to=1-4]
			\arrow["{|\overline{h}|}", from=1-3, to=1-4]
		\end{tikzcd}\]
		\end{minipage}
	};
	%------------ Free category Header ---------------------
	\node[fancytitle, right=10pt] at (box.north west) {Free category};
\end{tikzpicture}

%------------ Small categories ---------------
\begin{tikzpicture}
	\node [mybox] (box){%
		\begin{minipage}{0.3\textwidth}
			A category is called \textit{small} if it has a small set of objects and arrows. (i.e., not classes). It is called large otherwise.
			
			A category $ \mathbf{C} $ is \textit{locally small} if for all objects $ X,Y \in \mathbf{C} $, the collection $ \mathrm{Hom}_\mathbf{C}(X,Y)=\{f \in \mathbf{C}_1 \mid f: X \to Y \}$ is a small set.
		\end{minipage}
	};
	%------------ Small categories Header ---------------------
	\node[fancytitle, right=10pt] at (box.north west) {Small categories};
\end{tikzpicture}

%------------ Types of morphisms ---------------
\begin{tikzpicture}
	\node [mybox] (box){%
		\begin{minipage}{0.3\textwidth}
			Monomoprhism
		\end{minipage}
	};
	%------------ Types of morphisms Header ---------------------
	\node[fancytitle, right=10pt] at (box.north west) {Types of morphisms};
\end{tikzpicture}

% ==============================================================
\end{multicols*}
\end{document}
