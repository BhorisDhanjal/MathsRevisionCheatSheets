\documentclass[dvipsnames]{article}
\usepackage{geometry}
\geometry{a3paper, landscape, margin=2in}
\usepackage{url}
\usepackage{multicol}
\usepackage{esint}
\usepackage{amsfonts}
\usepackage{tikz}
\usetikzlibrary{decorations.pathmorphing}
\usepackage{amsmath,amssymb}
\usepackage{enumitem}
\usepackage{colortbl}
\usepackage{mathtools}
\usepackage{ mathrsfs }	
\usepackage{ dsfont }
\usepackage{ gensymb }

\usepackage[activate={true,nocompatibility},final,tracking=true,kerning=true,spacing=true,factor=1100,stretch=10,shrink=10]{microtype}
\makeatletter
\setlist[itemize]{noitemsep, topsep=0pt}
\usepackage{quiver}
\newcommand*\bigcdot{\mathpalette\bigcdot@{.5}}
\newcommand*\bigcdot@[2]{\mathbin{\vcenter{\hbox{\scalebox{#2}{$\m@th#1\bullet$}}}}}
\makeatother
%Sheet made by Boris, Template is by Drew Ulick https://de.overleaf.com/articles/130-cheat-sheet/ntwtkmpxmgrp
\usepackage[english]{babel}
\usepackage{lmodern}
\usepackage{bm}
	\definecolor{cobalt}{rgb}{0.0, 0.28, 0.67}
\advance\topmargin-1.2in
\advance\textheight3in
\advance\textwidth3in
\advance\oddsidemargin-1.5in
\advance\evensidemargin-1.5in
\parindent0pt
\parskip2pt
\newcommand{\hr}{\centerline{\rule{3.5in}{1pt}}}
\newcommand{\R}{\mathbb{R}}
\newcommand{\Q}{\mathbb{Q}}
\newcommand{\C}{\mathbb{C}}
\newcommand{\Z}{\mathbb{Z}}
\newcommand{\N}{\mathbb{N}}
\newcommand{\D}{\mathbb{D}}
\newcommand{\F}{\mathbb{F}}
% \colorbox[HTML]{e4e4e4}{\makebox[\textwidth-2\fboxsep][l]{texto}}
\begin{document}
\definecolor{orange}{RGB}{255, 115,0}
\begin{center}{\fontsize{35}{60}{\textcolor{orange}{\textbf{Category Theory Cheat Sheet}}}}\\
\end{center}
\begin{multicols*}{3}
\tikzset{every loop/.style={min distance=5mm,in=20,out=90,looseness=1}}
\tikzstyle{mybox} = [draw=orange, fill=white, very thick,
    rectangle, rounded corners, inner sep=10pt, inner ysep=10pt]
\tikzstyle{fancytitle} =[fill=orange	, text=white, font=\bfseries]

% ------------------------------------------------------
% Start of Sheet
% ------------------------------------------------------

%------------ Category ---------------
\begin{tikzpicture}
\node [mybox] (box){%
    \begin{minipage}{0.3\textwidth}
	    A \textit{category} consists of the following,
	    \begin{itemize}
	    	\item Objects: A,B,C,\dots
	    	\item Arrows/Morphisms: f,g,h,\dots
	    	\item For each $ f $ there exists, $ \mathrm{dom}(f) , \mathrm{cod}(f)$ called domain and codomain of $ f $. We write $ f: A \to B $ to indicate $ A=\mathrm{dom}(f) $ and $ B=\mathrm{cod}(f) $.
	    	\item Given  $ f: A \to B$ and $ g: B \to C $ there exists, $ g \circ f: A \to C $ called the \textit{composite} of $ f $ and $ g $.
	    	\item For each $ A $, there exists $ 1_A:A\to A $ called the \textit{identity arrow} of $ A $.
	    	\item Arrows should also satisfy the following,
	    	\begin{itemize}
	    		\item Associativity: $ h \circ(g \circ f) = (h \circ g) \circ f,$ for all $ f:A \to B, g:B \to C, h: C \to D $.
	    		\item Unit: $ f\circ 1_A=f=1_B\circ f, $ for all $ f:A \to B $.	
	    	\end{itemize}
	    \end{itemize}
    \[\begin{tikzcd}
    	A \arrow[loop above,"1_A"] & {} & B \arrow[loop above,"1_B"]\\
    	&&& {} \\
    	&& C \arrow[loop below,"1_C"] && D \arrow[loop below,"1_D"]
    	\arrow["f", from=1-1, to=1-3]
    	\arrow["g", from=1-3, to=3-3]
    	\arrow["{g \circ f}"', from=1-1, to=3-3]
    	\arrow["{h \circ g}", from=1-3, to=3-5]
    	\arrow["h"', from=3-3, to=3-5]
    \end{tikzcd}\]
    \end{minipage}
};
%------------ Category Header ---------------------
\node[fancytitle, right=10pt] at (box.north west) {Category};
\end{tikzpicture}


%------------ Functor ---------------
\begin{tikzpicture}
	\node [mybox] (box){%
		\begin{minipage}{0.3\textwidth}
		A \textit{functor} $ F: C \to D $ between $ C $ and $D $ is a mapping of objects and arrows to arrows, such that
		\begin{itemize}
			\item $ F(f:A\to B) =F(f):F(A)\to F(B)$
			\item $ F(1_a)=1_{F(A)} $
			\item $ F(g \circ f) = F(g)\circ F(f)$.
		\end{itemize}
		\end{minipage}
	};
%------------ Functor Header ---------------------
	\node[fancytitle, right=10pt] at (box.north west) {Functor};
\end{tikzpicture}

%------------ Isomorphism ---------------
\begin{tikzpicture}
	\node [mybox] (box){%
		\begin{minipage}{0.3\textwidth}
			In any category $ \mathbf{C}, $ an arrow $ f:A\to B$ is called an \textit{isomorphism} if there exists an arrow $ g:B \to C $ s.t. $ g \circ f=1_A  $ and $ f \circ g=1_B $. We say, $ g=f^{-1} $. And that $ A \cong B $, i.e., $ A $ is isomorphic to $ B $.
		\end{minipage}
	};
%------------ Isomorphism Header ---------------------
	\node[fancytitle, right=10pt] at (box.north west) {Isomorphism};
\end{tikzpicture}

%------------ Monoid ---------------
\begin{tikzpicture}
	\node [mybox] (box){%
		\begin{minipage}{0.3\textwidth}
			A set $ M $ with binary operation $ \cdot $ is called a \textit{monoid} if it is associative and has an identity $ u \in M $, i.e., for $ x,y,z \in M $
			\begin{itemize}
				\item $ x\cdot (y\cdot z)=(x \cdot y)\cdot z $
				\item $ u\cdot x=x=x\cdot u $.
			\end{itemize}
		
		A monoid with an inverse for each element is called a \textit{group}.
		\begin{itemize}
			\item \textit{Cayley's theorem: }Every group G is isomorphic to a group of permutations.
		\end{itemize}
		\end{minipage}
	};
%------------ Monoid Header ---------------------
	\node[fancytitle, right=10pt] at (box.north west) {Monoid};
\end{tikzpicture}

%------------ Constructions on categories ---------------
\begin{tikzpicture}
	\node [mybox] (box){%
		\begin{minipage}{0.3\textwidth}
		\begin{itemize}
			\item \textbf{Product category: } The product of two categories $ \mathbf{C} $ and $ \mathbf{D}$ written as $ \mathbf{C}\times \mathbf{D} $ has objects of the form $ (C,D) $ for $ C \in \mathbf{C} $ and $ D \in \mathbf{D} $, and arrows of the form $ (f,g) :(C,D)\to (C', D')$ for $ f:C \to C' \in \mathbf{C} $ and $ g:D\to D' \in \mathbf{D} $.
			
			Composition and units are defined componentwise, i.e. $ (f',g')\circ(f,g)=(f'\circ f, g' \circ g) $ and $ 1_{(C,D)}=(1_C,1_D)$.
			\item \textbf{Opposite/Dual category: }For category $ \mathbf{C} $ its opposite category $ \mathbf{C}^{\mathrm op} $ has the same objects as $ \mathbf{C} $ but an arrow $ f:C \to D $ in $ \mathbf{C}^{\mathrm op} $ is an arrow $ f: D \to C $ in $ \mathbf{C} $.
			
			For notational simplicity we say $ f^*:D^*\to C^* $ in $ \mathbf{C}^{\mathrm op} $ for $ f:C \to D $ in $ \mathbf{C} $. Composition and units are therefore defined as follows, $ f^*\circ g*=(g\circ f)^*$ and $ (1_{C^*} =1_C)^*$
		\end{itemize}
		\end{minipage}
	};
%------------ Constructions on categories Header ---------------------
	\node[fancytitle, right=10pt] at (box.north west) {Constructions on categories};
\end{tikzpicture}

%------------ Constructions on categories contd. ---------------
\begin{tikzpicture}
	\node [mybox] (box){%
		\begin{minipage}{0.3\textwidth}
			\begin{itemize}
				\item \textbf{Arrow category: }For category $ \mathbf{C} $ its arrow category $ \mathbf{C}^\to $ has the arrows of $ \mathbf{C} $ as objects and an arrow $ g $ from $ f:A\to B $ to $f':A' \to B' $ in $ \mathbf{C}^\to  $ is the following commutative square	
				\[\begin{tikzcd}
					A & {A'} \\
					B & {B'}
					\arrow["{g_1}", from=1-1, to=1-2]
					\arrow["f"', from=1-1, to=2-1]
					\arrow["{g_2}"', from=2-1, to=2-2]
					\arrow["{f'}", from=1-2, to=2-2]
				\end{tikzcd}\]
			where $ g_1,g_2 $ are arrows in $ \mathbf{C} $, i.e. an arrow is a pair of arrows $ g=(g_1,g_2) $ s.t. $ g_2 \circ f = f' \circ g_1 $. The identity of an object $ f:A \to B $ is the pair $ (1_A,1_B) $ and composition is componentwise.	
				\item \textbf{Slice category: }For category $ \mathbf{C} $ its slice category over $ C \in \mathbf{C} $ is $ \mathbf{C}/C$ where its objects are all arrows $ f\in \mathbf{C} $ s.t. $ \mathrm{cod} (f)=C$. An arrow $ a $ from $ f:X\to C $ to $ f':X' \to C $ is an arrow $ a: X \to X' $ in $ \mathbf{C} $ s.t. $ f'\circ a=f  $, i.e.
				\[\begin{tikzcd}
					X && {X'} \\
					& C
					\arrow["a", from=1-1, to=1-3]
					\arrow["f"', from=1-1, to=2-2]
					\arrow["{f'}"', from=2-2, to=1-3]
				\end{tikzcd}\]
			
				If $ g:C \to D $ is any arrow then there exists a composition functor $ g_*:\mathbf{C}/C \to \mathbf{C}/D $ defined as $ g_*(f) = g \circ f$. Therefore the slice category of $ \mathbf{C} $ with any of its objects is a functor from $ \mathbf{C} \to \mathbf{\mathrm Cat}$. This is called the \textit{forgetful} functor as the base object is ''forgetten".
			\end{itemize}
		\end{minipage}
	};
%------------ Constructions on categories contd. Header ---------------------
	\node[fancytitle, right=10pt] at (box.north west) {Constructions on categories contd.};
\end{tikzpicture}

%------------ Free monoid ---------------
\begin{tikzpicture}
	\node [mybox] (box){%
		\begin{minipage}{0.3\textwidth}
			For a set $ A $ a \textit{word} over $ A $ is any finite sequence of its elements. The \textit{Kleene closure} of $ A $ is defined to be the set of all words over $ A $. Define the binary operation of concatenation on $ A* $. Since it is associative, $ A $ along with $ * $ with the empty word $ ``-'' $ is a monoid called the \textbf{free monoid} on the set $ A $.		
			
			\textbf{Universal mapping property (UMP) of free monoid: }Let $ M(A) $ be the free monoid on a set $ A $. There is a function $ i: A\to |M(A)|, $ and given any monoid $ N $ and any function $ f:A \to |N| $, there is a unique monoid homomorphism $ \overline{f}:M(A)\to N$ s.t. $ |\overline{f}|\circ i=f $.
			\[\begin{tikzcd}
				{M(A)} && {|M(A)|} & {|N|} \\
				N && A 
				\arrow["{\overline{f}}", dashed, from=1-1, to=2-1]
				\arrow["i", from=2-3, to=1-3]
				\arrow["f", from=2-3, to=1-4]
				\arrow["{|\overline{f}|}", from=1-3, to=1-4]
			\end{tikzcd}\]
			$ A* $ has the UMP of the free monoid on $ A $.
		\end{minipage}
	};
%------------ Free monoid Header ---------------------
	\node[fancytitle, right=10pt] at (box.north west) {Free monoid};
\end{tikzpicture}

%------------ Free category ---------------
\begin{tikzpicture}
	\node [mybox] (box){%
		\begin{minipage}{0.3\textwidth}
		A directed graph $ G $ ``generates'' a free category $ \mathbf{C}(G) $ whose objects are the vertices of the graph and its arrows are paths. Composition of arrows is defined as concatenation of paths.
		
		\textbf{UMP of} $ \mathbf{C}(G)$ There is a graphic homomorphism $ i:G\to |\mathbf{C}(G)| ,$ and given any category $ \mathbf{D} $ and any graph homomorphism $ h: G \to |\mathbf{D}|, $ there is a unique functor $ \overline{h}:\mathbf{C}(G) \to \mathbf{D} $ with $ \overline{h} \circ i=h $.
		\[\begin{tikzcd}
			{\mathbf{C}(G)} && {|\mathbf{C}(G)|} & {|\mathbf{D}|} \\
			\mathbf{D} && G
			\arrow["{\overline{h}}", dashed, from=1-1, to=2-1]
			\arrow["i", from=2-3, to=1-3]
			\arrow["h", from=2-3, to=1-4]
			\arrow["{|\overline{h}|}", from=1-3, to=1-4]
		\end{tikzcd}\]
		\end{minipage}
	};
%------------ Free category Header ---------------------
	\node[fancytitle, right=10pt] at (box.north west) {Free category};
\end{tikzpicture}

%------------ Small categories ---------------
\begin{tikzpicture}
	\node [mybox] (box){%
		\begin{minipage}{0.3\textwidth}
			A category is called \textit{small} if it has a small set of objects and arrows. (i.e., not classes). It is called large otherwise.
			
			A category $ \mathbf{C} $ is \textit{locally small} if for all objects $ X,Y \in \mathbf{C} $, the collection $ \mathrm{Hom}_\mathbf{C}(X,Y)=\{f \in \mathbf{C}_1 \mid f: X \to Y \}$ is a small set.
		\end{minipage}
	};
%------------ Small categories Header ---------------------
	\node[fancytitle, right=10pt] at (box.north west) {Small categories};
\end{tikzpicture}

%------------ Types of morphisms ---------------
\begin{tikzpicture}
	\node [mybox] (box){%
		\begin{minipage}{0.3\textwidth}
			\textbf{Monomoprhism: }In any category $ \mathbf{C} $, an arrow	$ f: A \to B  $ is called a monomorphism (monic), if for any $ g,h, :C \to A , fg=fh \implies g=h $.

			\[\begin{tikzcd}
			\arrow[r,shift right,swap,"h"] \arrow[r,shift left,"g"]	C &  
				 A & B
				\arrow["f", from=1-2, to=1-3]
			\end{tikzcd}\]
			
			\textbf{Epimorphism: }In any category $ \mathbf{C} $, an arrow $ f: A \to B $ is called an epimorphism (epic), if for any $ i,j: B \to D $ $ if=jf\implies i=j $.
			
			\[\begin{tikzcd}
				A &  \arrow[r,shift right,swap,"i"] \arrow[r,shift left,"j	"]
				B & D
				\arrow["f", from=1-1, to=1-2]
			\end{tikzcd}\]
			\begin{itemize}
				\item We say, $ f: A\rightarrowtail B $ if $ f $ is a monomorphism and $ f: A \twoheadrightarrow B $ if $ f $ is an epimorphism.
				\item Every isomorphism is both a monomorphism and an epimorphism. The converse need not be true.
				\item A \textbf{split} mono (epi) is an arrow $ m : A \to B$ with a left (right) inverse $ r $. The inverse arrow $ r $ is called the \textit{retraction}, $ m $ is called a \textit{section} of $ r $ and $ A $ is called a \textit{retract} of $ B $.
			\end{itemize}
			

		\end{minipage}
	};
%------------ Types of morphisms Header ---------------------
	\node[fancytitle, right=10pt] at (box.north west) {Types of morphisms};
\end{tikzpicture}

%------------ Initial and terminal objects ---------------
\begin{tikzpicture}
	\node [mybox] (box){%
		\begin{minipage}{0.3\textwidth}
			An object $ 0 \in \mathbf{C} $ is \textit{initial} if for any object $ C \in \mathbf{C} \exists!$ morphism $ 0 \to C $.\\
			An object $ 1 \in \mathbf{C} $ is \textit{terminal} if for any object $ C \in \mathbf{C} \exists!$ morphism $ C \to 1 $.
			
			Initial and terminal objects are unique up to isomorphism.
		\end{minipage}
	};
%------------ Initial and terminal objects Header ---------------------
	\node[fancytitle, right=10pt] at (box.north west) {Initial and terminal objects};
\end{tikzpicture}

%------------ Generalized elements ---------------
\begin{tikzpicture}
	\node [mybox] (box){%
		\begin{minipage}{0.3\textwidth}
			For an object $ A \in \mathbf{C} $ arbitrary arrows $ x:X\to A $ are called the \textit{generalized elements} of $ A $ with stage of definition given by $ X $.
		\end{minipage}
	};
%------------ Generalized elements Header ---------------------
	\node[fancytitle, right=10pt] at (box.north west) {Generalized elements};
\end{tikzpicture}

%------------ Product of objects ---------------
\begin{tikzpicture}
	\node [mybox] (box){%
		\begin{minipage}{0.3\textwidth}
			In any category $ \mathbf{C} $, a product diagram for the objects $ A, B $ consists of an object $ P $ and arrows 
			\[\begin{tikzcd}
				A & P & B
				\arrow["{p_1}"', from=1-2, to=1-1]
				\arrow["{p_2}", from=1-2, to=1-3]
			\end{tikzcd}\]
		satisfying the following UMP. Given any diagram of the form
		\[\begin{tikzcd}
			A & X & B
			\arrow["{x_1}"', from=1-2, to=1-1]
			\arrow["{x_2}", from=1-2, to=1-3]
		\end{tikzcd}\]
		there exists a unique arrow $ u:X \to P $, making the following diagram commute
		\[\begin{tikzcd}
			& X \\
			A & P & B
			\arrow["{p_1}"', from=2-2, to=2-1]
			\arrow["{p_2}", from=2-2, to=2-3]
			\arrow["u", dashed, from=1-2, to=2-2]
			\arrow["{x_1}"', from=1-2, to=2-1]
			\arrow["{x_2}", from=1-2, to=2-3]
		\end{tikzcd}\]
	
		The product $ P $ is unique up to isomorphism. 
	 
		\end{minipage}
	};
%------------ Product of objects Header ---------------------
	\node[fancytitle, right=10pt] at (box.north west) {Product of objects};
\end{tikzpicture}


%------------ Categories with products ---------------
\begin{tikzpicture}
	\node [mybox] (box){%
		\begin{minipage}{0.3\textwidth}
			A category which has a product for every pair of objects is said to have \textit{binary products}.
			
			A category is said to have \textit{all finite products}, if it has a terminal object and all binary products.
			
			A category has \textit{all small products} if every set of objects has a product.
		\end{minipage}
	};
%------------ Categories with products Header ---------------------
	\node[fancytitle, right=10pt] at (box.north west) {Categories with products};
\end{tikzpicture}


%------------ Covariant representable functor ---------------
\begin{tikzpicture}
	\node [mybox] (box){%
		\begin{minipage}{0.3\textwidth}
		The functor $ \mathrm{Hom}(A,-): \mathbf{C}\to \mathbf{Sets}$ is called a covariant representable functor (for some object $ A \in \mathbf{C} $).
		
		For a category with products a covariant representable functor preserves products.
		\end{minipage}
	};
%------------ Covariant representable functor Header ---------------------
	\node[fancytitle, right=10pt] at (box.north west) {Covariant representable functor};
\end{tikzpicture}

%------------ Duality ---------------
\begin{tikzpicture}
	\node [mybox] (box){%
		\begin{minipage}{0.3\textwidth}
			If any statement about categories holds for all categories then so does the dual statement.
		\end{minipage}
	};
%------------ Duality Header ---------------------
	\node[fancytitle, right=10pt] at (box.north west) {Duality};
\end{tikzpicture}

%------------ Coproducts ---------------
\begin{tikzpicture}
	\node [mybox] (box){%
		\begin{minipage}{0.3\textwidth}
			A diagram $ A \xrightarrow{q_1}Q\xleftarrow{q_2}B $ is a coproduct of $ A $ and $ B $ if for any $ Z $ and $ A \xrightarrow{z_1}Z\xleftarrow{z_2}B $ there is a unique $ u: Q \to Z $ making the diagram commute.
			\[\begin{tikzcd}
				& Z \\
				A & Q & B
				\arrow["{z_1}", from=2-1, to=1-2]
				\arrow["{z_2}"', from=2-3, to=1-2]
				\arrow["{q_1}"', from=2-1, to=2-2]
				\arrow["{q_2}", from=2-3, to=2-2]
				\arrow["u"', dashed, from=2-2, to=1-2]
			\end{tikzcd}\]
		 \end{minipage}
	};
%------------ Coproducts Header ---------------------
	\node[fancytitle, right=10pt] at (box.north west) {Coproducts};
\end{tikzpicture}

%------------ Equalizers ---------------
\begin{tikzpicture}
	\node [mybox] (box){%
		\begin{minipage}{0.3\textwidth}
			In some category $ \mathbf{C} $ given the following diagram
			\[\begin{tikzcd}
				A & B
				\arrow["g"', shift right=2, from=1-1, to=1-2]
				\arrow["f", shift left=2, from=1-1, to=1-2]
			\end{tikzcd}\]
			We say an \textit{equalizer} of $ f,g $ consists of an object $ E $ and an arrow $ e:E \to A $ universal such that \[ f \circ e = g \circ e \]
			i.e., for any $ z:Z\to A $ with $ f \circ z= g \circ z $, there exists a unique $ u:Z \to E $ with $ e \circ u=z $
			\[\begin{tikzcd}
				E & A & B \\
				Z
				\arrow["g"', shift right=2, from=1-2, to=1-3]
				\arrow["f", shift left=2, from=1-2, to=1-3]
				\arrow["e", from=1-1, to=1-2]
				\arrow["u", dashed, from=2-1, to=1-1]
				\arrow["z", from=2-1, to=1-2]
			\end{tikzcd}\]
		\begin{itemize}
			\item Equalizers are monic.
		\end{itemize}
		\end{minipage}
	};
%------------ Equalizers Header ---------------------
	\node[fancytitle, right=10pt] at (box.north west) {Equalizers};
\end{tikzpicture}

%------------ Coequalizers ---------------
\begin{tikzpicture}
	\node [mybox] (box){%
		\begin{minipage}{0.3\textwidth}
				In some category $ \mathbf{C} $ given the following diagram
			\[\begin{tikzcd}
				A & B
				\arrow["g"', shift right=2, from=1-1, to=1-2]
				\arrow["f", shift left=2, from=1-1, to=1-2]
			\end{tikzcd}\]
			We say a \textit{coequalizer} of $ f,g $ consists of an object $ Q $ and an arrow $ q:B \to Q $ universal such that \[ q \circ f = q \circ g \]
			i.e., for any $ z:B\to Z $ with $ z \circ f= z \circ g $, there exists a unique $ u:Q \to Z $ with $ u \circ q=z $
			\[\begin{tikzcd}
				A & B & Q \\
				&& Z
				\arrow["g"', shift right=2, from=1-1, to=1-2]
				\arrow["f", shift left=2, from=1-1, to=1-2]
				\arrow["q", from=1-2, to=1-3]
				\arrow["z"', from=1-2, to=2-3]
				\arrow["u", dashed, from=1-3, to=2-3]
			\end{tikzcd}\]
		\begin{itemize}
			\item Coequalizers are epic.
		\end{itemize}
		\end{minipage}
	};
%------------ Coequalizers Header ---------------------
	\node[fancytitle, right=10pt] at (box.north west) {Coequalizers};
\end{tikzpicture}

%------------ Groups in a category ---------------
\begin{tikzpicture}
	\node [mybox] (box){%
		\begin{minipage}{0.3\textwidth}
			 A group ($ \mathrm{Group}(\mathbf{C}) $) can be defined over a category $ \mathbf{C} $.
			\[\begin{tikzcd}
				{G\times G} & G & G \\
				& 1
				\arrow["m", from=1-1, to=1-2]
				\arrow["i"', from=1-3, to=1-2]
				\arrow["u", from=2-2, to=1-2]
			\end{tikzcd}\]
		Where the arrows obey the following,
		$ m $ is associative, $ u $ is a unit, and $ i $ is an inverse for $ m $, i.e. the following diagrams commute
			\[\begin{tikzcd}
				{(G \times G) \times G} && {G \times (G \times G)} && G & {G \times G} \\
				{G \times G} && {G \times G} && {G\times G} & G \\
				& G \\
				{G \times G} & G & {G \times G} \\
				{G \times G} & G & {G \times G}
				\arrow["m"', from=2-1, to=3-2]
				\arrow["m", from=2-3, to=3-2]
				\arrow["{m \times 1}"', from=1-1, to=2-1]
				\arrow["{1 \times m}", from=1-3, to=2-3]
				\arrow["\cong", from=1-1, to=1-3]
				\arrow["m"', from=2-5, to=2-6]
				\arrow["{1_G}"', from=1-5, to=2-6]
				\arrow["m", from=1-6, to=2-6]
				\arrow["{\langle 1_G, u \rangle}"', from=1-5, to=2-5]
				\arrow["{\langle u, 1_G \rangle}", from=1-5, to=1-6]
				\arrow["{\langle 1_G, 1_G \rangle}"', from=4-2, to=4-1]
				\arrow["{\langle 1_G, 1_G \rangle}", from=4-2, to=4-3]
				\arrow["{1_G \times i}"', from=4-1, to=5-1]
				\arrow["u"', from=4-2, to=5-2]
				\arrow["{i \times 1_G}", from=4-3, to=5-3]
				\arrow["m"', from=5-1, to=5-2]
				\arrow["m", from=5-3, to=5-2]
			\end{tikzcd}\]
		\begin{itemize}
		\item A homomorphism $ h:G \to H $ of groups in a category $ \mathbf{C} $ is an arrow such that, $ h $ preserves $ m,u,i $, i.e. the following diagrams commmute.
		\[\begin{tikzcd}
			{G \times G} & {H \times H} && G & H && G & H \\
			G & H && 1 &&& G & H
			\arrow["{h \times h}", from=1-1, to=1-2]
			\arrow["m"', from=1-1, to=2-1]
			\arrow["m", from=1-2, to=2-2]
			\arrow["h"', from=2-1, to=2-2]
			\arrow["u", from=2-4, to=1-4]
			\arrow["h", from=1-4, to=1-5]
			\arrow["u"', from=2-4, to=1-5]
			\arrow["h", from=1-7, to=1-8]
			\arrow["i", from=1-8, to=2-8]
			\arrow["i"', from=1-7, to=2-7]
			\arrow["h"', from=2-7, to=2-8]
		\end{tikzcd}\]
		\item The objects in the category of groups (i.e. $ \mathrm{Group}(\mathbf{Grp}) $) are abelian groups.
		\end{itemize}
		\end{minipage}
	};
%------------ Groups in a category Header ---------------------
	\node[fancytitle, right=10pt] at (box.north west) {Groups in a category};
\end{tikzpicture}

%------------ Congruence ---------------
\begin{tikzpicture}
	\node [mybox] (box){%
		\begin{minipage}{0.3\textwidth}
			A \textit{congruence} on a category is a equivalance relation on arrows ($ f \sim g $) s.t.
			\begin{itemize}
				\item $ f \sim g \implies \mathrm{dom}(f)=\mathrm{dom}(f)$ and $ \mathrm{cod}(f)=\mathrm{cod}(g) $.
				\item $ f \sim g \implies bfa \sim bga $
			\end{itemize}
		Let $ C_0, C_1 $ denote the class of objects and arrows for a category $ \mathbf{C} $. Then a \textit{congruence category} $ \mathbf{C}^\sim  $ is defined as follows,
		\begin{itemize}
			\item $ (\mathbf{C}^\sim)_0=\mathbf{C}_0 $
			\item $ (\mathbf{C}^\sim)_1=\{\langle f,g \rangle| f \sim g\} $
			\item $ \tilde{1}_C=\langle 1_C, 1_C \rangle $
			\item $ \langle f', g' \rangle \circ \langle f,g \rangle = \langle f'f, g'g \rangle  $
		\end{itemize}
		\[\begin{tikzcd}
			{\mathbf{C}^\sim} & {\mathbf{C}}
			\arrow["{p_1}", shift left=2, from=1-1, to=1-2]
			\arrow["{p_2}"', shift right=2, from=1-1, to=1-2]
		\end{tikzcd}\]
		We define the \textit{quotient category} of the congruence as the coequalizer, i.e,
		\[\begin{tikzcd}
			{\mathbf{C}^\sim} & {\mathbf{C}} & {\mathbf{C}/\sim}
			\arrow["{p_1}", shift left=2, from=1-1, to=1-2]
			\arrow["{p_2}"', shift right=2, from=1-1, to=1-2]
			\arrow["\pi", from=1-2, to=1-3]
		\end{tikzcd}\]
		\end{minipage}
	};
%------------ Congruence Header ---------------------
	\node[fancytitle, right=10pt] at (box.north west) {Congruence};
\end{tikzpicture}

%------------ Finitely presented category ---------------
\begin{tikzpicture}
	\node [mybox] (box){%
		\begin{minipage}{0.3\textwidth}
			Consider the free category $ \mathbf{C} (G)$ on a finite graph $ G $. And the finite set of relations $ \sum  $ to be relations of the form $ (g_1\circ \dots \circ g_n)= (g_1'\circ \dots \circ g_m') $ for $ g_i \in G $ and $ \mathrm{dom}(g_n)=\mathrm{dom}(g_m') $ and $ \mathrm{cod}(g_1) =\mathrm{cod}(g_1')$. Let $ \sim_\Sigma  $ be the smallest congruence $ g \sim g' $ if $ g=g' \in \sum  $. We call the quotient by this congruence to be a \textit{fintely presented category}.
		\end{minipage}
	};
%------------ Finitely presented category Header ---------------------
	\node[fancytitle, right=10pt] at (box.north west) {Finitely presented category};
\end{tikzpicture}


% ==============================================================
\end{multicols*}
\end{document}
