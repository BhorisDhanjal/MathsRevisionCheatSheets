\documentclass[dvipsnames]{article}
\usepackage[usenames]{xcolor}
\usepackage{geometry}
\geometry{a3paper, landscape, margin=2in}
\usepackage{url}
\usepackage{multicol}
\usepackage{esint}
\usepackage{amsfonts}
\usepackage{tikz}
\usetikzlibrary{decorations.pathmorphing}
\usepackage{amsmath,amssymb}
\usepackage{enumitem}
\usepackage{colortbl}

\usepackage{mathtools}
\usepackage{ mathrsfs }	
\usepackage{ dsfont }
\usepackage{ gensymb }

\usepackage[activate={true,nocompatibility},final,tracking=true,kerning=true,spacing=true,factor=1100,stretch=10,shrink=10]{microtype}
\makeatletter
\setlist[itemize]{noitemsep, topsep=0pt}
\setlist[enumerate]{noitemsep, topsep=0pt}
\usepackage{quiver}

\newcommand*\bigcdot{\mathpalette\bigcdot@{.5}}
\newcommand*\bigcdot@[2]{\mathbin{\vcenter{\hbox{\scalebox{#2}{$\m@th#1\bullet$}}}}}
\makeatother
%Sheet made by Boris, Template is by Drew Ulick https://de.overleaf.com/articles/130-cheat-sheet/ntwtkmpxmgrp
\usepackage[english]{babel}
\usepackage{palatino}
\usepackage{bm}
	\definecolor{cobalt}{rgb}{0.0, 0.28, 0.67}
\advance\topmargin-1.2in
\advance\textheight3in
\advance\textwidth3in
\advance\oddsidemargin-1.5in
\advance\evensidemargin-1.5in
\parindent0pt
\parskip2pt
\newcommand{\hr}{\centerline{\rule{3.5in}{1pt}}}
\newcommand{\R}{\mathbb{R}}
\newcommand{\Q}{\mathbb{Q}}
\newcommand{\C}{\mathbb{C}}
\newcommand{\Z}{\mathbb{Z}}
\newcommand{\N}{\mathbb{N}}
\newcommand{\D}{\mathbb{D}}
\newcommand{\F}{\mathbb{F}}
% \colorbox[HTML]{e4e4e4}{\makebox[\textwidth-2\fboxsep][l]{texto}}

\begin{document}
\definecolor{orange}{RGB}{255, 115,0}
	\begin{center}{\fontsize{35}{60}{\textcolor{Cerulean}{\textbf{Commutative Algebra Cheat Sheet}}}}\\
\end{center}
\begin{multicols*}{3}

\tikzstyle{mybox} = [draw=Cerulean, fill=white, very thick,
    rectangle, rounded corners, inner sep=10pt, inner ysep=10pt]
\tikzstyle{fancytitle} =[fill=Cerulean, text=white, font=\bfseries]

% ------------------------------------------------------
% Start of Sheet
% ------------------------------------------------------

%------------ Rings ---------------
\begin{tikzpicture}
\node [mybox] (box){%
    \begin{minipage}{0.3\textwidth}
	    A \textbf{ring} $A$ is a set with two binary operations addition and multiplication such that
	    \begin{itemize}
	    	\item $A$ is an abelian group with addition.
	    	\item Multiplication is associative and distributive over addition.
	    \end{itemize}
	    Additionally we consider rings with commutativity and existence of multiplicative identity 1.
	    
	    A function $\varphi: A\to B$ between rings is a \textbf{homomorphism} if it preserves addition multiplication and sends $1$ to $1$.
	    
	    A \textbf{subring} is a subset of a ring that is also a ring with the induced relations.
    \end{minipage}
};
%------------ Rings Header ---------------------
\node[fancytitle, right=10pt] at (box.north west) {Rings};
\end{tikzpicture}


%------------ Ideals ---------------
\begin{tikzpicture}
	\node [mybox] (box){%
		\begin{minipage}{0.3\textwidth}
			An \textbf{ideal} $\mathfrak{a}$ of a ring $A$ is a subset of $A$ which is a additive subgroup group and for $x \in \mathfrak{a}, xA \subseteq \mathfrak{a}$.
			
			The cosets of $\mathfrak{a} \in A$ form a quotient ring $A/\mathfrak{a}$.
			
			\textbf{Correspondence theorem for rings:} There is a bijection between ideals of $A$ containing $\mathfrak{a}$ and the ideals of $A/\mathfrak{a}$.
		\end{minipage}
	};
%------------ Ideals Header ---------------------
	\node[fancytitle, right=10pt] at (box.north west) {Ideals};
\end{tikzpicture}


%------------ Zero divisors, units ---------------
\begin{tikzpicture}
	\node [mybox] (box){%
		\begin{minipage}{0.3\textwidth}
			An element is called a \textbf{zero divisor} if its product with a non zero element gives 0.	
			
			A commutative ring with the only zero divisor being zero is called an \textbf{integral domain}.
			
			An element is called a \textbf{unit} if its product with some element gives 1.
			\begin{itemize}
				\item $x\in A$ is a unit $\iff \langle x \rangle= \{ax\ |\ a \in A\}=A=\langle 1 \rangle	$
			\end{itemize}
			
			A ring in which every non zero element is a unit is called a \textbf{field}.
			\begin{itemize}
				\item All fields are integral domains.
				\item All finite integral domains are fields.
				\item The only ideals in a field $F$ are 0 and $\langle 1 \rangle=F$
			\end{itemize}
			
			
		\end{minipage}
	};
%------------ Zero divisors, units Header ---------------------
	\node[fancytitle, right=10pt] at (box.north west) {Zero divisors, units};
\end{tikzpicture}

%------------ Prime and Maximal ideals ---------------
\begin{tikzpicture}
	\node [mybox] (box){%
		\begin{minipage}{0.3\textwidth}
			A proper ideal $\mathfrak{p} \in A$ is called \textbf{prime} if for $xy \in \mathfrak{a} \implies x \in \mathfrak{p}$ or $y \in \mathfrak{p}$ alternatively if $A/\mathfrak{p}$ is an integral domain.
			
			A proper ideal $\mathfrak{m} \in A$ is called \textbf{maximal} if it is maximal with respect to inclusion alternatively if $A/\mathfrak{m}$ is a field.
			
			A ring with exactly one maximal ideal is called a \textbf{local} ring. And its subsequent quotient is called the \textbf{residue field} of the ring. If number of maximal ideals are finite then it is called \textbf{semi local}.
		\end{minipage}
	};
%------------ Prime and Maximal ideals Header ---------------------
	\node[fancytitle, right=10pt] at (box.north west) {Prime and Maximal ideals};
\end{tikzpicture}

%------------ Ideal operations ---------------
\begin{tikzpicture}
	\node [mybox] (box){%
		\begin{minipage}{0.3\textwidth}
			For ideals $\mathfrak{a}, \mathfrak{b} \in A $,
\begin{itemize}

\end{itemize}
		\end{minipage}
	};
%------------ Ideal operations Header ---------------------
	\node[fancytitle, right=10pt] at (box.north west) {Ideal operations};
\end{tikzpicture}

%------------ Radical ideals, ideal quotients ---------------
\begin{tikzpicture}
	\node [mybox] (box){%
		\begin{minipage}{0.3\textwidth}
			
		\end{minipage}
	};
%------------ Radical ideals, ideal quotients Header ---------------------
	\node[fancytitle, right=10pt] at (box.north west) {Radical ideals, ideal quotients};
\end{tikzpicture}

%------------ Nilradical and Jacobson ideal ---------------
\begin{tikzpicture}
	\node [mybox] (box){%
		\begin{minipage}{0.3\textwidth}
			
		\end{minipage}
	};
%------------ Nilradical and Jacobson ideal Header ---------------------
	\node[fancytitle, right=10pt] at (box.north west) {Nilradical and Jacobson ideal};
\end{tikzpicture}

%------------ Extension and Contraction of ideals ---------------
\begin{tikzpicture}
	\node [mybox] (box){%
		\begin{minipage}{0.3\textwidth}
			
		\end{minipage}
	};
%------------ Extension and Contraction of ideals Header ---------------------
	\node[fancytitle, right=10pt] at (box.north west) {Extension and Contraction of ideals};
\end{tikzpicture}


% ==============================================================
\end{multicols*}
\end{document}
