\documentclass[dvipsnames]{article}
\usepackage[usenames]{xcolor}
\usepackage{geometry}
\geometry{a3paper, landscape, margin=2in}
\usepackage{url}
\usepackage{multicol}
\usepackage{esint}
\usepackage{amsfonts}
\usepackage{tikz}
\usetikzlibrary{decorations.pathmorphing}
\usepackage{amsmath,amssymb}
\usepackage{enumitem}
\usepackage{colortbl}

\usepackage{mathtools}
\usepackage{ mathrsfs }	
\usepackage{ dsfont }
\usepackage{ gensymb }

\usepackage[activate={true,nocompatibility},final,tracking=true,kerning=true,spacing=true,factor=1100,stretch=10,shrink=10]{microtype}
\makeatletter
\setlist[itemize]{noitemsep, topsep=0pt}
\setlist[enumerate]{noitemsep, topsep=0pt}
\usepackage{quiver}

\newcommand*\bigcdot{\mathpalette\bigcdot@{.5}}
\newcommand*\bigcdot@[2]{\mathbin{\vcenter{\hbox{\scalebox{#2}{$\m@th#1\bullet$}}}}}
\makeatother
%Sheet made by Boris, Template is by Drew Ulick https://de.overleaf.com/articles/130-cheat-sheet/ntwtkmpxmgrp
\usepackage[english]{babel}
\usepackage{palatino}
\usepackage{bm}
	\definecolor{cobalt}{rgb}{0.0, 0.28, 0.67}
\advance\topmargin-1.2in
\advance\textheight3in
\advance\textwidth3in
\advance\oddsidemargin-1.5in
\advance\evensidemargin-1.5in
\parindent0pt
\parskip2pt
\newcommand{\hr}{\centerline{\rule{3.5in}{1pt}}}
\newcommand{\R}{\mathbb{R}}
\newcommand{\Q}{\mathbb{Q}}
\newcommand{\C}{\mathbb{C}}
\newcommand{\Z}{\mathbb{Z}}
\newcommand{\N}{\mathbb{N}}
\newcommand{\D}{\mathbb{D}}
\newcommand{\F}{\mathbb{F}}
% \colorbox[HTML]{e4e4e4}{\makebox[\textwidth-2\fboxsep][l]{texto}}

\begin{document}
\definecolor{orange}{RGB}{255, 115,0}
	\begin{center}{\fontsize{35}{60}{\textcolor{Cerulean}{\textbf{Commutative Algebra Cheat Sheet}}}}\\
\end{center}
\begin{multicols*}{3}

\tikzstyle{mybox} = [draw=Cerulean, fill=white, very thick,
    rectangle, rounded corners, inner sep=10pt, inner ysep=10pt]
\tikzstyle{fancytitle} =[fill=Cerulean, text=white, font=\bfseries]

% ------------------------------------------------------
% Start of Sheet
% ------------------------------------------------------

%------------ Rings ---------------
\begin{tikzpicture}
\node [mybox] (box){%
    \begin{minipage}{0.3\textwidth}
	    A \textbf{ring} $A$ is a set with two binary operations addition and multiplication such that
	    \begin{itemize}
	    	\item $A$ is an abelian group with addition.
	    	\item Multiplication is associative and distributive over addition.
	    \end{itemize}
	    Additionally we consider rings with commutativity and existence of multiplicative identity 1.
	    
	    A function $\varphi: A\to B$ between rings is a \textbf{homomorphism} if it preserves addition multiplication and sends $1$ to $1$.
	    
	    A \textbf{subring} is a subset of a ring that is also a ring with the induced relations.
    \end{minipage}
};
%------------ Rings Header ---------------------
\node[fancytitle, right=10pt] at (box.north west) {Rings};
\end{tikzpicture}

%------------ Universal property ---------------
\begin{tikzpicture}
	\node [mybox] (box){%
		\begin{minipage}{0.3\textwidth}
			A universal property is some notion of a construction that uniquely determines it. In particular for some functor $F$ between categories $\mathbf{C}, \mathbf{D}$ an arrow $X\to F$ for $X \in \mathbf{D}$ is said to be universal if there exists some arrow $u$ and $C \in \mathbf{C}$ such that all other arrows $X \to F(C')$ necessarily factor through $u: X \to F(C)$ uniquely.
			
			This will be very useful in characterizing certain constructions later.
			
			For example in the first isomorphism theorem infact the quotient of kernel is universal as all mappings to the image algebra factor through it.
		\end{minipage}
	};
	%------------ Universal property Header ---------------------
	\node[fancytitle, right=10pt] at (box.north west) {Universal mapping property};
\end{tikzpicture}




%------------ Ideals ---------------
\begin{tikzpicture}
	\node [mybox] (box){%
		\begin{minipage}{0.3\textwidth}
			An \textbf{ideal} $\mathfrak{a}$ of a ring $A$ is a subset of $A$ which is a additive subgroup group and for $x \in \mathfrak{a}, xA \subseteq \mathfrak{a}$.
			
			The cosets of $\mathfrak{a} \in A$ form a quotient ring $A/\mathfrak{a}$.
			
			\textbf{Correspondence theorem for rings:} There is a bijection between ideals of $A$ containing $\mathfrak{a}$ and the ideals of $A/\mathfrak{a}$.
		\end{minipage}
	};
%------------ Ideals Header ---------------------
	\node[fancytitle, right=10pt] at (box.north west) {Ideals};
\end{tikzpicture}


%------------ Zero divisors, units ---------------
\begin{tikzpicture}
	\node [mybox] (box){%
		\begin{minipage}{0.3\textwidth}
			An element is called a \textbf{zero divisor} if its product with a non zero element gives 0.	
			
			A commutative ring with the only zero divisor being zero is called an \textbf{integral domain}.
			
			An element is called a \textbf{unit} if its product with some element gives 1.
			\begin{itemize}
				\item $x\in A$ is a unit $\iff \langle x \rangle= \{ax\ |\ a \in A\}=A=\langle 1 \rangle	$
			\end{itemize}
			
			A ring in which every non zero element is a unit is called a \textbf{field}.
			\begin{itemize}
				\item All fields are integral domains.
				\item All finite integral domains are fields.
				\item The only ideals in a field $F$ are 0 and $\langle 1 \rangle=F$
			\end{itemize}
			
			
		\end{minipage}
	};
%------------ Zero divisors, units Header ---------------------
	\node[fancytitle, right=10pt] at (box.north west) {Zero divisors, units};
\end{tikzpicture}

%------------ Prime and Maximal ideals ---------------
\begin{tikzpicture}
	\node [mybox] (box){%
		\begin{minipage}{0.3\textwidth}
			A proper ideal $\mathfrak{p} \in A$ is called \textbf{prime} if for $xy \in \mathfrak{a} \implies x \in \mathfrak{p}$ or $y \in \mathfrak{p}$ alternatively if $A/\mathfrak{p}$ is an integral domain.
			
			A proper ideal $\mathfrak{m} \in A$ is called \textbf{maximal} if it is maximal with respect to inclusion alternatively if $A/\mathfrak{m}$ is a field.
		
		\end{minipage}
	};
%------------ Prime and Maximal ideals Header ---------------------
	\node[fancytitle, right=10pt] at (box.north west) {Prime and Maximal ideals};
\end{tikzpicture}

%------------ Local rings ---------------
\begin{tikzpicture}
	\node [mybox] (box){%
		\begin{minipage}{0.3\textwidth}
			F	A ring with exactly one maximal ideal is called a \textbf{local} ring. And its subsequent quotient is called the \textbf{residue field} of the ring. If number of maximal ideals are finite then it is called \textbf{semi local}.
			\begin{itemize}
				\item A ring is local iff its set of non units form an ideal (which is consequently maximal)
				\item If $M\subset A$ is maximal then if $1+m$ is a unit for all $m \in M \implies A$ is local.
				\item A local ring has no idempotents except 0 and 1.
			\end{itemize}
		\end{minipage}
	};
	%------------ Local rings Header ---------------------
	\node[fancytitle, right=10pt] at (box.north west) {Local rings};
\end{tikzpicture}

%------------ Ideal operations ---------------
\begin{tikzpicture}
	\node [mybox] (box){%
		\begin{minipage}{0.3\textwidth}
			For ideals $\mathfrak{a}, \mathfrak{b} \in A $,
			
\begin{itemize}
	\item $\mathfrak{a}+\mathfrak{b}$ forms an ideal and is the smallest ideal containing $\mathfrak{a}$ and $\mathfrak{b}$.
	\item Intersection of ideals is an ideal.
	\item Product of ideals is an ideal.
	\item Unions are in general not ideals.
	\item $\mathfrak{ab} \subseteq \mathfrak{a} \cap \mathfrak{b} \subseteq{a+b}$.
	\item The distributive laws hold.
\end{itemize}		
The \textbf{ideal quotient/colon ideal} is defined as $(\mathfrak{a}:\mathfrak{b})=\{x \in A: x\mathfrak{b} \subseteq \mathfrak a\}$ and $\mathfrak{a} \subseteq (\mathfrak{a}:\mathfrak{b})$.

For a ring homomorphism $\varphi: A \to B$ and some ideals $\mathfrak{a} \in A, \mathfrak{b}\in B$ we define the extension of $\mathfrak{a}$ as $\mathfrak{a}^e$ as the ideal generated by $\varphi(\mathfrak{a})$.

And the contraction of $\mathfrak{b}$ just its preimage in $A$ which is always an ideal.

		\end{minipage}
	};
%------------ Ideal operations Header ---------------------
	\node[fancytitle, right=10pt] at (box.north west) {Ideal operations};
\end{tikzpicture}

%------------ Radical ideals ---------------
\begin{tikzpicture}
	\node [mybox] (box){%
		\begin{minipage}{0.3\textwidth}
			For an ideal $\mathfrak{a}$ its radical ideal is denoted as $\sqrt{\mathfrak{a}}$ or $r(\mathfrak{a})=\{x \in A \ | \ x^n \in \mathfrak{a}\}$
		\end{minipage}
	};
%------------ Radical ideals Header ---------------------
	\node[fancytitle, right=10pt] at (box.north west) {Radical ideals};
\end{tikzpicture}

%------------ Nilradical and Jacobson ideal ---------------
\begin{tikzpicture}
	\node [mybox] (box){%
		\begin{minipage}{0.3\textwidth}
			The \textbf{nilradical} of a ring $A$ is denoted by $ N(R)$ and consists of the set of all nilpotent elements of $A$. Equivalently it is the intersection of all prime ideals.
			
			This shows that a radical of an ideal is just the intersection of prime ideals containing it. 
			
			The \textbf{Jacobson} radical of a ring $A$ denoted by $J(R)$ is the intersection of all its maximal ideals. An element $x$ is in the Jacobson radical $ \iff 1-xy $ is a unit in $ A, \forall y \in A. $
		\end{minipage}
	};
%------------ Nilradical and Jacobson ideal Header ---------------------
	\node[fancytitle, right=10pt] at (box.north west) {Nilradical and Jacobson ideal};
\end{tikzpicture}

%------------ Prime Avoidance theorem  ---------------
\begin{tikzpicture}
	\node [mybox] (box){%
		\begin{minipage}{0.3\textwidth}
			For ring $ A $ consider $ \mathfrak{a} \subset A$ that is stable under addition and multiplication and $ \mathfrak{p}_1, \dots , \mathfrak{p}_n $ ideals such that $ \mathfrak{p}_3, \dots, \mathfrak{p}_n $ are prime in $ A $. If $ \mathfrak{a} $ is contained in the union of all $ \mathfrak{p}_i $ then $ \mathfrak a  \subset \mathfrak p_i$ for some $ i. $
		\end{minipage}
	};
	%------------ Prime Avoidance theorem  Header ---------------------
	\node[fancytitle, right=10pt] at (box.north west) {Prime avoidance theorem};
\end{tikzpicture}



%------------ Chinese Remainder Theorem ---------------
\begin{tikzpicture}
	\node [mybox] (box){%
		\begin{minipage}{0.3\textwidth}
			For a ring $ A $, let $ I_1, \dots, I_n $ be ideals of the ring $ A $. Consider the map $ \pi: A \rightarrow A/I_1 \times \cdots \times A/I_n $ defined as $ \pi(a)=(a\mod I_1, \dots, a \mod I_n) $. Then $ \ker \pi = I_1 \bigcap \cdots \bigcap I_n $, i.e. it is surjective iff $ I_1,\cdots I_n $ are pairwise comaximal. If $ \pi $ is a surjection we have, \[ A/\bigcap I_k = A/\prod I_k \cong \prod (A/I_k)\]
		\end{minipage}
	};
%------------ Chinese Remainder Theorem Header ---------------------
	\node[fancytitle, right=10pt] at (box.north west) {Chinese Remainder Theorem};
\end{tikzpicture}

%------------ Modules  ---------------
\begin{tikzpicture}
	\node [mybox] (box){%
		\begin{minipage}{0.3\textwidth}
			For ring $A$ we define an $ A-$\textbf{module} $M$ to be an abelian group with addition along with scalar multiplication $A \times M \to M$ such that the following hold for $a,b \in A, m,n \in M$
			\begin{itemize}
				\item $a(m+n)=am+an$ and $(a+b)m=am+bm$
				\item $a(bm)=(ab)m$
				\item $1m=m$
			\end{itemize}
We also define the following notions pertaining to modules,
			\begin{itemize}
				\item A \textbf{submodule} $N$ of $M$ is simply a subgroup closed under scalar multiplication.
				\item For a submodule $N$ of $M$ its quotient module $M/N$ is defined naturally.
				\item The submodules of a module form a complete lattice w.r.t. inclusion.
				\item Define an \textbf{annihilator} of $m \in M$ to be $\mathrm{Ann}(m)=\{x \in A \mid xm=0\}$ this extends to the full module as well. They are ideals of $A$. Furthermore $M$ can be seen as a $A/\mathfrak{a}$ module for $\mathfrak{a} \subset \mathrm{Ann}(M)$ naturally.
				
				
			\end{itemize}
		\end{minipage}
	};
	%------------ Modules  Header ---------------------
	\node[fancytitle, right=10pt] at (box.north west) {Modules};
\end{tikzpicture}


%------------ Module homomorphisms.	 ---------------
\begin{tikzpicture}
	\node [mybox] (box){%
		\begin{minipage}{0.3\textwidth}
			For $A-$modules $M,N$ a mapping $f: M \to N$ is a \textbf{module homomorphism} if $f(x+y)=f(x)+f(y)$ and $f(ax)=af(x)$ for $a \in A, x,y \in M$
			\begin{itemize}
				 \item The set of all homomorphisms between modules, i.e. $\mathrm{Hom}_A(M,N)$ determines another $A-$module naturally.
				 \item $\mathrm{Hom}_A(A,M) \cong M$ where $\varphi(f(x))=f(1)$ defines the isomorphism.
				 \item For a module homomorphism $f:M\to M$, its kernel and image are submodules.
				 \item We define cokernel and coimages as the categorical dual of kernels and images. Explicitly as follows, for $f: M \to N$ homomorphism between $A-$modules, $\mathrm{Coker}(f)=N/\mathrm{Im}(f)$ and $\mathrm{Coim}(f)=M/\mathrm{Ker}(f)$
				 \item $\mathrm{Ker}(f)$ has the following UMP, for $i$ defined to be the inclusion,
				 \[\begin{tikzcd}
				 	& N \\
				 	& {\mathrm{Ker}(f)} & M \\
				 	K
				 	\arrow["f"', from=2-3, to=1-2]
				 	\arrow["i", from=2-2, to=2-3]
				 	\arrow["0"', from=2-2, to=1-2]
				 	\arrow["g"', curve={height=12pt}, from=3-1, to=2-3]
				 	\arrow["0", curve={height=-6pt}, from=3-1, to=1-2]
				 	\arrow["{\exists!}"', dashed, from=3-1, to=2-2]
				 \end{tikzcd}\]
				 \item The module of endomorphisms $\mathrm{Hom}_A(M,M)$ also denoted $\mathrm{End}_A(M)$ is in fact a ring (not necessarily commutative). A module is \textbf{faithful} if the map from $A \to \mathrm{End}_A(M)$ is injective equivalently if $\mathrm{Ann}(M)=0$. Also $\mathrm{End}_R(M) \subset \mathrm{End}_{\mathbb{Z}}R(M)$ as we can consider $M$ as an abelian group hence as a $\mathbb{Z}-$module
			\end{itemize}
			We have the following isomorphism theorems in modules, $A-$modules $L \subset N, M \subset N$
			\begin{itemize}
				\item $L+M$ forms a module and $f: L\to L+M \to (L+M)/M$ has kernel $L \cap M$ so we have $L/(L \cap M)\cong (L+M)/M$
				\item If $L \subset M$ we have $N/M \cong (N/L)/(M/L))$ as $f: \to N \to N/L \to (N/L)/(M/L)$ forms a surjection with kernel $M$.
			\end{itemize} 
		\end{minipage}
	};
	%------------  Modules homomorphisms. Header ---------------------
	\node[fancytitle, right=10pt] at (box.north west) {Module homomorphisms};
\end{tikzpicture}



%------------ Nakayama's Lemma	 ---------------
\begin{tikzpicture}
	\node [mybox] (box){%
		\begin{minipage}{0.3\textwidth}
			For $ M $ a finitely generated $ A $ module then for its Jacobson radical $J(A) $ we have $ J(A)M=M \implies M=0 $
		\end{minipage}
	};
	%------------ Nakayama's Lemma Header ---------------------
	\node[fancytitle, right=10pt] at (box.north west) {Nakayama's Lemma};
\end{tikzpicture}

%------------ Localization ---------------
\begin{tikzpicture}
	\node [mybox] (box){%
		\begin{minipage}{0.3\textwidth}
			For a multiplicatively closed subset $S$ of $A$ the \textbf{localization} of $A$ at $S$ denoted as $S^{-1}A$ is a ring which consists of tuples $(a,s)$ for $a \in A, s \in S$ and is characterized by the equivalence relation $$(a,s) \sim (a',s') \iff \exists t \in S, t(as'-sa')=0$$.
			
			For representatives of equivalence classes given as $[(a,s)] , [(a',s')]$ addition is defined as $[(as'+sa',ss')]$ and multiplication is componentwise.
		\end{minipage}
	};
	%------------ Localization Header ---------------------
	\node[fancytitle, right=10pt] at (box.north west) {Localization};
\end{tikzpicture}



% ==============================================================
\end{multicols*}
\end{document}
