\documentclass[dvipsnames]{article}
\usepackage{geometry}
\geometry{a3paper, landscape, margin=2in}
\usepackage{url}
\usepackage{multicol}
\usepackage{esint}
\usepackage{amsfonts}
\usepackage{tikz}
\usetikzlibrary{decorations.pathmorphing}
\usepackage{amsmath,amssymb}
\usepackage{enumitem}
\usepackage{colortbl}
\usepackage{mathtools}
\usepackage{ mathrsfs }	
\usepackage{ dsfont }
\usepackage{ gensymb }

\usepackage[activate={true,nocompatibility},final,tracking=true,kerning=true,spacing=true,factor=1100,stretch=10,shrink=10]{microtype}
\makeatletter
\setlist[itemize]{noitemsep, topsep=0pt}
\usepackage{quiver}
\newcommand*\bigcdot{\mathpalette\bigcdot@{.5}}
\newcommand*\bigcdot@[2]{\mathbin{\vcenter{\hbox{\scalebox{#2}{$\m@th#1\bullet$}}}}}
\makeatother
%Sheet made by Boris, Template is by Drew Ulick https://de.overleaf.com/articles/130-cheat-sheet/ntwtkmpxmgrp
\usepackage[english]{babel}
\usepackage{lmodern}
\usepackage{bm}
	\definecolor{cobalt}{rgb}{0.0, 0.28, 0.67}
\advance\topmargin-1.2in
\advance\textheight3in
\advance\textwidth3in
\advance\oddsidemargin-1.5in
\advance\evensidemargin-1.5in
\parindent0pt
\parskip2pt
\newcommand{\hr}{\centerline{\rule{3.5in}{1pt}}}
\newcommand{\R}{\mathbb{R}}
\newcommand{\Q}{\mathbb{Q}}
\newcommand{\C}{\mathbb{C}}
\newcommand{\Z}{\mathbb{Z}}
\newcommand{\N}{\mathbb{N}}
\newcommand{\D}{\mathbb{D}}
\newcommand{\F}{\mathbb{F}}
% \colorbox[HTML]{e4e4e4}{\makebox[\textwidth-2\fboxsep][l]{texto}}
\begin{document}
\definecolor{orange}{RGB}{255, 115,0}
\begin{center}{\fontsize{35}{60}{\textcolor{DarkOrchid}{\textbf{Algebraic Topology Cheat Sheet}}}}\\
\end{center}
\begin{multicols*}{3}
\tikzset{every loop/.style={min distance=5mm,in=20,out=90,looseness=1}}
\tikzstyle{mybox} = [draw=DarkOrchid, fill=white, very thick,
    rectangle, rounded corners, inner sep=10pt, inner ysep=10pt]
\tikzstyle{fancytitle} =[fill=DarkOrchid	, text=white, font=\bfseries]

% ------------------------------------------------------
% Start of Sheet
% ------------------------------------------------------
%------------ Topology ---------------
\begin{tikzpicture}
	\node [mybox] (box){%
		\begin{minipage}{0.3\textwidth}
			A topology on a set $X$ is the tuple $(X,T)$ where $T$ is a collection of (open) subsets of $X$ such that 
			\begin{itemize}
				\item $\emptyset \in T, X \in T$.
				\item Closed under unions.
				\item Closed under finite intersections.
			\end{itemize}
		\end{minipage}
	};
	%------------ Topology Header ---------------------
	\node[fancytitle, right=10pt] at (box.north west) {Topology};
\end{tikzpicture}

%------------ Constructions on topologies ---------------
\begin{tikzpicture}
	\node [mybox] (box){%
		\begin{minipage}{0.3\textwidth}
			For a topological space $(X,T)$. 
			
			For a subset $Y $ of $X$ there exists a \textbf{subspace topology} naturally endowed upon $Y$ characterized by the following UMP, continuous mappings from another topology $Z $ to $X$ factors through $Y$ with the inclusion mapping, i.e. $g$ is continuous iff $f$ is,
			\[\begin{tikzcd}
				&& X \\
				\\
				Z && Y
				\arrow["{g= f \circ i}", from=3-1, to=1-3]
				\arrow["i"', from=3-3, to=1-3]
				\arrow["f", dashed, from=3-1, to=3-3]
			\end{tikzcd}\]
			
			The \textbf{quotient topology} of $X$ on some set $S$ given a surjection $\pi: X \to X$ is characterized by the following UMP, continuous mappings from $X \to Z$ factor through $S$,
			\[\begin{tikzcd}
				&& X \\
				\\
				Z && S
				\arrow["\pi", from=1-3, to=3-3]
				\arrow["f"', dashed, from=3-3, to=3-1]
				\arrow["{g=\pi\circ f}"', from=1-3, to=3-1]
			\end{tikzcd}\]
			For a family of topologies on $X_a$ the product topology $X = \prod_a X_a$ is characterized by the following UMP, continuous mappings from a topology $Z \to X_a$ factor through $X$ for all $a$,
			\[\begin{tikzcd}
				&& {X_a} \\
				\\
				Z && X
				\arrow["{\pi_a}", from=3-3, to=1-3]
				\arrow["f"', dotted, from=3-1, to=3-3]
				\arrow["{g=f\circ \pi_a}", from=3-1, to=1-3]
			\end{tikzcd}\]
			The coproduct topology (i.e. the disjoint unions) is characterized similarly simply as the dual of the above.
		\end{minipage}
	};
	%------------ Constructions on topologies  Header ---------------------
	\node[fancytitle, right=10pt] at (box.north west) {Constructions on topologies};
\end{tikzpicture}


%------------ Deformation retraction ---------------
\begin{tikzpicture}
\node [mybox] (box){%
    \begin{minipage}{0.3\textwidth}
	    A \textbf{deformation retract} of a space onto a subspace is a family of continuous maps from identity to the subspace and every map restricted to the subspace is identity. In particular for a topology $X$ and subspace $A$ it is a family of maps $f_t(x), t\in [0,1]$ continuous in $x$ such that $f_0=\mathrm{Id}$ and $f_1(X)=A, f_t|_A= \mathrm{Id}$
  	
    \end{minipage}
};
%------------ Deformation retraction Header ---------------------
\node[fancytitle, right=10pt] at (box.north west) {Deformation retraction};
\end{tikzpicture}

%------------ Homotopy ---------------
\begin{tikzpicture}
	\node [mybox] (box){%
		\begin{minipage}{0.3\textwidth}
			A \textbf{homotopy} is a generalization of deformation retractions. It is defined to be a family of maps between arbitrary topologies $X, Y$, $f_t:X\to Y$ for $t \in [0,1]$ such that $f_0=X, f_1=Y$ and the map $H:X\times [0,1]\to Y, H(x,t)=f_t(x)$ is continuous.
			
			Topologies are said to be \textbf{homotopically equivalent} if there exists continuous maps $f:X \to Y$ and $g:Y \to X$ such that $g \circ f$ is homotopic to $\mathrm{Id}_X$ and $f \circ g$ is homotopic to $\mathrm{Id}_Y$.
			
			Homeomorphism $\implies$ homotopy equivalent but the converse need not be true.
			
			A topology is \textbf{contractible} if its identity map is homotopic to a constant function (i.e. \textbf{null homotopic})
		\end{minipage}
	};
	%------------ Homotopy Header ---------------------
	\node[fancytitle, right=10pt] at (box.north west) {Homotopy};
\end{tikzpicture}


%------------ CW Complex ---------------
\begin{tikzpicture}
	\node [mybox] (box){%
		\begin{minipage}{0.3\textwidth}
			A CW complex is a space $X=\cup_n X^n$ where each $X^n$ is inductively built with the following process,
			\begin{itemize}
				\item $X_0$ is an discrete set.
				\item $X_n$ is constructed from $X_{n-1}$ by `attaching' $n-$cells $e_\alpha^n$ (spaces homeomorphic to $n-$dimensional discs $D^n$) this attachment is via continuous maps $\varphi_\alpha: S^{n-1}\to X^{n-1}$ (note that $S^{n-1}$ is just $\partial D^k$), in particular this is just a quotient map where the points on the boundaries are being identified with points in $X_{n-1}$
			\end{itemize}
			To labour the point, $X_n$ is given as the pushout of the corner consisting of the inclusion mapping from the disjoint union of spheres into disks and $\varphi_\alpha$
			
		\end{minipage}
	};
	%------------ CW Complex Header ---------------------
	\node[fancytitle, right=10pt] at (box.north west) {CW Complex};
\end{tikzpicture}

%------------ Constructions on CW complexes ---------------
\begin{tikzpicture}
	\node [mybox] (box){%
		\begin{minipage}{0.3\textwidth}
			
		\end{minipage}
	};
	%------------ Constructions on CW complexes Header ---------------------
	\node[fancytitle, right=10pt] at (box.north west) {Constructions on CW complexes};
\end{tikzpicture}



% ==============================================================
\end{multicols*}
\end{document}
