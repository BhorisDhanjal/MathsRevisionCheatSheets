\documentclass[dvipsnames]{article}
\usepackage{geometry}
\geometry{a3paper, landscape, margin=2in}
\usepackage{url}
\usepackage{multicol}
\usepackage{esint}
\usepackage{amsfonts}
\usepackage{tikz}
\usetikzlibrary{decorations.pathmorphing}
\usepackage{amsmath,amssymb}
\usepackage{enumitem}
\usepackage{colortbl}
\usepackage{mathtools}
\usepackage{ mathrsfs }	
\usepackage{ dsfont }
\usepackage{ gensymb }

\usepackage[activate={true,nocompatibility},final,tracking=true,kerning=true,spacing=true,factor=1100,stretch=10,shrink=10]{microtype}
\makeatletter
\setlist[itemize]{noitemsep, topsep=0pt}
\usepackage{quiver}
\newcommand*\bigcdot{\mathpalette\bigcdot@{.5}}
\newcommand*\bigcdot@[2]{\mathbin{\vcenter{\hbox{\scalebox{#2}{$\m@th#1\bullet$}}}}}
\makeatother
%Sheet made by Boris, Template is by Drew Ulick https://de.overleaf.com/articles/130-cheat-sheet/ntwtkmpxmgrp
\usepackage[english]{babel}
\usepackage{palatino}
\usepackage{bm}
	\definecolor{cobalt}{rgb}{0.0, 0.28, 0.67}
\advance\topmargin-1.2in
\advance\textheight3in
\advance\textwidth3in
\advance\oddsidemargin-1.5in
\advance\evensidemargin-1.5in
\parindent0pt
\parskip2pt
\newcommand{\hr}{\centerline{\rule{3.5in}{1pt}}}
\newcommand{\R}{\mathbb{R}}
\newcommand{\Q}{\mathbb{Q}}
\newcommand{\C}{\mathbb{C}}
\newcommand{\Z}{\mathbb{Z}}
\newcommand{\N}{\mathbb{N}}
\newcommand{\D}{\mathbb{D}}
\newcommand{\F}{\mathbb{F}}
% \colorbox[HTML]{e4e4e4}{\makebox[\textwidth-2\fboxsep][l]{texto}}
\begin{document}
\definecolor{lavender}{RGB}{167,88,220}
\begin{center}{\fontsize{35}{60}{\textcolor{lavender}{\textbf{Introductory Galois Theory Cheat Sheet}}}}\\
\end{center}
\begin{multicols*}{3}

\tikzstyle{mybox} = [draw=lavender, fill=white, very thick,
    rectangle, rounded corners, inner sep=10pt, inner ysep=10pt]
\tikzstyle{fancytitle} =[fill=lavender	, text=white, font=\bfseries]

% ------------------------------------------------------
% Start of Sheet
% ------------------------------------------------------

%------------ Definition of a Field ---------------
\begin{tikzpicture}
\node [mybox] (box){%
    \begin{minipage}{0.3\textwidth}
	    A field $ F $ is a set with two binary operators $ (+,\times) $ satisfying the following axioms,
	    \begin{itemize}
	    	\item (F,+) is an abelian group with identity $ 0 $.
	    	\item The non zero elements of $ F $ form an abelian group under multiplication with identity $ 1 \neq 0 $.
	    	\item Left and right distributivity
	    \end{itemize}
    \end{minipage}
};
%------------ Definition of a Field Header ---------------------
\node[fancytitle, right=10pt] at (box.north west) {Definition of a Field};
\end{tikzpicture}

%------------ Characteristic of Fields ---------------
\begin{tikzpicture}
	\node [mybox] (box){%
		\begin{minipage}{0.3\textwidth}
			A characteristic of a field $ F $, denoted by $ \mathrm{ch}(F) $ is defined as is the smallest integer $ p $ such that $ \underbrace{1+1+\cdots+1}_\text{$p$ times}=0.$ If such a $ p $ does not, exist $ \mathrm{ch}(F)=0. $
		\end{minipage}
	};
%------------ Characteristic of Fields Header ---------------------
	\node[fancytitle, right=10pt] at (box.north west) {Characteristic of Fields};
\end{tikzpicture}

%------------ K-algebra ---------------
\begin{tikzpicture}
	\node [mybox] (box){%
		\begin{minipage}{0.3\textwidth}
			A K-algebra (or algebra over a field) is a ring $ A $ which is a module over field $ K $ with multiplication being K-bilinear, (i.e., $ k_1a_1 \cdot k_2a_2=k_1k_2a_1a_2$).
		\end{minipage}
	};
	%------------ K-algebra Header ---------------------
	\node[fancytitle, right=10pt] at (box.north west) {K-algebra};
\end{tikzpicture}

%------------ Field Extensions ---------------
\begin{tikzpicture}
	\node [mybox] (box){%
		\begin{minipage}{0.3\textwidth}
			For fields $ K, L $. We say $ L $ is a field extension of $ K $ if $K $ is a subfield of $ L $.\\
			Alternatively, $ L $ is a field extension of $ K $, if $ L $ is a K-algebra.	
		\end{minipage}
	};
	%------------ Field Extensions Header ---------------------
	\node[fancytitle, right=10pt] at (box.north west) {Field Extensions};
\end{tikzpicture}

%------------ Algebraic elements and Algebraic extensions---------------
\begin{tikzpicture}
	\node [mybox] (box){%
		\begin{minipage}{0.3\textwidth}
			For a field extension $ K \subset L $.\\
			\textbf{Algebraic element:} $ \alpha \in L $ is called algebraic if $\exists P\not\equiv0 \in K[x] $ s.t. $ P(\alpha)=0.$\\
			\textbf{Transcendental element:} If such a $ P $ does not exist then $ \alpha $	is transcendental. \\
Consider the following definitions, 
			\begin{itemize}
				\item Denote the smallest subfield of $ L $ containing $ K $ and $ \alpha $ to be $ K(\alpha) $.
				\item Denote the smallest sub ring of $ L $ containing $ K $ and $ \alpha  $ to be $ K[\alpha] $.
			\end{itemize}
		The following statements are equivalent,
		\begin{itemize}
			\item $ \alpha $ is algebraic over $ K $.
			\item $ K[\alpha] $ is finite dimensional algebra over $ K $.
			\item $ K[\alpha]=K(\alpha) $.
		\end{itemize}
	\textbf{Algebraic extension:} $ L $ is called algebraic over $ K $ if all $ \alpha \in L$ are algebraic over $ K $.
	\begin{itemize}
		\item If $ L $ is algebraic over $ K $ then any $ K $-subalgebra of $ L $ is a field. 
		\item Consider $ K \subset L \subset M $. If $ \alpha \in M $ is algebraic over $ K $, then it is algebraic over $ L $, also its minimal polynomial over $ L $ divides its minimal polynomial over $ K $.
		\item If $ K \subset L \subset M $ then $ M $ is an algebraic extension over $ K \iff$ $ M $ is algebraic over $ L $ and $ L $ is algebraic over $ K $.
	\end{itemize}
\textbf{Algebraic closure of $L$ over $ K $:} A subfield $ L' $ of $ L $ s.t. $ L'=\{\alpha \in L \mid \alpha \text{ is algebraic over } K \} $

		\end{minipage}
	};
	%------------ Algebraic elements and Algebraic extensions Header ---------------------
	\node[fancytitle, right=10pt] at (box.north west) {Algebraic elements and Algebraic extensions};
\end{tikzpicture}

%------------ Minimal Polynomial ---------------
\begin{tikzpicture}
	\node [mybox] (box){%
		\begin{minipage}{0.3\textwidth}
			If $ \alpha $ is an algebraic element then $\exists ! $ monic polynomial $ P $ of minimal degree such that $ P(\alpha) =0$ such a polynomial is called the \textbf{minimal polynomial}.
			\begin{itemize}
				\item The minimal polynomial is irreducible
				\item Any other polynomial $ Q $ s.t. $ Q(\alpha) =0$ will be divisible by $ P $.
			\end{itemize}
		\end{minipage}
	};
	%------------Minimal Polynomial Header ---------------------
	\node[fancytitle, right=10pt] at (box.north west) {Minimal Polynomial};
\end{tikzpicture}

%------------ Pritmitive polynomials and Gauss' lemma ---------------
\begin{tikzpicture}
	\node [mybox] (box){%
		\begin{minipage}{0.3\textwidth}
			\textbf{Primitive polynomial: }A polynomial $ P \in \Z[X] $ is called primitive if if has a positive degree and the gcd of its coefficients is 1.\\
			\textbf{Gauss' lemma: }A non-constant polynomial $ P\in \Z[X] $ is irreducible over $ \Z[X] \iff $ it is primitive and irreducible over $ \Q[x] $
		\end{minipage}
	};
	%------------ Pritmitive polynomials and Gauss' lemma Header ---------------------
	\node[fancytitle, right=10pt] at (box.north west) {Pritmitive polynomials and Gauss' lemma};
\end{tikzpicture}

%------------ Eisenstein criterion for irreducibility ---------------
\begin{tikzpicture}
	\node [mybox] (box){%
		\begin{minipage}{0.3\textwidth}
			A polynomial $f(x)=a_nx^n+a_{n-1}x^{n-1}+\dots+ a_0\in \Z[x] $ is irreducible if $\exists  p $ prime s.t. $ p $ divides all coefficients except $ a_n $ and $ p^2 $ does not divide $ a_0 $.
		\end{minipage}
	};
	%------------ Eisenstein criterion for irreducibility Header ---------------------
	\node[fancytitle, right=10pt] at (box.north west) {Eisenstein criterion for irreducibility};
\end{tikzpicture}	

%------------ Finite extensions ---------------
\begin{tikzpicture}
	\node [mybox] (box){%
		\begin{minipage}{0.3\textwidth}
		For a field extension $ K \subset L $. $ L $ is called a \textbf{finite extension} of $ K $ if the vector space of $ L $ over $ K $ has a finite dimension.\\
		\textbf{Degree of finite extension:} Denoted as $ [L:K] = \dim_K	L$
		\begin{itemize}
			\item $ K \subset L \subset M $. Then $ M $ is finite over $ K \iff M $ is finite over L and $ L $ is finite over $ K $. Also in this case, $ [M:K]=[M:L][L:K] $.
			\item Let $K(\alpha_1, \dots, \alpha_n) \subset L $ denote the smallest subfield of $ L $ containing $ K $ and $ \alpha_i \in L $. This $ K(\alpha_1, \dots, \alpha_n) $ is generated by $ \alpha_1, \dots, \alpha_n. $
			\item $ L $ is finite over $ K \iff L$ is generated by a finite number of algebraic elements over $ K $ .
			\item $[K(\alpha):K]=\deg P_{\min} (\alpha, K)$
		\end{itemize}
		\end{minipage}
	};
	%------------ Finite extensions Header ---------------------
	\node[fancytitle, right=10pt] at (box.north west) {Finite extensions};
\end{tikzpicture}	




%------------ Stem field ---------------
\begin{tikzpicture}
	\node [mybox] (box){%
		\begin{minipage}{0.3\textwidth}
			Let $ P \in K[X] $ be an irreducible monic polynomial. A field extension $ E $ is called a \textbf{stem field} of $ P $ if $ \exists \alpha \in E$, s.t. $ \alpha $ is a root of $ P $ and $ E=K[\alpha] $
			\begin{itemize}
				\item If $ E, E'$ are two stem fields for $ P \in K[x] $, s.t. $ E=K[\alpha], E'=K[\alpha']$ where $ \alpha, \alpha'  $ are roots of $ P $. Then $ \exists ! $ isomorphism $ E \cong E' $ of K-algebras which maps $ \alpha $ to $ \alpha' $.
				\item If a stem field contains two roots of $ P $, then $ \exists ! $ automorphism that maps one root to another.
				\item If $ E $ is a stem field, $ [E:K] =\deg P$
				\item If $ [E:K] = \deg P$ and $ E $ contains a root of $ P $ then $ E $ is a stem field.
			\end{itemize}
		Some irreducibility criteria,
		\begin{itemize}
			\item $ P \in K[X] $ is irreducible over $ K \iff $ it does not have roots in $ L/K $ of degree $ \leq \deg P/ 2 $. 
			\item $ P \in K[X] $ is irreducible over $K$ with $ \deg P=n $. If $ L/K $ with $ [L:K]=m $ if $ \gcd (m,n)=1 $ then $ P $ is irreducible over $ L $.
		\end{itemize}
		\end{minipage}
	};
	%------------ Stem field Header ---------------------
	\node[fancytitle, right=10pt] at (box.north west) {Stem field};
\end{tikzpicture}	


%------------ Splitting field ---------------
\begin{tikzpicture}
	\node [mybox] (box){%
		\begin{minipage}{0.3\textwidth}
			Let $ P \in K[X] $. The \textbf{splitting field} of $ P $ over $ K $ is an extension of $ L $ where $ P $ is split into linear factors and the roots of $ P $ generate $ L $ (alternatively if $ P $ cannot be factored into any intermediate field smaller than $ L $).
			\begin{itemize}
				\item Splitting field $ L $ exists and its degree is $ \leq d! $, where $ d=\deg P $. And it is unique up to isomorphism as $ K-$algebras.
				\item Degree of the splitting field divides $ d! $.
			\end{itemize}
		\end{minipage}
	};
	%------------ Splitting field Header ---------------------
	\node[fancytitle, right=10pt] at (box.north west) {Splitting field};
\end{tikzpicture}


%------------ Algebraic closure ---------------
\begin{tikzpicture}
	\node [mybox] (box){%
		\begin{minipage}{0.3\textwidth}
			\begin{itemize}
				\item A field $ K $ is algebraically closed if any non-constant polynomial $ P \in K[X] $ has a root in $ K $.
				\item $ L $ is called an \textbf{algebraic closure} of $ K $ if it is algebraically closed and an algebraic extension over $ K $.
				\item Every field has an algebraic closure.
				\item Algebraic closures of $ K $ are unique up to isomorphism as $ K- $algebras.
			\end{itemize}
			
		\end{minipage}
	};
	%------------ Algebraic closure Header ---------------------
	\node[fancytitle, right=10pt] at (box.north west) {Algebraic closure};
\end{tikzpicture}

%------------ Properties of finite fields ---------------
\begin{tikzpicture}
	\node [mybox] (box){%
		\begin{minipage}{0.3\textwidth}
			Let $p$ be a prime integer and let $q=p^r$ for some positive integer $r$. Then the following statements hold.
			\begin{itemize}[noitemsep,topsep=0pt]
				\item There exists a field of order $q.$
				\item Any two fields of order $q$ are isomorphic.
				\item Let $K$ be a field of order $q$. The multiplicative group $K^\times$ of non-zero elements of $K$ is a cyclic group of order $q-1$.
				\item Let $K$ be a field of order $q$. The elements of $K$ are the roots of $x^q-x \in \F_p[x]$.
				\item A field of order $p^r$ contains a field of order $p^k \iff k|r$
				\item The irreducible factors of $x^q-x$ over $\F_p$ are the irreducible polynomials in $\F_p[x]$ whose degree divides $r$.
				\item The splitting field of $ x^q-x $ has $ q $ elements.
				\item $  \F_q$ is a stem field and a splitting field of any irreducible polynomial $ P \in \F_p  $ of degree $ r $.
			\end{itemize}
		\end{minipage}
	};
	%------------ Properties of finite fields Header ---------------------
	\node[fancytitle, right=10pt] at (box.north west) {Properties of finite fields};
\end{tikzpicture}

%------------ Frobenius homomorphism ---------------
\begin{tikzpicture}
	\node [mybox] (box){%
		\begin{minipage}{0.3\textwidth}
			Let $ K $ be a field, $ \mathrm{ch}(K)=p>0 $. There exists a homomorphism $ \varphi: K \rightarrow K $, s.t. $ \varphi(x)=x^p$. This is the Frobenius homomorphism.
			\begin{itemize}
				\item The group of automorphisms over $ \F_{p^r} $ over $ \F_p $ is cyclic and is generated by the Frobenius map.
			\end{itemize}
		\end{minipage}
	};			
	%------------ Frobenius homomorphism Header ---------------------
	\node[fancytitle, right=10pt] at (box.north west) {Frobenius homomorphism};
\end{tikzpicture}

%------------ Separability ---------------
\begin{tikzpicture}
	\node [mybox] (box){%
		\begin{minipage}{0.3\textwidth}
			\begin{itemize}
				\item \textbf{Separable polynomial: }An irreducible polynomial $ P \in K[X] $ is called separable if $ \gcd(P, P')=1 $, i.e. it has distinct roots.
				\item \textbf{Degree of separability: } $ \deg_{\mathrm{sep}} P=\deg Q $ for some $ P(X)=Q\left(X^{p^r}\right) $
				\item \textbf{Degree of inseparability: }$ \deg_i P=\frac{\deg P}{\deg Q} $
				\item \textbf{Purely inseparable polynomial: }$ P $ is purely inseparable if $ \deg_i P = \deg P  $. Also if $ P $ is purely inseparable $ P=X^{p^r}-a$
				\item \textbf{Separable element: }If $ L/K $ is an algebraic extension, then $ \alpha \in L $ is called separable if its minimal polynomial over $ K $ is separable. And vice versa.
				\item If $ \alpha \in K $ is separable then $ |\mathrm{Hom}(K(\alpha), \overline{K})|=\deg P_{\min} (\alpha, K)) $
				\item \textbf{Separable degree: } For $ L/K $, we have $ [	L:K]_{\mathrm{sep}}=|\mathrm{Hom}_K (K(\alpha),\overline{K}	)| $. Inseparable degree is degree of extension divided by separable degree.
				\item \textbf{Separable extension: }$ L $ is separable over $ K $ if $ [L:K]_{\mathrm{sep}}=[L:K]$.
				\begin{itemize}
					\item If $ \mathrm{ch} (K) =0$ then any extension of $ K $ is separable.
					\item If $ \mathrm{ch} (K)=p$ then pure inseparable extension has degree $ p^r $ with degree of inseparability $ p^r $
				\end{itemize}
				\item Separable degrees obey the multiplicative property.
				\item TFAE for finite $ L/K $
				\begin{itemize}
					\item $ L $ is separable over $ K $
					\item Any element of $ L $ is separable over $ K $
					\item $L = K (\alpha_1, \alpha_2, \dots, \alpha_n)$, where each $ \alpha_i  $ is separable over $ K$.
					\item $ L=K(\alpha_1, \alpha_2, \dots, \alpha_n) $, then $ \alpha_i $ is separable over $ K(\alpha_1, \dots, \alpha_{i-1}) $.
				\end{itemize}
				\item \textbf{Separable closure: }$ L^{\mathrm{sep}}=\{x \mid x \text{ separable over } K\}$ for $ x\in \overline{K} $
			\end{itemize}
		\end{minipage}
	};			
	%------------ Separability Header ---------------------
	\node[fancytitle, right=10pt] at (box.north west) {Separability};
\end{tikzpicture}


%------------ Multilinear map ---------------
\begin{tikzpicture}
	\node [mybox] (box){%
		\begin{minipage}{0.3\textwidth}
			For a module $ M $ over ring $ A $. A function $ L $ from $ M^r =\underbrace{M \times M \times \cdots \times M}_{r\text{ times}}$ into $ A $ is called multilinear if $ L(\alpha_1, \dots, \alpha_r) $ is linear as a function of each $ \alpha_i $ when the other $ \alpha_j $ are fixed.
		\end{minipage}
	};			
	%------------ Multilinear map Header ---------------------
	\node[fancytitle, right=10pt] at (box.north west) {Multilinear map };
\end{tikzpicture}

%------------ Tensor product ---------------
\begin{tikzpicture}
	\node [mybox] (box){%
		\begin{minipage}{0.3\textwidth}
			Consider a ring $ A $ and two $ A- $modules, $M,N$. The tensor product is denoted as $ M \otimes_A N$ and is an $ A- $module along with a $ A- $bilinear map, \\$ \varphi:M\times N \rightarrow 	M \otimes_A N $ which satisfies a ``universal property''.\\
			\textbf{Universal property of tensor product:}\\
			For a $ A- $module $ P $, if for an $A-$bilinear map, $ f: M \times N \rightarrow P $, then $  \exists ! $ homomorphism $\tilde{f}$ of $A- $modules s.t. $ f=\tilde{f} \circ \varphi$
			\[\begin{tikzcd}
				{M \times N} && {M \otimes_A N} \\
				\\
				&& P
				\arrow["{\tilde{f}}", dashed, from=1-3, to=3-3]
				\arrow["\varphi", from=1-1, to=1-3]
				\arrow["f"', from=1-1, to=3-3]
			\end{tikzcd}\]
			\begin{itemize}
				\item Commutativity of tensor product $ M \otimes_A N \cong N \otimes_A M $
				\item $A \otimes_A M \cong M$
				\item The basis for the tensor product of free modules is the tensor product of their individual basis elements.
				\item The tensor product is associative.
			\end{itemize}
		\textbf{Base change theorem:} For a ring $ A $, $ B $ an $ A- $algebra, $ M $ an $ A- $module and $N $ a $ B- $module. Then we have the following bijection
		\[ \mathrm{Hom}_A(M,N) \leftrightarrow \mathrm{Hom}_B(B\otimes_A M,N)\]
		\begin{itemize}
			\item For $ I $ an ideal of a ring $ A$ and $ M $ an $ A- $module we have, $ A/I \otimes_A M \cong M/IM $
		\end{itemize}
		\end{minipage}
	};			
	%------------ Tensor product Header ---------------------
	\node[fancytitle, right=10pt] at (box.north west) {Tensor product};
\end{tikzpicture}


%------------ Chinese remainder theorem ideals ---------------
\begin{tikzpicture}
	\node [mybox] (box){%
		\begin{minipage}{0.3\textwidth}
			\textbf{Comaximal ideals: }Two ideals of a ring are called comaximal (or coprime) if their sum gives the ring itself.
			\begin{itemize}
				\item If $ I,J $ are comaximal then $ IJ=I\bigcap J $
				\item If $ I_1,\dots,I_k $ comaximal w.r.t $J $ then $ \prod_{i=1}^k I_i $ is also comaximal with $ J $.
				\item If $ I,J $ are comaximal then so are $ I^m, J^n $ for any $ m,n $.
			\end{itemize}
			\textbf{Chinese remainder theorem: } 
			For a ring $ A $, consider two comaximal ideals $ I,J $, then $ \forall a,b \in R, \exists! x \in A $ s.t. $ x \equiv a (\mathrm{mod } I)$ and $ x \equiv b (\mathrm{mod } J) $\\
			\textbf{Generalized Chinese remainder theorem: } For a ring $ A $, let $ I_1, \dots, I_n $ be ideals of the ring $ A $. Consider the map $ \pi: A \rightarrow A/I_1 \times \cdots \times A/I_n $ defined as $ \pi(a)=(a\mod I_1, \dots, a \mod I_n) $. Then $ \ker \pi = I_1 \bigcap \cdots \bigcap I_n $, i.e. it is surjective iff $ I_1,\cdots I_n $ are pairwise comaximal. If $ \pi $ is a surjection we have, \[ A/\bigcap I_k = A/\prod I_k \cong \prod (A/I_k)\]
		\end{minipage}
	};			
	%------------ Chinese remainder theorem Header ---------------------
	\node[fancytitle, right=10pt] at (box.north west) {Chinese remainder theorem};
\end{tikzpicture}


%------------ Structure of finite algebras ---------------
\begin{tikzpicture}
	\node [mybox] (box){%
		\begin{minipage}{0.3\textwidth}
				Let $ A $ be a finite $ K- $algebra then,
				\begin{itemize}
					\item There are only finitely many maximal ideals in $ A $.
					\item For finitely many maximal ideals $ m_i $. Let $ J=m_1\bigcap \cdots \bigcap m_r.$ Then $ J^n=0 $ for some $ n $.
					\item $ A \cong A/m_1^{n_1} \times \cdots \times A/m_r^{n_r}$ for some (not necessarily unique) $ n_1, \dots, n_r.$
				\end{itemize}	
			\textbf{Reduced \textit{K}-Algebra: }If it has no nilpotent elements.\\
			\textbf{Local ring: }If it has only one maximal ideal. A non zero ring in which every element is either a unit or nilpotent is local.
		\end{minipage}
	};			
	%------------ Structure of finite algebras Header ---------------------
	\node[fancytitle, right=10pt] at (box.north west) {Structure of finite algebras};
\end{tikzpicture}


%------------ Further results on separability  ---------------
\begin{tikzpicture}
	\node [mybox] (box){%
		\begin{minipage}{0.3\textwidth}
			Let $ L $ be a finite extension over $  K $ then the following hold,
			\begin{itemize}
				\item $ L $ is separable $ \iff  L \otimes_K \overline{K} $ is reduced.
				\item $ L $ is purely inseparable $ \iff  L \otimes_K \overline{K} $ is local.
				\item $ L $ is separable $ \iff  \forall $ algebraic extensions $ \Omega, L \otimes_K \Omega $ is reduced.
				\item $ L $ is purely separable $ \iff  \forall $ algebraic extensions $ \Omega, L \otimes_K \Omega $ is local.
				\item If $ L $ is separable then the map $ \varphi: L \otimes_K \overline{K} \rightarrow \overline{K}^n $ defined as $ \varphi(l \otimes k) = (k \varphi_1(l),\dots, k \varphi_n(l))$ (where $\varphi_i  $ are distinct homomorphisms from $ L $ to $ \overline{K} $), is an isomorphism.
				\item Let $ L $ be a finite separable extension of $ K $ then it has only finitely many intermediate extensions.	
			\end{itemize}
		\end{minipage}
	};			
	%------------ Further results on separability Header ---------------------
	\node[fancytitle, right=10pt] at (box.north west) {Further results on separability};
\end{tikzpicture}

%------------ Primitive element theorem  ---------------
\begin{tikzpicture}
	\node [mybox] (box){%
		\begin{minipage}{0.3\textwidth}
			There exists $ \alpha \in L  $ s.t. $ L=K(\alpha) $ whenever $ L $ is finite and separable.
		\end{minipage}
	};			
%------------ Primitive element theorem Header ---------------------
	\node[fancytitle, right=10pt] at (box.north west) {Primitive element theorem};
\end{tikzpicture}


%------------ Normal extensions  ---------------
\begin{tikzpicture}
	\node [mybox] (box){%
		\begin{minipage}{0.3\textwidth}
			A \textbf{normal extension} of $ K $ is an algebraic extension which is a splitting field of a family of polynomials in $ K[X] $.\\
			TFAE for an extension $ L $ of $ K $,
			\begin{itemize}
				\item $ \forall x \in L, P_{\min} (x,K) $ splits in $ L $.
				\item $ L $ is a normal extension.
				\item All homomorphisms from $ L $ to $ \overline{K} $ have the same image.
				\item The group of automorphisms, $ \mathrm{Aut}(L/K) $ acts transitively on $ \mathrm{Hom}_K(L,\overline{K}) $.
			\end{itemize}
		Some properties of normal extensions,
		\begin{itemize}
			\item $ K \subset L \subset M $, if $ M $ is normal over $ K $ then it is normal over $ L $, but $ L $ need not be normal over $ K $.
			\item Extensions with degree 2 are normal.	
			
		\end{itemize}
		\end{minipage}
	};			
%------------ Normal extensions Header ---------------------
	\node[fancytitle, right=10pt] at (box.north west) {Normal extensions};
f\end{tikzpicture}

%------------ Galois extensions ---------------
\begin{tikzpicture}
	\node [mybox] (box){%
		\begin{minipage}{0.3\textwidth}
			
			An algebraic extension that is both normal and separable is called a\textbf{ Galois extension.}
			\begin{itemize}
				\item For a finite extension $ L $ over $ K $ the number of automorphisms $ |\mathrm{Aut}(L/K)| \leq [L:K]$. Equality holds iff $ L $ is a Galois` extension. 
			\end{itemize}

		If $ L $ is normal over $ K $ then,
		\begin{itemize}
			\item Isomorphism of sub extensions extend to automorphisms of $ L $.
			\item $ \mathrm{Aut} (L/K)$ acts transitively on the roots of any irreducible polynomial in $ K[X] $.
			\item If $ \mathrm{Aut} (L/K)$ fixes $ x \notin K$. Then $ x $ is purely inseparable.
		\end{itemize}
	
		\end{minipage}
	};			
%------------ Galois extensions Header ---------------------
	\node[fancytitle, right=10pt] at (box.north west) {Galois extensions};
	f\end{tikzpicture}

%------------ Galois group ---------------
\begin{tikzpicture}
	\node [mybox] (box){%
		\begin{minipage}{0.3\textwidth}
		If $ L $ is a Galois extension, $ G=\mathrm{Gal}(L/K) = \mathrm{Aut}(L/K)$ is called the \textbf{Galois group} of the extension.
		\begin{itemize}
			\item $ L^{\mathrm{Gal}(L/K) } =K $, (i.e. the set of invariants in $ L $ with the action of the Galois group is equal to $ K $).
			\item Let $ L $ be a field and $ G $ a subgroup of $ \mathrm{Aut} (L)$, then
			\begin{itemize}
				\item If all orbits of $ G $ are finite, then $ L $ is a Galois extension of $ L^G $.
				\item If order of $ G $ is finite then, $ [L:L^G] = |G|$ and $ G $ is a Galois group.
			\end{itemize}
		\end{itemize}
	
		\end{minipage}
	};			
%------------ Galois group Header ---------------------
	\node[fancytitle, right=10pt] at (box.north west) {Galois groups};
\end{tikzpicture}

%------------ The Fundamental theorem of Galois theory ---------------
\begin{tikzpicture}
	\node [mybox] (box){%
		\begin{minipage}{0.3\textwidth}
			Let $ L/K $ be a Galois extension, and $ \mathrm{Aut}(L/K) =\mathrm{Gal} (L/K)$ is its Galois group.
			\begin{itemize}
				\item If $ L $ is finite over $ K $, then for a intermediate field $ F $ and a subgroup $ H \subset \mathrm{Gal} (L/K)$ we have the following correspondence,
				\begin{itemize}
					\item $ F \rightarrow \mathrm{Gal}(L/F) $
					\item $ H \rightarrow L^H $
				\end{itemize}
			\item $ F $ is Galois over $ K \iff g(F)=F,\ \forall g\in \mathrm{Gal}(L/K)  \iff$ $ \mathrm{Gal}(L/F) \trianglelefteq \mathrm{Gal}(L/K)$ 
			\end{itemize}
		\end{minipage}
	};			
%------------ The Fundamental theorem of Galois theory Header ---------------------
	\node[fancytitle, right=10pt] at (box.north west) {The Fundamental theorem of Galois theory};
\end{tikzpicture}


%------------ Discriminant ---------------
\begin{tikzpicture}
	\node [mybox] (box){%
		\begin{minipage}{0.3\textwidth}
			For a polynomial $ P $ with roots $ x_i $, the \textbf{discriminant} is $ \Delta = \prod_{i<j} (x_i-x_j)^2$. For $ \mathrm{Gal}(P) \subset S_n$. For a separable polynomial,
			\begin{itemize}
				\item $ \Delta $ is preserved by any permutation.
				\item $ \sqrt{\Delta} $ is preserved only by even permutations
				\item $ G \subset A_n \iff \sqrt{\Delta}\in K$
			\end{itemize}
		\end{minipage}
	};			
%------------ Discriminant Header ---------------------
	\node[fancytitle, right=10pt] at (box.north west) {Discriminant};
\end{tikzpicture}

%------------ Cyclotomic polynomials and extensions ---------------
\begin{tikzpicture}
	\node [mybox] (box){%
		\begin{minipage}{0.3\textwidth}
			Let $ P_n=X^n -1$ where $ p \nmid n $ if $ \mathrm{ch}(K)=p>0  $. \\
			$ P_n $ has $ n $ distinct roots which form a cyclic multiplicative subgroup $ \mu_n \subset \overline{K}^\times$.\\
			Let $ \mu_n* $ denote the set of \textbf{primitive $ \mathbf{n^{th}} $ roots of unity} (no roots of degree $ <n $).
			\begin{itemize}
				\item $ |\mu_n*|=\varphi(n) $
			\end{itemize}
		\textbf{Cyclotomic polynomials: }$ \Phi_n=\prod_{\alpha \in \mu_n*} (X-\alpha) \in \overline{K}[X].$
		\begin{itemize}
			\item $ P_n=\prod_{d \mid n} \Phi_d $.
			\item $ \Phi_n  $ has coefficients in prime fields.
			\item If $ \mathrm{ch} (K)=0$ then $ \Phi_n\in \Z[X] $, else if $ \mathrm{ch} (K)=p$, we have $ \Phi_n  $ is the reduction mod $ p $ of the $ n^{th }$ cyclotomic polynomial over $ \Z $.
			\item If $ \mathrm{ch} (K)=0, $ then $ \Phi_n  $ is irreducible over $ \Z[X] $.
		\end{itemize}
		Consider $ L $, splitting field of $ K $
		\begin{itemize}
			\item The splitting field of $ P_n $ over $ K $ is $ K(\zeta ) $ where $ \zeta  $ is a root of $ \Phi_n	 $.
			\item All $ g \in \mathrm{Gal} (L/K)\texttt{}$ acts as $\zeta \rightarrow \zeta^{a^g}, (a^g,n) =1 $.
			\item $ \mathrm{Gal}(L/K)$ injects into $ \Z/n\Z^\times $ and this is an isomorphism when $ \Phi_n  $ is irreducible over $ K $.
		\end{itemize}
		\end{minipage}
	};			
%------------ Cyclotomic polynomial and extensions Header ---------------------
	\node[fancytitle, right=10pt] at (box.north west) {Cyclotomic polynomials and extensions};
\end{tikzpicture}



%------------ Kummer extensions ---------------
\begin{tikzpicture}
	\node [mybox] (box){%
		\begin{minipage}{0.3\textwidth}
			A field extension $ L/K $ is called a \textbf{Kummer extension} if for some integer $ n>1 $
			\begin{itemize}
				\item $ K $ contains $ n $ distinct $ n^{th} $ roots of unity.
				\item $\mathrm{Gal}(L/K)$ is abelian group with lcm of the orders of group elements (exponent	) equal to $ n $.
			\end{itemize}
		Consider $ K $ s.t. for some $ n, (\mathrm{ch}(K),n)=1$ and $ X^n-1 $ splits in $ K $, for any $  a \in K$ take $ d= \min\{i \mid a^{i/n}\in K	\} $ then we have,
		\begin{itemize}
			\item $ d \mid n $ and $ P_{\min}(a^{1/n})= X^d-a^{d/n}$
			\item $ K(a^{1/n}) $ is Galois extension with cyclic Galois group of order $ d $.
		\end{itemize}
	The converse is also true.
		\end{minipage}
	};			
%------------ Kummer extensions Header ---------------------
	\node[fancytitle, right=10pt] at (box.north west) {Kummer extensions};
\end{tikzpicture}


%------------ Artin-Schreier extensions ---------------
\begin{tikzpicture}
	\node [mybox] (box){%
		\begin{minipage}{0.3\textwidth}
			Let $ L/K $ be a field extension s.t. $ \mathrm{ch}(K)=p $ for prime $ p $. It is called \textbf{Artin-Schreier extension} if degree of extension $ L $ is $ p $.\\
			\textbf{Artin-Schreier theorem: }Let $ \mathrm{ch}(K)-p$ and let $ P=X^p-X-a \in K[X] $. Then $ P $ is either irreducible or splits in $ K $. Let $ \alpha  $ be a root of $ P $.
			\begin{itemize}
				\item If $ P $ is irreducible, then $ K(\alpha ) $ is a cyclic extension (i.e. Galois group is cyclic) of $ K $ of degree $ p $.
				\item Any cyclic extension of degree $ p $ is obtained in the same way.
			\end{itemize}
		\end{minipage}
	};			
%------------ Artin-Schreier extensions Header ---------------------
	\node[fancytitle, right=10pt] at (box.north west) {Artin-Schreier extensions};
\end{tikzpicture}

%------------ Composite extensions ---------------
\begin{tikzpicture}
	\node [mybox] (box){%
		\begin{minipage}{0.3\textwidth}
			Let $ L_1,L_2 $ be two intermediate extensions of $ K $ and some $ L/K $ that contains them both. Then $ L_1L_2=L_2L_1=K(L_1\bigcup L_2) $ the smallest extension that contains both $ L_1, L_2 $ is called \textbf{composite extension.}
			\begin{itemize}
				\item If $ L_1 $ and $ L_2 $ are separable/purely inseparable/normal/finite over $ K $ then its composite field also possess that property.
			\end{itemize}
		
		\end{minipage}
	};			
%------------ Composite extensions Header ---------------------
	\node[fancytitle, right=10pt] at (box.north west) {Composite extensions};
\end{tikzpicture}

%------------ Linearly disjoint extensions ---------------
\begin{tikzpicture}
	\node [mybox] (box){%
		\begin{minipage}{0.3\textwidth}
			TFAE for algebraic extensions,
			\begin{itemize}
				\item $ L_1 \otimes_K L_2 $ is a field.
				\item $ L_1 \otimes_K L_2 \rightarrow L $ is an injection.
				\item A linearly independent set in $ L_1 $ is also linearly independent in $ L_2 $.
				\item For linearly independent sets (over $ K $) $ A \in L_1, B \in L_2 $ we have $ A \times B $ is linearly independent over $ K $
			\end{itemize}
		$ L_1,L_2 $ satisfying these properties are called \textbf{linearly disjoint extensions.}
		\begin{itemize}
			\item If $ \deg L_1 $ is finite then $ [L_1L_2:L_2] = [L_1:K]$ equivalently $ [L_1L_2:K] =[L_1:K][L_2:K]$
			\item Extensions which are relatively prime degrees are linearly disjoint.	
		\end{itemize}
	For $ \overline{K} $ the algebraic closure of $ K $,
	\begin{itemize}
		\item Let $ L_1,L_2 \subset \overline{K}$, if $ L_1 $ is Galois over $ K $ and let $ K'=L_1 \bigcap L_2 $. Then $ L_1L_2 $ is Galois over $ L_2 $. The map $ \phi: g \rightarrow g|_{L_1} $ of $ \mathrm{Gal}(L_1L_2/L_2) \rightarrow \mathrm{Gal} (L_1/K) $ is injective with image $ \mathrm{Gal} (L_1/K') $ and $ L_1,L_2 $ linearly disjoint over $ K'. $
	\end{itemize}
		\end{minipage}
	};			
%------------ Linearly disjoint extensions Header ---------------------
	\node[fancytitle, right=10pt] at (box.north west) {Linearly disjoint extensions};
\end{tikzpicture}

%------------ Solvable extensions and polynomials ---------------
\begin{tikzpicture}
	\node [mybox] (box){%
		\begin{minipage}{0.3\textwidth}
			 \textbf{Solvable extension: }A finite extension $ E $ of $ K $ is solvable by radicals if $\exists \alpha_1,\cdots,\alpha_r  $ generating $ E $ such that $ \alpha_i^{n_i} \in K(\alpha_1, \dots, \alpha_{i-1})$ for some $ n_i $.\\
			\textbf{Solvable polynomials: } $ P \in K[X] $ is solvable by radicals if $ \exists  $ a solvable extension $ E/K $ containing its roots.
			\begin{itemize}
				\item A composite of solvable extensions is solvable.
				\item For finite $L/K$ solvable $ \implies  \exists $ finite Galois extension also solvable when $ \mathrm{ch}(K)=0 $.	
			\end{itemize}
			
			
		\end{minipage}
	};			
%------------ Solvable extensions and polynomials Header ---------------------
	\node[fancytitle, right=10pt] at (box.north west) {Solvable extensions and polynomials};
\end{tikzpicture}


%------------ Solvable groups ---------------
\begin{tikzpicture}
	\node [mybox] (box){%
		\begin{minipage}{0.3\textwidth}
		A group $ G $ is called \textbf{solvable} if it has a finite sequence of normal subgroups, ($ I=G_0 \trianglelefteq G_1 \trianglelefteq \cdots \trianglelefteq G_r = G $) and also $ G_{i+1}/G_i$ is abelian.
		\begin{itemize}
			\item Subgroups of solvable groups are solvable.
			\item If $ G $ is solvable and $ H \trianglelefteq G $ then $ G/H $ is solvable.
			\item If $ G $ if a finite abelian group then $ G $ is solvable
			\item $ S_n $ is not solvable for $ n \geq 5. $
		\end{itemize}
		\end{minipage}
	};			
%------------ Solvable groups Header ---------------------
	\node[fancytitle, right=10pt] at (box.north west) {Solvable groups};
\end{tikzpicture}


%------------ Solvability by radicals  ---------------
\begin{tikzpicture}
	\node [mybox] (box){%
		\begin{minipage}{0.3\textwidth}
			Let	$ P \in K[X], \mathrm{ch}(K)=0$. $ P $ is a polynomial solvable by radicals iff $\mathrm{Gal}(P) $ is solvable. Here $\mathrm{Gal}(P)=\mathrm{Gal}(F/K) $, where $ F $ is a splitting field of $ P $ over $ K $.
		\end{minipage}
	};			
%------------ Solvability by radicals Header ---------------------
	\node[fancytitle, right=10pt] at (box.north west) {Solvability by radicals};
\end{tikzpicture}

%------------ Abel-Ruffini theorem  ---------------
\begin{tikzpicture}
	\node [mybox] (box){%
		\begin{minipage}{0.3\textwidth}
			General polynomials of degree $ n\geq 5 $ are not solvable by radicals since $ S_n $ for $ n\geq5  $ is not solvable.
		\end{minipage}
	};			
%------------ Abel-Ruffini theorem Header ---------------------
	\node[fancytitle, right=10pt] at (box.north west) {Abel-Ruffini theorem};
\end{tikzpicture}

%------------ Group representations ---------------
\begin{tikzpicture}
	\node [mybox] (box){%
		\begin{minipage}{0.3\textwidth}
			For vector space $V$, a \textbf{representation} of a finite group $G$ is a homomorphism $ \varphi: G \rightarrow GL(V) $, where $ GL(V) $ is the 		group of automorphisms of $ V $.\\
			\textbf{Regular representation:} For vector space $ V $ generated by elements of group $ G $. A homomorphism involving permuting this basis is called regular.
			\begin{itemize}
				\item For $ L/K $ as a vector space over $ K $ we have a representation of the Galois group $ \varphi: \mathrm{Gal}(L/K) \rightarrow GL_K(L)$. This is a regular representation.
			\end{itemize}
		\end{minipage}
	};			
%------------ Group representations Header ---------------------
	\node[fancytitle, right=10pt] at (box.north west) {Group representations};
\end{tikzpicture}

%------------ Normal basis theorem ---------------
\begin{tikzpicture}
	\node [mybox] (box){%
		\begin{minipage}{0.3\textwidth}
			For $ L/K $ a finite Galois extension, $\exists x\in L/K $ s.t. $ \{gx \mid g \in G\} $ is a $ K- $basis of $ L $.
		\end{minipage}
	};			
%------------ Normal basis theorem Header ---------------------
	\node[fancytitle, right=10pt] at (box.north west) {Normal basis theorem};
\end{tikzpicture}

%------------ Integral elements---------------
\begin{tikzpicture}
	\node [mybox] (box){%
		\begin{minipage}{0.3\textwidth}
			\textbf{Integral elements: }For a integral domain $ A $ and $ B $ an extension ring of $ A $. An element $ \alpha \in B  $ is said to be integral over $ A $ if $ \alpha $ is the root of a monic polynomial in $ A[X] $.\\
			TFAE,
			\begin{itemize}
				\item $ \alpha  $ is integral over $ A $.
				\item $ A[\alpha ] $ is a finitely generated $ A- $module.
				\item $ A[\alpha ] \subset C \subset B$ where $ C $ is a finitely generated $ A $ module.
			\end{itemize}
		\end{minipage}
	};			
%------------ Integral elements Header ---------------------
	\node[fancytitle, right=10pt] at (box.north west) {Integral elements};
\end{tikzpicture}
%------------ Field Norm and Trace ---------------
\begin{tikzpicture}
	\node [mybox] (box){%
		\begin{minipage}{0.3\textwidth}
			Let $ K \hookrightarrow E $ be a separable field extension, for $ \alpha \in K $ its field norm is defined as $ N_{E/K} (\alpha) = \prod_{\sigma_i: E \hookrightarrow \overline{K}} \sigma_i(\alpha) $. The trace ($ \mathrm{Tr} $) is the same with sum instead.
			\begin{itemize}
				\item Norm is multiplicative, trace is additive and $ k- $linear.
				\item If $ E=K(\alpha) $, $ N_{E/K} =(-1)^{[E:K]}(\text{Constant coeff of } P_{\min}(\alpha, K))$, $ \mathrm{Tr}_{E/K} (\alpha ) = -(\text{Coefficient of }X^{[E:K]-1})$.
				\item For a tower $ K \subset F \subset E $, $ N_{E/K} =N_{F/K} \circ N_{E/F}, \mathrm{Tr}_{E/K}=\mathrm{Tr}_{F/K} \circ \mathrm{Tr}_{E/F}$.
				\item $ T: E \times E \rightarrow K$ as $ (x,y)\rightarrow \mathrm{Tr}(x,y) $ is a non-degenerate $ K-$bilinear.
				\item If $ \alpha  $ is integral over $ \Z. $ Then $ N_{E/\Q}(\alpha),\mathrm{Tr}_{E/\Q}(\alpha ) $ are integers.
			\end{itemize}
		\end{minipage}
	};			
	%------------ Field Norm and Trace Header ---------------------
	\node[fancytitle, right=10pt] at (box.north west) {Field Norm and Trace};
\end{tikzpicture}

%------------ Integral extensions, closures---------------
\begin{tikzpicture}
	\node [mybox] (box){%
		\begin{minipage}{0.3\textwidth}
		\textbf{Integral extension: }For $ A \subset B, B$ is said to be an integral extension of $ A $ if every element of $ B $ is an integral element over $ A $.
		\begin{itemize}
			\item $ A \subset B \subset C $ if $ B $ is integral over $ A $ and $ C $ integral over $ B \implies C$ is integral over $ A $.
			\item $ B $ is finitely generated over $ A $ as a module $ \iff B = A[\alpha_1, \dots, \alpha_r] $ where each $ \alpha_i  $ is integral over $ A $.
			\item Elements of $ B $ integral over $ A $ forms a subring of $ B $. This is the integral closure of $ A $ in $ B $.
		\end{itemize}
		\textbf{Integrally closed: }$ A $ is integrally closed in $ B $ if the integral closure of $ A $ in $ B $ is same as $ A $. In general $ A $ is integrally closed if $ A $ is integrally closed in its field of fractions.
		\begin{itemize}
			\item $ \Z  $ is integrally closed.
			\item Any UFD is integrally closed.
		\end{itemize}
		Let $ K $ be a Number field, the integral closure of $ \Z  $ in $ K $ is $ O_K $ the ring of integers.
		\begin{itemize}
			\item $ \forall \alpha \in K, $ there exists $ d \in \Z^* $ such that $ d \alpha \in O_K. $
			\item $ \alpha \in O_K \implies P_{\min}(\alpha, \Q) \in Z[X] $.
			\item $ O_K $ is a finitely generated, free $ \Z- $module of rank $ n=[K,\Q ]$.
		\end{itemize}
		\end{minipage}
	};			
	%------------ Integral extensions, closures Header ---------------------
	\node[fancytitle, right=10pt] at (box.north west) {Integral extensions, closures};
\end{tikzpicture}



%------------ Reduction modulo prime ---------------
\begin{tikzpicture}
	\node [mybox] (box){%
		\begin{minipage}{0.3\textwidth}
			Let $ P \in \Z[X] $ be an irreducible polynomial, and $ K $ its splitting field over $ \Q $. With $ [K:\Q ] =n$. Let $ G=\mathrm{Gal}(P) $. Let $ \alpha_1, \dots, \alpha_n  $ be roots of $ P $. Consider $ A=O_K $ and let $ J_1, \dots, J_r $ be all the maximal ideals of $ A $ containing some prime $ p $. Consider $ D_i \subset G, D_i = \{g \in G \mid g J_i=J_i\} $ and let $ k_i=A/J_i $. There exists a natural homomorphism $ D_i \rightarrow \mathrm{Gal}(k_i,\F_p) $\\
			We then have the following,
			\begin{itemize}
				\item $ G $ acts transitively on $ \{J_1,\dots,J_r\} $ and $ D_i $ maps surjectively into $ \mathrm{Gal} (k_i/\F_p)$.
				\item If reduction $ \overline{P} = P \mod p $ does not have multiple roots then the map $ D_i \leftrightarrow \mathrm{Gal} (k_i/\F_p )$ is a bijection and $ k_i $ is a splitting field of $ \overline{P} $ for some $ i $.
			\end{itemize}
			\textbf{Example: }If for $ P \in \Z[X] $ is irreducible and $ \exists  $ prime $ p $ such that $ \overline{P} =P \mod p$ is also irreducible. Then we have that $ \mathrm{Gal} (P)$ contains an $ n- $cycle permutation.
		\end{minipage}
	};			
	%------------ Reduction modulo prime Header ---------------------
	\node[fancytitle, right=10pt] at (box.north west) {Reduction modulo prime};
\end{tikzpicture}

% ==============================================================
\end{multicols*}
\end{document}
