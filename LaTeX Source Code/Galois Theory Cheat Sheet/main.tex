\documentclass[dvipsnames]{article}
\usepackage{geometry}
\geometry{a3paper, landscape, margin=2in}
\usepackage{url}
\usepackage{multicol}
\usepackage{esint}
\usepackage{amsfonts}
\usepackage{tikz}
\usetikzlibrary{decorations.pathmorphing}
\usepackage{amsmath,amssymb}
\usepackage{enumitem}
\usepackage{colortbl}
\usepackage{mathtools}
\usepackage{ mathrsfs }	
\usepackage{ dsfont }
\usepackage{ gensymb }

\usepackage[activate={true,nocompatibility},final,tracking=true,kerning=true,spacing=true,factor=1100,stretch=10,shrink=10]{microtype}
\makeatletter
\setlist[itemize]{noitemsep, topsep=0pt}
\usepackage{quiver}
\newcommand*\bigcdot{\mathpalette\bigcdot@{.5}}
\newcommand*\bigcdot@[2]{\mathbin{\vcenter{\hbox{\scalebox{#2}{$\m@th#1\bullet$}}}}}
\makeatother
%Sheet made by Boris, Template is by Drew Ulick https://de.overleaf.com/articles/130-cheat-sheet/ntwtkmpxmgrp
\usepackage[english]{babel}
\usepackage{palatino}
\usepackage{bm}
	\definecolor{cobalt}{rgb}{0.0, 0.28, 0.67}
\advance\topmargin-1.2in
\advance\textheight3in
\advance\textwidth3in
\advance\oddsidemargin-1.5in
\advance\evensidemargin-1.5in
\parindent0pt
\parskip2pt
\newcommand{\hr}{\centerline{\rule{3.5in}{1pt}}}
\newcommand{\R}{\mathbb{R}}
\newcommand{\Q}{\mathbb{Q}}
\newcommand{\C}{\mathbb{C}}
\newcommand{\Z}{\mathbb{Z}}
\newcommand{\N}{\mathbb{N}}
\newcommand{\D}{\mathbb{D}}
\newcommand{\F}{\mathbb{F}}
% \colorbox[HTML]{e4e4e4}{\makebox[\textwidth-2\fboxsep][l]{texto}}
\begin{document}
\definecolor{lavender}{RGB}{167,88,220}
\begin{center}{\fontsize{35}{60}{\textcolor{lavender}{\textbf{Introductory Galois Theory Cheat Sheet}}}}\\
\end{center}
\begin{multicols*}{3}

\tikzstyle{mybox} = [draw=lavender, fill=white, very thick,
    rectangle, rounded corners, inner sep=10pt, inner ysep=10pt]
\tikzstyle{fancytitle} =[fill=lavender	, text=white, font=\bfseries]

% ------------------------------------------------------
% Start of Sheet
% ------------------------------------------------------

%------------ Definition of a Field ---------------
\begin{tikzpicture}
\node [mybox] (box){%
    \begin{minipage}{0.3\textwidth}
	    A field $ F $ is a set with two binary operators $ (+,\times) $ satisfying the following axioms,
	    \begin{itemize}
	    	\item (F,+) is an abelian group with identity $ 0 $.
	    	\item The non zero elements of $ F $ form an abelian group under multiplication with identity $ 1 \neq 0 $.
	    	\item Left and right distributivity
	    \end{itemize}
    \end{minipage}
};
%------------ Definition of a Field Header ---------------------
\node[fancytitle, right=10pt] at (box.north west) {Definition of a Field};
\end{tikzpicture}

%------------ Characteristic of Fields ---------------
\begin{tikzpicture}
	\node [mybox] (box){%
		\begin{minipage}{0.3\textwidth}
			A characteristic of a field $ F $, denoted by $ \text{ch}(F) $ is defined as is the smallest integer $ p $ such that $ \underbrace{1+1+\cdots+1}_\text{$p$ times}=0.$ If such a $ p $ does not, exist $ \text{ch}(F)=0. $
		\end{minipage}
	};
%------------ Characteristic of Fields Header ---------------------
	\node[fancytitle, right=10pt] at (box.north west) {Characteristic of Fields};
\end{tikzpicture}

%------------ K-algebra ---------------
\begin{tikzpicture}
	\node [mybox] (box){%
		\begin{minipage}{0.3\textwidth}
			A K-algebra (or algebra over a field) is a ring $ A $ which is a module over field $ K $ with multiplication being K-bilinear, (i.e., $ k_1a_1 \cdot k_2a_2=k_1k_2a_1a_2$).
		\end{minipage}
	};
	%------------ K-algebra Header ---------------------
	\node[fancytitle, right=10pt] at (box.north west) {K-algebra};
\end{tikzpicture}

%------------ Field Extensions ---------------
\begin{tikzpicture}
	\node [mybox] (box){%
		\begin{minipage}{0.3\textwidth}
			For fields $ K, L $. We say $ L $ is a field extension of $ K $ if $K $ is a subfield of $ L $.\\
			Alternatively, $ L $ is a field extension of $ K $, if $ L $ is a K-algebra.	
		\end{minipage}
	};
	%------------ Field Extensions Header ---------------------
	\node[fancytitle, right=10pt] at (box.north west) {Field Extensions};
\end{tikzpicture}

%------------ Algebraic elements and Algebraic extensions---------------
\begin{tikzpicture}
	\node [mybox] (box){%
		\begin{minipage}{0.3\textwidth}
			For a field extension $ K \subset L $.\\
			\textbf{Algebraic element:} $ \alpha \in L $ is called algebraic if $\exists P \in K[x] $ s.t. $ P(\alpha)=0.$\\
			\textbf{Transcendental element:} If such a $ P $ does not exist then $ \alpha $	is transcendental. \\
Consider the following definitions, 
			\begin{itemize}
				\item Denote the smallest subfield of $ L $ containing $ K $ and $ \alpha $ to be $ K(\alpha) $.
				\item Denote the smallest sub ring of $ L $ containing $ K $ and $ \alpha  $ to be $ K[\alpha] $.
			\end{itemize}
		The following statements are equivalent,
		\begin{itemize}
			\item $ \alpha $ is algebraic over $ K $.
			\item $ K[\alpha] $ is finite dimensional algebra over $ K $.
			\item $ K[\alpha]=K(\alpha) $.
		\end{itemize}
	\textbf{Algebraic extension:} $ L $ is called algebraic over $ K $ if all $ \alpha \in L$ are algebraic over $ K $.
	\begin{itemize}
		\item If $ L $ is algebraic over $ K $ then any $ K $-subalgebra of $ L $ is a field. 
		\item Consider $ K \subset L \subset M $. If $ \alpha \in M $ is algebraic over $ K $, then it is algebraic over $ L $, also its minimal polynomial over $ L $ divides its minimal polynomial over $ K $.
		\item If $ K \subset L \subset M $ then $ M $ is an algebraic extension over $ K \iff$ $ M $ is algebraic over $ L $ and $ L $ is algebraic over $ K $.
	\end{itemize}
\textbf{Algebraic closure:} A subfield $ L' $ of $ L $ s.t. $ L'=\{\alpha \in L \mid \alpha \text{ is algebraic over } K \} $

		\end{minipage}
	};
	%------------ Algebraic elements and Algebraic extensions Header ---------------------
	\node[fancytitle, right=10pt] at (box.north west) {Algebraic elements and Algebraic extensions};
\end{tikzpicture}

%------------ Minimal Polynomial ---------------
\begin{tikzpicture}
	\node [mybox] (box){%
		\begin{minipage}{0.3\textwidth}
			If $ \alpha $ is an algebraic element then $\exists ! $ monic polynomial $ P $ of minimal degree such that $ P(\alpha) =0$ such a polynomial is called the \textbf{minimal polynomial}.
			\begin{itemize}
				\item The minimal polynomial is irreducible
				\item Any other polynomial $ Q $ s.t. $ Q(\alpha) =0$ will be divisible by $ P $.
			\end{itemize}
		\end{minipage}
	};
	%------------Minimal Polynomial Header ---------------------
	\node[fancytitle, right=10pt] at (box.north west) {Minimal Polynomial};
\end{tikzpicture}

%------------ Pritmitive polynomials and Gauss' lemma ---------------
\begin{tikzpicture}
	\node [mybox] (box){%
		\begin{minipage}{0.3\textwidth}
			\textbf{Primitive polynomial: }A polynomial $ P \in \Z[X] $ is called primitive if if has a positive degree and the gcd of its coefficients is 1.\\
			\textbf{Gauss' lemma: }A polynomial $ P\in \Z[X] $ is irreducible over $ \Z[X] \iff $ it is primitive and irreducible over $ \Q[x] $
		\end{minipage}
	};
	%------------ Pritmitive polynomials and Gauss' lemma Header ---------------------
	\node[fancytitle, right=10pt] at (box.north west) {Pritmitive polynomials and Gauss' lemma};
\end{tikzpicture}

%------------ Eisenstein criterion for irreducibility ---------------
\begin{tikzpicture}
	\node [mybox] (box){%
		\begin{minipage}{0.3\textwidth}
			A polynomial $f(x)=a_nx^n+a_{n-1}x^{n-1}+\dots+ a_0\in \Z[x] $ is irreducible if $\exists  p $ prime s.t. $ p $ divides all coefficients except $ a_n $ and $ p^2 $ does not divide $ a_0 $.
		\end{minipage}
	};
	%------------ Eisenstein criterion for irreducibility Header ---------------------
	\node[fancytitle, right=10pt] at (box.north west) {Eisenstein criterion for irreducibility};
\end{tikzpicture}	

%------------ Finite extensions ---------------
\begin{tikzpicture}
	\node [mybox] (box){%
		\begin{minipage}{0.3\textwidth}
		For a field extension $ K \subset L $. $ L $ is called a \textbf{finite extension} of $ K $ if the vector space of $ L $ over $ K $ has a finite dimension.\\
		\textbf{Degree of finite extension:} Denoted as $ [L:K] = \dim_K	L$
		\begin{itemize}
			\item $ K \subset L \subset M $. Then $ M $ is finite over $ K \iff M $ is finite over L and $ L $ is finite over $ K $. Also in this case, $ [M:K]=[M:L][L:K] $.
			\item Let $K(\alpha_1, \dots, \alpha_n) \subset L $ denote the smallest subfield of $ L $ containing $ K $ and $ \alpha_i \in L $. This $ K(\alpha_1, \dots, \alpha_n) $ is generated by $ \alpha_1, \dots, \alpha_n. $
			\item $ L $ is finite over $ K \iff L$ is generated by a finite number of algebraic elements over $ K $ .
			\item $[K(\alpha):K]=\deg P_{\min} (\alpha, K)$
		\end{itemize}
		\end{minipage}
	};
	%------------ Finite extensions Header ---------------------
	\node[fancytitle, right=10pt] at (box.north west) {Finite extensions};
\end{tikzpicture}	




%------------ Stem field ---------------
\begin{tikzpicture}
	\node [mybox] (box){%
		\begin{minipage}{0.3\textwidth}
			Let $ P \in K[X] $ be an irreducible monic polynomial. A field extension $ E $ is called a stem field of $ P $ if $ \exists \alpha \in E$, s.t. $ \alpha $ is a root of $ P $ and $ E=K[\alpha] $
			\begin{itemize}
				\item If $ E, E'$ are two stem fields for $ P \in K[x] $, s.t. $ E=K[\alpha], E'=K[\alpha']$ where $ \alpha, \alpha'  $ are roots of $ P $. Then $ \exists ! $ isomorphism $ E \cong E' $ of K-algebras which maps $ \alpha $ to $ \alpha' $.
				\item If a stem field contains two roots of $ P $, then $ \exists ! $ automorphism that maps one root to another.
				\item If $ E $ is a stem field, $ [E:K] =\deg P$
				\item If $ [E:K] = \deg P$ and $ E $ contains a root of $ P $ then $ E $ is a stem field.
			\end{itemize}
		Some irreducibility criteria,
		\begin{itemize}
			\item $ P \in K[X] $ is irreducible over $ K \iff $ it does not have roots in $ L/K $ of degree $ \leq \deg P/ 2 $. 
			\item $ P \in K[X] $ is irreducible over $K$ with $ \deg P=n $. If $ L/K $ with $ [L:K]=m $ if $ \gcd (m,n)=1 $ then $ P $ is irreducible over $ L $.
		\end{itemize}
		\end{minipage}
	};
	%------------ Stem field Header ---------------------
	\node[fancytitle, right=10pt] at (box.north west) {Stem field};
\end{tikzpicture}	


%------------ Splitting field ---------------
\begin{tikzpicture}
	\node [mybox] (box){%
		\begin{minipage}{0.3\textwidth}
			Let $ P \in K[X] $. The splitting field of $ P $ over $ K $ is an extension of $ L $ where $ P $ is split into linear factors and the roots of $ P $ generate $ L $ (alternatively if $ P $ cannot be factored into any intermediate field).
			\begin{itemize}
				\item Splitting field $ L $ exists and its degree is $ \leq d! $, where $ d=\deg P $. And it is unique up to isomorphism.
			\end{itemize}
		\end{minipage}
	};
	%------------ Splitting field Header ---------------------
	\node[fancytitle, right=10pt] at (box.north west) {Splitting field};
\end{tikzpicture}


%------------ Algebraic closure ---------------
\begin{tikzpicture}
	\node [mybox] (box){%
		\begin{minipage}{0.3\textwidth}
			\begin{itemize}
				\item A field $ K $ is algebraically closed if any non-constant polynomial $ P \in K[X] $ has a root in $ K $.
				\item $ L $ is called an \textbf{algebraic closure} of $ K $ if it is algebraically closed and a field extension over $ K $.
				\item Every field has an algebraic closure.
				\item Algebraic closures of $ K $ are unique up to isomorphism as $ K- $algebras.
			\end{itemize}
			
		\end{minipage}
	};
	%------------ Algebraic closure Header ---------------------
	\node[fancytitle, right=10pt] at (box.north west) {Algebraic closure};
\end{tikzpicture}

%------------ Properties of finite fields ---------------
\begin{tikzpicture}
	\node [mybox] (box){%
		\begin{minipage}{0.3\textwidth}
			Let $p$ be a prime integer and let $q=p^r$ for some positive integer $r$. Then the following statements hold.
			\begin{itemize}[noitemsep,topsep=0pt]
				\item There exists a field of order $q.$
				\item Any two fields of order $q$ are isomorphic.
				\item Let $K$ be a field of order $q$. The multiplicative group $K^\times$ of non-zero elements of $K$ is a cyclic group of order $q-1$.
				\item Let $K$ be a field of order $q$. The elements of $K$ are the roots of $x^q-x \in \F_p[x]$.
				\item A field of order $p^r$ contains a field of order $p^k \iff k|r$
				\item The irreducible factors of $x^q-x$ over $\F_p$ are the irreducible polynomials in $\F_p[x]$ whose degree divides $r$.
				\item The splitting field of $ x^q-x $ has $ q $ elements.
				\item $  \F_q$ is a stem field and a splitting field of any irreducible polynomial $ P \in \F_p  $ of degree $ n $.
			\end{itemize}
		\end{minipage}
	};
	%------------ Properties of finite fields Header ---------------------
	\node[fancytitle, right=10pt] at (box.north west) {Properties of finite fields};
\end{tikzpicture}

%------------ Frobenius homomorphism ---------------
\begin{tikzpicture}
	\node [mybox] (box){%
		\begin{minipage}{0.3\textwidth}
			Let $ K $ be a field, $ \text{ch}(K)=p>0 $. There exists a homomorphism $ \varphi: K \rightarrow K $, s.t. $ \varphi(x)=x^p$. This is Frobenius homomorphism.
			\begin{itemize}
				\item The group of automorphisms over $ \F_{p^r} $ over $ \F_p $ is cyclic and is generated by the Frobenius map.
			\end{itemize}
		\end{minipage}
	};			
	%------------ Frobenius homomorphism Header ---------------------
	\node[fancytitle, right=10pt] at (box.north west) {Frobenius homomorphism};
\end{tikzpicture}

%------------ Separability ---------------
\begin{tikzpicture}
	\node [mybox] (box){%
		\begin{minipage}{0.3\textwidth}
			\begin{itemize}
				\item \textbf{Separable polynomial: }A irreducible polynomial $ P \in K[X] $ is called separable if $ \gcd(P, P')=1. $
				\item \textbf{Degree of separability: } $ \deg_{\text{sep}} P=\deg Q $ for some $ P(X)=Q(X^p) $
				\item \textbf{Degree of inseparability: }$ \deg_i P=\frac{\deg P}{\deg Q} $
				\item \textbf{Purely inseparable polynomial: }$ P $ is purely inseparable if $ \deg_i P = \deg P  $. Also if $ P $ is purely inseparable $ P=X^{p^r}-a$
				\item \textbf{Separable element: }If $ L/K $ is an algebraic extension, then $ \alpha \in L $ is called separable if its minimal polynomial over $ K $ is separable. And vice versa.
				\item If $ \alpha \in K $ is separable then $ |\text{Hom}(K(\alpha), \overline{K}|=\deg P_{\min} (\alpha, K)) $
				\item \textbf{Separable degree: } For $ L/K $, we have $ [	L:K]_{\text{sep}}=|\text{Hom}_K (K(\alpha),\overline{K}	)| $. Inseparable degree is degree of extension divided by separable degree.
				\item \textbf{Separable extension: }$ L $ is separable over $ K $ if $ [L:K]_{\text{sep}}=[L:K]$.
				\begin{itemize}
					\item If $ \text{ch} (K) =0$ then any extension of $ K $ is separable.
					\item If $ \text{ch} (K)=p$ then pure inseparable extension has degree $ p^r $ with degree of inseparability $ p^r $
				\end{itemize}
				\item Separable degrees obey the multiplicative property.
				\item TFAE
				\begin{itemize}
					\item $ L $ is separable over $ K $
					\item Any element of $ L $ is separable over $ K $
					\item $L = K (\alpha_1, \alpha_2, \dots, \alpha_n)$, where each $ \alpha_i  $ is separable over $ K$.
					\item $ L=K(\alpha_1, \alpha_2, \dots, \alpha_n) $, then $ \alpha_i $ is separable over $ K(\alpha_1, \dots, \alpha_{i-1}) $.
				\end{itemize}
				\item \textbf{Separable closure: }$ L^{\text{sep}}=\{x \mid x \text{ separable over } K\}$
			\end{itemize}
		\end{minipage}
	};			
	%------------ Separability Header ---------------------
	\node[fancytitle, right=10pt] at (box.north west) {Separability};
\end{tikzpicture}


%------------ Multilinear map ---------------
\begin{tikzpicture}
	\node [mybox] (box){%
		\begin{minipage}{0.3\textwidth}
			For a module $ M $ over ring $ A $. A function $ L $ from $ M^r =\underbrace{M \times M \times \cdots \times M}_{r\text{ times}}$ into $ A $ is called multilinear if $ L(\alpha_1, \dots, \alpha_r) $ is linear as a function of each $ \alpha_i $ when the other $ \alpha_j $ are fixed.
		\end{minipage}
	};			
	%------------ Multilinear map Header ---------------------
	\node[fancytitle, right=10pt] at (box.north west) {Multilinear map };
\end{tikzpicture}

%------------ Tensor product ---------------
\begin{tikzpicture}
	\node [mybox] (box){%
		\begin{minipage}{0.3\textwidth}
			Consider a ring $ A $ and two $ A- $modules, $M,N$. The tensor product is denoted as $ M \otimes_A N$ is an $ A- $module along with a $ A- $bilinear map, \\$ \varphi:M\times N \rightarrow 	M \otimes_A N $ which a ``universal property''.\\
			\textbf{Universal property of tensor product:}\\
			For a $ A- $module $ P $, if for an $A-$bilinear map, $ f: M \times N \rightarrow P $, then $  \exists ! $ homomorphism $\tilde{f}$ of $A- $modules s.t. $ f=\tilde{f} \circ \varphi$
			\[\begin{tikzcd}
				{M \times N} && {M \otimes_A N} \\
				\\
				&& P
				\arrow["{\tilde{f}}", dashed, from=1-3, to=3-3]
				\arrow["\varphi", from=1-1, to=1-3]
				\arrow["f"', from=1-1, to=3-3]
			\end{tikzcd}\]
			\begin{itemize}
				\item Commutativity of tensor product $ M \otimes_A N \cong N \otimes_A M $
				\item $A \otimes_A M \cong M$
				\item The basis for the tensor product of free modules is the tensor product of their individual basis elements.
				\item The tensor product is associative.
			\end{itemize}
		\textbf{Base change theorem:} For a ring $ A $, $ B $ an $ A- $algebra, $ M $ an $ A- $module and $N $ a $ B- $module. Then we have the following bijection
		\[ \text{Hom}_A(M,N) \leftrightarrow \text{Hom}_B(B\otimes_A M,N)\]
		\begin{itemize}
			\item For $ I $ an ideal of a ring $ A$ and $ M $ an $ A- $module we have, $ A/I \otimes_A M \cong M/IM $
		\end{itemize}
		\end{minipage}
	};			
	%------------ Tensor product Header ---------------------
	\node[fancytitle, right=10pt] at (box.north west) {Tensor product};
\end{tikzpicture}


%------------ Chinese remainder theorem ideals ---------------
\begin{tikzpicture}
	\node [mybox] (box){%
		\begin{minipage}{0.3\textwidth}
			\textbf{Comaximal ideals: }Two ideals of a ring are called comaximal (or coprime) if their sum gives the ring itself.
			\begin{itemize}
				\item If $ I,J $ are comaximal then $ IJ=I\bigcap J $
				\item If $ I_1,\dots,I_k $ comaximal w.r.t $J $ then $ \prod_{i=1}^k I_i $ is also relatively prime with $ J $.
				\item If $ I,J $ are comaximal then so are $ I^m, J^n $ for any $ m,n $.
			\end{itemize}
			\textbf{Chinese remainder theorem: } 
			For a ring $ A $, consider two comaximal ideals $ I,J $, then $ \forall a,b \in R, \exists x \in A $ s.t. $ x \equiv a (\text{mod } I)$ and $ x \equiv b (\text{mod } J) $\\
			\textbf{Generalized Chinese remainder theorem: } For a ring $ A $, let $ I_1, \dots, I_n $ be ideals of the ring $ A $. Consider the map $ \pi: A \rightarrow A/I_1 \times \cdots \times A/I_n $ defined as $ \pi(a)=(a\mod I_1, \dots, a \mod I_n) $. Then $ \ker \pi = I_1 \bigcap \cdots \bigcap I_n $, i.e. it is surjective iff $ I_1,\cdots I_n $ are pairwise comaximal. If $ \pi $ is a surjection we have, \[ A/\bigcap I_k \cong A/\prod I_k \cong \prod (A/I_k)\]
		\end{minipage}
	};			
	%------------ Chinese remainder theorem Header ---------------------
	\node[fancytitle, right=10pt] at (box.north west) {Chinese remainder theorem};
\end{tikzpicture}


%------------ Structure of finite algebras ---------------
\begin{tikzpicture}
	\node [mybox] (box){%
		\begin{minipage}{0.3\textwidth}
				Let $ A $ be a finite $ K- $algebra then,
				\begin{itemize}
					\item There are only finitely many maximal ideals in $ A $.
					\item For finitely many maximal ideals $ m_i $. Let $ J=m_1\bigcap \cdots \bigcap m_r.$ Then $ J^n=0 $ for some $ n $.
					\item $ A \cong A/m_1^{n_1} \times \cdots \times A/m_r^{n_r}$ for some (not necessarily unique) $ n_1, \dots, n_r.$
				\end{itemize}	
			\textbf{Reduced \textit{K}-Algebra: }If it has no nilpotent elements.\\
			\textbf{Local ring: }If it has only one maximal ideal. A non zero ring in which every element is either a unit or nilpotent is local.
		\end{minipage}
	};			
	%------------ Structure of finite algebras Header ---------------------
	\node[fancytitle, right=10pt] at (box.north west) {Structure of finite algebras};
\end{tikzpicture}


%------------ Further results on separability  ---------------
\begin{tikzpicture}
	\node [mybox] (box){%
		\begin{minipage}{0.3\textwidth}
			Let $ L $ be a finite extension over $  K $ then the following hold,
			\begin{itemize}
				\item $ L $ is separable $ \iff  L \otimes_K \overline{K} $ is reduced.
				\item $ L $ is purely inseparable $ \iff  L \otimes_K \overline{K} $ is local.
				\item $ L $ is separable $ \iff  \forall $ algebraic extensions $ \Omega, L \otimes_K \Omega $ is reduced.
				\item $ L $ is purely separable $ \iff  \forall $ algebraic extensions $ \Omega, L \otimes_K \Omega $ is local.
				\item If $ L $ is separable then the map $ \varphi: L \otimes_K \overline{K} \rightarrow \overline{K}^n $ defined as $ \varphi(l \otimes k) = (k \varphi_1(l),\dots, k \varphi_n(l))$ (where $\varphi_i  $ are distinct homomorphisms from $ L $ to $ \overline{K} $), is an isomorphism.
				\item Let $ L $ be a finite separable extension of $ K $ then it has only finitely many sub extensions.	
			\end{itemize}
		\end{minipage}
	};			
	%------------ Further results on separability Header ---------------------
	\node[fancytitle, right=10pt] at (box.north west) {Further results on separability};
\end{tikzpicture}

%------------ Primitive element theorem  ---------------
\begin{tikzpicture}
	\node [mybox] (box){%
		\begin{minipage}{0.3\textwidth}
			There exists $ \alpha \in L  $ s.t. $ L=K(\alpha) $ whenever $ L $ is finite and separable.
		\end{minipage}
	};			
%------------ Primitive element theorem Header ---------------------
	\node[fancytitle, right=10pt] at (box.north west) {Primitive element theorem};
\end{tikzpicture}


%------------ Normal extensions  ---------------
\begin{tikzpicture}
	\node [mybox] (box){%
		\begin{minipage}{0.3\textwidth}
			A normal extension of $ K $ is a splitting field of a family of polynomials in $ K[X] $.\\
			TFAE for an extension $ L $ of $ K $,
			\begin{itemize}
				\item $ \forall x \in L, P_{\min} (x,K) $ splits in $ L $.
				\item $ L $ is a normal extension.
				\item All homomorphisms from $ L $ to $ \overline{K} $ have the same image.
				\item The group of automorphisms, $ \text{Aut}(L/K) $ acts transitively on $ \text{Hom}_K(L,\overline{K}) $.
			\end{itemize}
		\end{minipage}
	};			
	%------------ Normal extensions Header ---------------------
	\node[fancytitle, right=10pt] at (box.north west) {Normal extensions};
f\end{tikzpicture}

%------------ Galois extensions ---------------
\begin{tikzpicture}
	\node [mybox] (box){%
		\begin{minipage}{0.3\textwidth}
			
			A field extension that is both normal and separable is called a Galois extension.
			\begin{itemize}
				\item For a finite extension $ L $ over $ K $ the number of automorphisms $ |\text{Aut}(L/K)| \leq [L:K]$. Equality holds iff $ L $ is a Galois` extension. 
			\end{itemize}

		If $ L $ is normal over $ K $ then,
		\begin{itemize}
			\item Isomorphism of sub extensions extend to automorphisms of $ L $.
			\item $ \text{Aut} (L/K)$ acts transitively on the roots of any irreducible polynomial in $ K[X] $.
			\item If $ \text{Aut} (L/K)$ fixes $ x \notin K$. Then $ x $ is purely inseparable.
		\end{itemize}
	
		\end{minipage}
	};			
%------------ Galois extensions Header ---------------------
	\node[fancytitle, right=10pt] at (box.north west) {Galois extensions};
	f\end{tikzpicture}

%------------ Galois group ---------------
\begin{tikzpicture}
	\node [mybox] (box){%
		\begin{minipage}{0.3\textwidth}
		If $ L $ is a Galois extension, $ G=\text{Gal}(L/K) = \text{Aut}(L/K)$ is called the Galois group of the extension.
		\begin{itemize}
			\item $ L^{\text{Gal}(L/K) } =K $, (i.e. the set of invariants in $ L $ with the action of the Galois group is equal to $ K $).
			\item Let $ L $ be a field and $ G $ a subgroup of $ \text{Aut} (L)$, then
			\begin{itemize}
				\item If all orbits of $ G $ are finite, then $ L $ is a Galois extension of $ L^G $.
				\item If order of $ G $ is finite then, $ [L,L^G] =n$ and $ G $ is a Galois group.
			\end{itemize}
		\end{itemize}
	
		\end{minipage}
	};			
%------------ Galois group Header ---------------------
	\node[fancytitle, right=10pt] at (box.north west) {Galois groups};
\end{tikzpicture}



% ==============================================================
\end{multicols*}
\end{document}
